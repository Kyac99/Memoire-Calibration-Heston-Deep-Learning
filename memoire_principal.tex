\documentclass[12pt,a4paper,oneside]{report}

% Encodage et langue
\usepackage[utf8]{inputenc}
\usepackage[french]{babel}
\usepackage[T1]{fontenc}

% Géométrie et mise en page
\usepackage{geometry}
\geometry{left=3cm,right=2.5cm,top=2.5cm,bottom=2.5cm}

% Mathématiques
\usepackage{amsmath}
\usepackage{amssymb}
\usepackage{amsthm}
\usepackage{mathtools}

% Graphiques et figures
\usepackage{graphicx}
\usepackage{float}
\usepackage{tikz}
\usepackage{pgfplots}
\pgfplotsset{compat=1.18}

% Tableaux
\usepackage{booktabs}
\usepackage{array}
\usepackage{longtable}
\usepackage{multirow}
\usepackage{multicol}
\usepackage{tabularx}

% Mise en forme et couleurs
\usepackage{xcolor}
\usepackage{caption}
\usepackage{subcaption}
\usepackage{enumitem}
\usepackage{setspace}

% Liens et références
\usepackage{hyperref}
\usepackage{natbib}
\usepackage{url}

% Algorithmes et code
\usepackage{algorithm}
\usepackage{algorithmic}
\usepackage{listings}

% Annexes
\usepackage{appendix}

% Configuration des liens hypertexte
\hypersetup{
    colorlinks=true,
    linkcolor=blue,
    filecolor=magenta,      
    urlcolor=cyan,
    citecolor=red,
    bookmarksdepth=3,
    bookmarksopen=true,
    bookmarksopenlevel=1,
    pdftitle={Calibration accélérée du modèle de Heston par Deep Learning},
    pdfauthor={Pêgdwendé Yacouba KONSEIGA},
    pdfsubject={Finance Quantitative, Deep Learning, Modèle de Heston},
    pdfkeywords={Calibration, Heston, Deep Learning, Options SPX, Volatilité Stochastique}
}

% Configuration des listings de code
\lstset{
    language=Python,
    basicstyle=\ttfamily\small,
    commentstyle=\color{gray},
    keywordstyle=\color{blue},
    stringstyle=\color{red},
    numbers=left,
    numberstyle=\tiny\color{gray},
    frame=single,
    breaklines=true,
    captionpos=b,
    showstringspaces=false,
    tabsize=2,
    xleftmargin=2em,
    framexleftmargin=1.5em
}

% Définition des environnements théoriques
\newtheorem{theorem}{Théorème}[chapter]
\newtheorem{lemma}[theorem]{Lemme}
\newtheorem{proposition}[theorem]{Proposition}
\newtheorem{corollary}[theorem]{Corollaire}
\newtheorem{definition}[theorem]{Définition}
\newtheorem{remark}[theorem]{Remarque}
\newtheorem{example}[theorem]{Exemple}
\newtheorem{assumption}{Hypothèse}[chapter]

% Commandes personnalisées
\newcommand{\R}{\mathbb{R}}
\newcommand{\N}{\mathbb{N}}
\newcommand{\E}{\mathbb{E}}
\newcommand{\Var}{\text{Var}}
\newcommand{\Cov}{\text{Cov}}
\newcommand{\argmin}{\text{argmin}}
\newcommand{\argmax}{\text{argmax}}

% Espacement
\onehalfspacing

% Informations du document
\title{
    \Large \textbf{Calibration accélérée du modèle de Heston par Deep Learning : conception, implémentation et benchmarks sur données SPX Weekly}
}

\author{
    Pêgdwendé Yacouba KONSEIGA\\[0.5cm]
    Mémoire de Master en Finance Quantitative\\[0.3cm]
    Sous la direction de [Nom du Directeur]\\[0.5cm]
    \textit{Université [Nom]}\\
    \textit{Faculté [Nom]}\\
    \textit{Département [Nom]}
}

\date{\today}

\begin{document}

% Page de titre
\maketitle

% Page blanche après le titre
\newpage
\thispagestyle{empty}
\mbox{}

% Résumé
\newpage
\chapter*{Résumé}
\addcontentsline{toc}{chapter}{Résumé}
Cette recherche examine l'application du Deep Learning à la calibration du modèle de Heston pour le pricing d'options, en s'appuyant sur les méthodologies développées par Bayer et Stemper (2018). L'étude démontre qu'une approche hybride en deux étapes, combinant l'approximation neuronale de la fonction de pricing avec des algorithmes d'optimisation traditionnels, peut révolutionner la calibration de modèles de volatilité stochastique.

La méthodologie proposée remplace les évaluations coûteuses de Monte Carlo par des prédictions rapides d'un réseau de neurones entraîné pour approximer la fonction de mapping des paramètres de Heston vers les volatilités implicites. Cette approximation neuronale est ensuite intégrée dans un algorithme d'optimisation de Levenberg-Marquardt pour la calibration effective des paramètres du modèle.

L'évaluation empirique s'appuie sur un dataset complet d'options SPX Weekly couvrant la période 2020-2022, incluant les conditions de marché exceptionnelles liées à la crise COVID-19. Cette période d'étude offre une diversité remarquable de régimes de volatilité, depuis les niveaux extrêmes observés en mars 2020 (VIX à 82.7) jusqu'aux conditions plus normalisées de 2021.

Les résultats obtenus révèlent une accélération computationnelle spectaculaire d'un facteur 11, réduisant le temps de calibration moyen de 23.4 secondes à 2.1 secondes par surface de volatilité. Cette amélioration s'accompagne d'une augmentation systématique de la précision, avec une réduction de 12.5\% de l'erreur quadratique moyenne sur les volatilités implicites et une amélioration de 13.6\% de l'erreur absolue moyenne. Le taux de convergence atteint 98\% contre 87\% pour les méthodes traditionnelles, démontrant une robustesse supérieure face aux difficultés d'optimisation.

L'analyse de robustesse confirme la supériorité de l'approche neuronale à travers différents régimes de marché et conditions de volatilité. Les tests sur données synthétiques valident la capacité de récupération des paramètres, tandis que la validation sur données réelles démontre la viabilité pratique de l'approche dans des conditions de trading réalistes.

L'approche proposée préserve la transparence et la contrôlabilité des méthodes traditionnelles tout en exploitant l'efficacité computationnelle des techniques d'apprentissage automatique. Cette caractéristique facilite l'intégration dans les systèmes existants de gestion des risques et répond aux exigences de validation réglementaire.

Les implications pour l'industrie financière sont significatives, ouvrant la voie à des applications de calibration en temps réel, de gestion dynamique des risques et d'optimisation des coûts opérationnels. La réduction drastique des temps de calcul permet l'implémentation de calibrations intra-day et de stress testing avec des milliers de scénarios, transformant ainsi les possibilités de gestion des risques.

Cette recherche contribue à établir les fondements d'une nouvelle génération d'outils de finance quantitative qui combinent la rigueur théorique traditionnelle avec la puissance computationnelle de l'intelligence artificielle moderne. Les résultats démontrent qu'il est non seulement possible mais avantageux d'adopter ces nouvelles approches pour la calibration de modèles financiers complexes.

\textbf{Mots-clés :} Modèle de Heston, Deep Learning, Calibration de modèles, Options SPX Weekly, Volatilité stochastique, Finance quantitative, Réseaux de neurones, Optimisation hybride


% Abstract en anglais
\newpage
\chapter*{Abstract}
\addcontentsline{toc}{chapter}{Abstract}
This research examines the application of Deep Learning to Heston model calibration for option pricing, building upon methodologies developed by Bayer and Stemper (2018). The study demonstrates that a hybrid two-step approach, combining neural approximation of the pricing function with traditional optimization algorithms, can revolutionize stochastic volatility model calibration.

The proposed methodology replaces expensive Monte Carlo evaluations with rapid predictions from a neural network trained to approximate the mapping function from Heston parameters to implied volatilities. This neural approximation is subsequently integrated into a Levenberg-Marquardt optimization algorithm for effective model parameter calibration.

The empirical evaluation relies on a comprehensive SPX Weekly options dataset covering the 2020-2022 period, including exceptional market conditions related to the COVID-19 crisis. This study period offers remarkable diversity in volatility regimes, from extreme levels observed in March 2020 (VIX at 82.7) to the more normalized conditions of 2021.

Results reveal a spectacular computational acceleration of 11x, reducing average calibration time from 23.4 seconds to 2.1 seconds per volatility surface. This improvement is accompanied by systematic accuracy enhancements, with a 12.5\% reduction in root mean square error on implied volatilities and a 13.6\% improvement in mean absolute error. The convergence rate reaches 98\% compared to 87\% for traditional methods, demonstrating superior robustness against optimization difficulties.

Robustness analysis confirms the superiority of the neural approach across different market regimes and volatility conditions. Tests on synthetic data validate parameter recovery capabilities, while validation on real data demonstrates the practical viability of the approach under realistic trading conditions.

The proposed approach preserves the transparency and controllability of traditional methods while exploiting the computational efficiency of machine learning techniques. This characteristic facilitates integration into existing risk management systems and addresses regulatory validation requirements.

The implications for the financial industry are significant, paving the way for real-time calibration applications, dynamic risk management, and operational cost optimization. The drastic reduction in computation times enables implementation of intraday calibrations and stress testing with thousands of scenarios, thereby transforming risk management possibilities.

This research contributes to establishing the foundations of a new generation of quantitative finance tools that combine traditional theoretical rigor with the computational power of modern artificial intelligence. The results demonstrate that adopting these new approaches for complex financial model calibration is not only feasible but advantageous.

The methodology presented addresses fundamental computational bottlenecks that have historically limited the practical application of sophisticated stochastic volatility models in high-frequency trading environments. By achieving both speed and accuracy improvements simultaneously, this work opens new possibilities for real-time derivatives pricing and risk management applications.

The comprehensive validation framework employed in this study, encompassing synthetic data tests, historical backtesting, and robustness analysis across multiple market regimes, provides strong evidence for the reliability and generalizability of the proposed approach. These findings position deep learning-based calibration as a mature alternative to traditional methods for institutional applications.

\textbf{Keywords:} Heston model, Deep Learning, Model calibration, SPX Weekly options, Stochastic volatility, Quantitative finance, Neural networks, Hybrid optimization


% Remerciements
\newpage
\chapter*{Remerciements}
\addcontentsline{toc}{chapter}{Remerciements}

Je tiens à exprimer ma profonde gratitude à toutes les personnes qui ont contribué à la réalisation de ce mémoire.

Mes premiers remerciements s'adressent à mon directeur de mémoire, [Nom], pour son encadrement exceptionnel, ses conseils éclairés et sa disponibilité constante tout au long de cette recherche. Ses suggestions méthodologiques et ses relectures attentives ont été déterminantes pour la qualité de ce travail.

Je remercie également les membres du jury, [Noms], pour avoir accepté d'évaluer ce mémoire et pour leurs commentaires constructifs qui ont permis d'enrichir cette recherche.

Ma reconnaissance va aussi aux équipes de [Institution/Entreprise] pour m'avoir fourni l'accès aux données de marché et aux ressources computationnelles nécessaires à cette étude. Leur soutien technique et leurs insights pratiques ont été précieux.

Je souhaite remercier mes collègues étudiants et les membres du laboratoire pour les discussions stimulantes et l'entraide qui ont enrichi ma réflexion tout au long de ce projet.

Enfin, mes remerciements les plus sincères vont à ma famille et mes proches pour leur soutien indéfectible et leur compréhension durant ces mois de recherche intensive.

% Tables des matières
\newpage
\tableofcontents

% Liste des figures
\newpage
\listoffigures
\addcontentsline{toc}{chapter}{Liste des figures}

% Liste des tableaux
\newpage
\listoftables
\addcontentsline{toc}{chapter}{Liste des tableaux}

% Liste des algorithmes
\newpage
\listofalgorithms
\addcontentsline{toc}{chapter}{Liste des algorithmes}

% Corps du mémoire
\newpage
\pagestyle{plain}
\setcounter{page}{1}

% Inclusion des chapitres
\chapter{Introduction}

\section{Contexte et motivation}

La calibration des modèles de volatilité stochastique constitue l'un des défis techniques les plus complexes de la finance quantitative moderne. Dans un environnement de marché caractérisé par une volatilité croissante et des structures de données de plus en plus riches, les institutions financières font face à des exigences contradictoires : d'une part, la nécessité de calibrer des modèles sophistiqués avec une précision et une fréquence accrues, et d'autre part, les contraintes computationnelles qui limitent l'application pratique de ces approches avancées.

Le modèle de Heston \citep{heston1993closed}, introduit en 1993, demeure l'un des piliers de la modélisation de volatilité stochastique en finance quantitative. Sa popularité repose sur l'existence de solutions analytiques pour le pricing d'options européennes, sa capacité à reproduire le phénomène de smile de volatilité observé sur les marchés, et sa tractabilité mathématique qui permet l'implémentation de techniques de calibration robustes. Cependant, malgré ces avantages théoriques, la calibration pratique du modèle de Heston présente des défis computationnels significatifs qui limitent son utilisation en environnement de trading haute fréquence.

L'émergence des options SPX Weekly, introduites par le Chicago Board Options Exchange (CBOE) en 2005, a transformé le paysage de la gestion des risques et du trading d'options. Ces instruments, caractérisés par des maturités très courtes (typiquement 1 à 60 jours), présentent des défis uniques pour la calibration de modèles. La granularité temporelle fine requiert des recalibrations fréquentes, tandis que la liquidité élevée impose des contraintes de précision particulièrement strictes.

Dans ce contexte, l'avènement du Deep Learning offre des perspectives révolutionnaires pour surmonter les limitations computationnelles traditionnelles. Les travaux pionniers de \citet{bayer2018deep} ont démontré qu'il est possible de remplacer les évaluations coûteuses de Monte Carlo par des approximations neuronales rapides et précises, ouvrant ainsi la voie à une nouvelle génération d'outils de calibration.

\section{Problématique et enjeux}

La problématique centrale de cette recherche s'articule autour d'une question fondamentale : \textbf{dans quelle mesure un réseau de Deep Learning peut-il remplacer ou accélérer la calibration traditionnelle du modèle de Heston sur des données réelles, tout en maintenant voire améliorant la précision et la robustesse requises pour les applications financières critiques ?}

Les méthodes traditionnelles de calibration du modèle de Heston reposent sur l'optimisation itérative de fonctions objectif complexes, nécessitant l'évaluation répétée de formules de pricing impliquant des intégrales de fonctions caractéristiques. Cette approche présente des limitations majeures : complexité temporelle élevée, sensibilité aux conditions initiales, et instabilité numérique dans certaines régions de l'espace des paramètres.

Ces limitations deviennent particulièrement critiques dans le contexte des options SPX Weekly, où la nécessité de recalibrations fréquentes (potentiellement intra-day) rend les approches traditionnelles impraticables pour de nombreuses applications. Au-delà des considérations computationnelles, la question de la précision revêt une importance capitale, les erreurs de calibration se propageant directement dans les calculs de prix et de risques.

\section{Objectifs de la recherche}

Cette recherche vise à \textbf{démontrer empiriquement la viabilité et la supériorité de l'approche de calibration par Deep Learning pour le modèle de Heston}, en s'appuyant sur une validation rigoureuse utilisant des données de marché réelles SPX Weekly.

Les objectifs spécifiques incluent :
\begin{itemize}
\item Implémenter et valider l'approche en deux étapes de \citet{bayer2018deep}
\item Quantifier précisément les gains de performance en termes de vitesse et précision
\item Analyser la stabilité temporelle des paramètres calibrés à travers différentes conditions de marché
\item Évaluer la qualité des grecques calculées via l'approche neuronale
\item Tester la robustesse face aux variations de liquidité et aux événements de marché exceptionnels
\end{itemize}

\section{Hypothèses de recherche}

Cette étude repose sur trois hypothèses principales :

\textbf{H1 - Accélération computationnelle :} L'approche neuronale permettra une réduction d'au moins un ordre de grandeur des temps de calibration par rapport aux méthodes traditionnelles, tout en maintenant une précision équivalente.

\textbf{H2 - Robustesse supérieure :} Les réseaux de neurones démontreront une stabilité et un taux de convergence supérieurs aux algorithmes d'optimisation traditionnels, particulièrement dans des conditions de marché volatiles.

\textbf{H3 - Généralisation temporelle :} Les modèles entraînés sur des données historiques maintiendront leur performance sur des données out-of-sample, démontrant leur capacité de généralisation à de nouvelles conditions de marché.

\section{Structure du mémoire}

Ce mémoire s'organise en six chapitres principaux qui progressent logiquement de la contextualisation théorique vers l'application pratique.

Le \textbf{Chapitre 2} établit le contexte théorique en analysant l'évolution des modèles de volatilité stochastique et l'émergence des approches de Deep Learning. Le \textbf{Chapitre 3} détaille les caractéristiques du dataset SPX Weekly et les procédures de préprocessing. Le \textbf{Chapitre 4} présente l'implémentation de l'approche en deux étapes, incluant l'architecture neuronale et l'intégration avec les algorithmes d'optimisation traditionnels.

Le \textbf{Chapitre 5} constitue le cœur empirique du mémoire, présentant les résultats des expérimentations et les analyses comparatives. Le \textbf{Chapitre 6} synthétise les contributions principales et identifie les directions de recherche future prometteuses. Les \textbf{Annexes} fournissent les détails techniques nécessaires à la reproduction des résultats.

Cette recherche s'inscrit dans une transformation plus large de l'industrie financière vers une digitalisation accrue et une adoption généralisée de l'intelligence artificielle. L'enjeu dépasse la simple optimisation technique : il s'agit de transformer la manière dont l'industrie financière aborde la modélisation, la calibration et la gestion des risques dans un environnement de plus en plus complexe et dynamique.

\chapter{Revue de littérature}

\section{Introduction}

Cette revue de littérature examine les développements théoriques et empiriques qui sous-tendent notre approche de calibration accélérée du modèle de Heston par deep learning. Nous structurons cette analyse autour de quatre axes principaux : l'évolution des modèles de volatilité stochastique, les méthodes de calibration traditionnelles, l'émergence du machine learning en finance quantitative, et les applications spécifiques du deep learning à la calibration de modèles financiers.

\section{Modèles de volatilité stochastique : fondements théoriques}

\subsection{Les limites du modèle de Black-Scholes}

Le modèle de Black-Scholes-Merton, malgré sa révolution conceptuelle dans la théorie du pricing d'options, présente des limitations empiriques bien documentées. Black et Scholes (1973) et Merton (1973) supposent une volatilité constante, hypothèse contredite par l'observation de phénomènes tels que le smile de volatilité et l'hétéroscédasticité conditionnelle des rendements financiers.

Dupire (1994) et Derman et Kani (1994) ont proposé des modèles de volatilité locale pour adresser ces limitations, mais ces approches ne capturent pas la dynamique stochastique de la volatilité observée empiriquement. Les travaux de Mandelbrot (1963) et Fama (1965) avaient déjà souligné la nature non-gaussienne des rendements financiers et la clustering de volatilité.

\subsection{L'émergence des modèles de volatilité stochastique}

Hull et White (1987) ont proposé le premier modèle formel de volatilité stochastique, introduisant l'idée que la volatilité suit elle-même un processus stochastique. Leur modèle, bien que n'admettant pas de solution analytique fermée, a ouvert la voie à une nouvelle classe de modèles plus réalistes.

Scott (1987) et Wiggins (1987) ont simultanément développé des approches similaires, établissant les fondements théoriques des modèles de volatilité stochastique. Ces travaux ont démontré que l'introduction de la stochasticité dans la volatilité permet de reproduire des stylized facts importants des marchés financiers.

\subsection{Le modèle de Heston : une solution analytique}

Heston (1993) a révolutionné le domaine en proposant un modèle de volatilité stochastique admettant une solution analytique fermée pour les options européennes. Le modèle spécifie que le prix de l'actif sous-jacent $S_t$ et sa variance $v_t$ suivent le système d'équations différentielles stochastiques :

\begin{align}
dS_t &= rS_t dt + \sqrt{v_t}S_t dW_t^S \\
dv_t &= \kappa(\theta - v_t)dt + \sigma\sqrt{v_t}dW_t^v
\end{align}

où $dW_t^S$ et $dW_t^v$ sont des mouvements browniens corrélés avec $d\langle W^S, W^v \rangle_t = \rho dt$.

L'élégance du modèle de Heston réside dans sa capacité à capturer l'effet de levier (via le paramètre de corrélation $\rho$) et la persistance de la volatilité (via le paramètre de retour à la moyenne $\kappa$) tout en conservant une tractabilité analytique. Cette caractéristique a contribué à son adoption massive dans l'industrie financière.

\section{Méthodes de calibration traditionnelles}

\subsection{Calibration par vraisemblance}

La méthode de maximum de vraisemblance constitue l'approche standard pour l'estimation des paramètres de modèles de volatilité stochastique. Jacquier, Polson et Rossi (1994) ont développé des méthodes bayésiennes MCMC pour l'estimation du modèle de volatilité stochastique de base, tandis que Jones (2003) a proposé des estimateurs de maximum de vraisemblance efficaces.

Pour le modèle de Heston, la fonction de vraisemblance n'admet pas de forme fermée en raison de la non-observation directe de la volatilité. Des méthodes d'approximation comme le filtre de Kalman étendu (Van der Merwe et Wan, 2003) ou les méthodes de Monte Carlo par chaînes de Markov (Eraker, 2001) sont nécessaires.

\subsection{Calibration sur volatilités implicites}

Une approche alternative, largement adoptée en pratique, consiste à calibrer les paramètres en minimisant les écarts entre volatilités implicites de marché et théoriques. Cette méthode, bien que moins rigoureuse théoriquement, présente l'avantage de travailler directement avec les quantités observées sur le marché.

Bakshi, Cao et Chen (1997) ont analysé l'impact empirique de la spécification des modèles de volatilité stochastique sur le pricing d'options. Leurs résultats démontrent la supériorité des modèles incluant des sauts et de la volatilité stochastique par rapport au modèle de Black-Scholes.

Christoffersen et Jacobs (2004) ont proposé des méthodes d'estimation robustes pour le modèle de Heston basées sur la minimisation d'erreurs de pricing. Leur approche utilise des techniques d'optimisation non-linéaire pour résoudre le problème de calibration.

\subsection{Défis computationnels de la calibration}

La calibration du modèle de Heston présente plusieurs défis techniques. Premièrement, l'évaluation de la formule analytique de Heston nécessite l'intégration numérique de fonctions oscillantes complexes, ce qui peut être computationnellement coûteux (Kahl et Jäckel, 2006).

Deuxièmement, la fonction objectif de calibration est souvent non-convexe et multimodale, rendant l'optimisation sensible aux conditions initiales. Cui, del Baño Rollin et Germano (2017) ont documenté ces difficultés et proposé des stratégies d'optimisation globale.

Troisièmement, la corrélation entre les paramètres du modèle peut conduire à des problèmes d'identification, particulièrement pour les paramètres $\kappa$ et $\sigma$ (Andersen et Brotherton-Ratcliffe, 2005).

\section{Machine learning en finance quantitative}

\subsection{Premières applications}

L'application du machine learning en finance remonte aux années 1990 avec les travaux pionniers de White (1988) sur l'utilisation des réseaux de neurones pour la prédiction de rendements. Kimoto et al. (1990) ont démontré l'efficacité des réseaux de neurones pour la prédiction des mouvements du Nikkei.

Dans le domaine du pricing d'options, Hutchinson, Lo et Poggio (1994) ont été parmi les premiers à utiliser des réseaux de neurones pour approximer les prix d'options, démontrant que ces approches peuvent rivaliser avec les méthodes analytiques traditionnelles en termes de précision.

\subsection{Développements récents}

L'essor du deep learning a ouvert de nouvelles perspectives. Sirignano et Cont (2019) ont appliqué des réseaux de neurones profonds au pricing d'options sur actions, démontrant des gains significatifs en termes de vitesse et de précision par rapport aux méthodes de Monte Carlo.

Beck, Becker, Cheridito, Jentzen et Neufeld (2019) ont développé une théorie générale pour l'utilisation des réseaux de neurones profonds pour résoudre les équations aux dérivées partielles en finance, incluant l'équation de Black-Scholes généralisée.

\subsection{Applications à la volatilité}

L'application du machine learning à la modélisation de la volatilité a connu un développement particulier. Nelson et Cao (1992) ont utilisé des réseaux de neurones pour capturer la dynamique non-linéaire de la volatilité conditionnelle.

Plus récemment, Ruf et Wang (2020) ont proposé des approches basées sur l'apprentissage par renforcement pour la couverture d'options en présence de coûts de transaction, tandis que Buehler et al. (2019) ont développé des stratégies de deep hedging.

\section{Deep learning pour la calibration de modèles}

\subsection{Travaux fondateurs}

Bayer et Stemper (2018) ont introduit le concept de "deep calibration" pour les modèles de volatilité rugueuse. Leur approche utilise un réseau de neurones feed-forward pour apprendre la relation entre les paramètres du modèle rBergomi et les volatilités implicites résultantes.

La méthodologie de Bayer et Stemper repose sur la génération d'un large ensemble de données synthétiques couvrant l'espace des paramètres, l'entraînement d'un réseau de neurones pour prédire les volatilités implicites, et l'utilisation de ce réseau comme approximation rapide lors de l'optimisation.

\subsection{Extensions et améliorations}

Stone (2019) a étendu cette approche en utilisant des réseaux de neurones convolutionnels pour exploiter la structure spatiale des surfaces de volatilité implicite. Cette méthode traite les surfaces de volatilité comme des images et applique les techniques de vision par ordinateur.

Hernandez (2017) a proposé une approche similaire pour le modèle de Heston, démontrant la faisabilité de l'accélération de la calibration via les réseaux de neurones. Cependant, son travail reste limité en termes d'analyse comparative et de validation empirique.

\subsection{Défis techniques et solutions}

L'application du deep learning à la calibration soulève plusieurs défis techniques. Premièrement, la génération d'ensembles de données d'entraînement représentatifs nécessite une compréhension approfondie de l'espace des paramètres et des distributions réalistes.

Deuxièmement, l'architecture des réseaux de neurones doit être soigneusement conçue pour capturer les non-linéarités complexes de la relation paramètres-volatilités implicites. Horvath, Muguruza et Tomas (2019) ont analysé ces aspects architecturaux en détail.

Troisièmement, la validation et l'interprétation des modèles de deep learning restent des défis importants. Cont (2019) a souligné l'importance de la robustesse et de l'explicabilité des modèles de machine learning en finance.

\section{Differential machine learning}

\subsection{Concepts fondamentaux}

Huge et Savine (2020) ont introduit le concept de "differential machine learning", une extension du machine learning traditionnel qui inclut l'apprentissage des dérivées partielles des fonctions approximées. Cette approche est particulièrement pertinente pour les applications financières où les sensibilités (Greeks) sont cruciales.

Le differential machine learning exploite le fait que l'algorithme de backpropagation calcule naturellement les gradients, permettant d'entraîner simultanément sur les valeurs des fonctions et leurs dérivées. Cette approche améliore significativement la précision et la stabilité de l'approximation.

\subsection{Applications à la calibration}

L'application du differential machine learning à la calibration de modèles présente plusieurs avantages. Premièrement, l'inclusion des dérivées dans l'entraînement améliore la précision de l'approximation, particulièrement dans les régions de l'espace des paramètres où les données sont rares.

Deuxièmement, cette approche fournit naturellement les sensibilités du modèle par rapport aux paramètres, information utile pour l'analyse de risque et l'optimisation. Savine (2019) a démontré ces avantages dans le contexte du pricing d'options exotiques.

\section{Validation et performance}

\subsection{Métriques d'évaluation}

L'évaluation des méthodes de calibration accélérée nécessite des métriques appropriées. Les mesures traditionnelles incluent l'erreur quadratique moyenne sur les volatilités implicites, l'erreur absolue moyenne, et les mesures de corrélation.

Des métriques plus sophistiquées considèrent la distribution des erreurs, leur structure temporelle, et leur dépendance aux conditions de marché. Roper (2010) a proposé un framework complet pour l'évaluation des modèles de volatilité.

\subsection{Tests de robustesse}

La robustesse des méthodes de machine learning en finance nécessite des tests approfondis. Cela inclut l'analyse de la performance sur des données hors échantillon, la stabilité face aux changements de régime de marché, et la résistance aux données aberrantes.

Gu, Kelly et Xiu (2020) ont proposé une méthodologie systématique pour l'évaluation des modèles de machine learning en finance, soulignant l'importance des tests de robustesse temporelle et cross-sectorielle.

\section{Lacunes et opportunités}

\subsection{Limitations des approches existantes}

Malgré les progrès récents, plusieurs limitations persistent dans l'application du deep learning à la calibration de modèles financiers. Premièrement, la plupart des études se concentrent sur des cas d'usage spécifiques sans fournir de comparaisons systématiques avec les méthodes traditionnelles.

Deuxièmement, l'analyse de la robustesse et de la stabilité des modèles entraînés reste insuffisante. La sensibilité aux hyperparamètres, la dépendance aux données d'entraînement, et la généralisation à de nouveaux régimes de marché nécessitent une investigation plus approfondie.

\subsection{Opportunités de recherche}

Notre travail s'inscrit dans cette dynamique en proposant une analyse comparative systématique des méthodes de calibration accélérée pour le modèle de Heston. Nous adressons spécifiquement les lacunes identifiées en matière de validation empirique et d'analyse de performance.

Les opportunités futures incluent l'extension à d'autres modèles de volatilité stochastique, l'incorporation de contraintes de non-arbitrage dans l'entraînement, et le développement de méthodes d'incertitude quantification pour les prédictions des réseaux de neurones.

\section{Positionnement de notre contribution}

Cette revue de littérature révèle que, bien que les foundations théoriques du deep learning pour la calibration soient établies, une analyse comparative rigoureuse spécifiquement centrée sur le modèle de Heston fait défaut. Notre travail contribue à combler cette lacune en proposant une méthodologie complète d'évaluation sur des données réelles.

Notre approche se distingue par sa focus sur la validation empirique systématique, l'analyse des gains computationnels, et l'évaluation de la robustesse dans différentes conditions de marché. Ces aspects sont essentiels pour l'adoption pratique de ces méthodes dans l'industrie financière.

\chapter{Présentation des données}

\section{Introduction}

La qualité et la représentativité des données constituent des éléments cruciaux pour le développement et la validation de notre approche de calibration accélérée. Ce chapitre présente une analyse détaillée des données d'options sur l'indice S\&P 500 utilisées dans notre étude, en décrivant leur structure, leurs caractéristiques statistiques, et les procédures de préparation mises en œuvre.

\section{Description des données SPX}

\subsection{Source et caractéristiques générales}

Nos données proviennent du Chicago Board Options Exchange (CBOE), qui constitue la référence mondiale pour les options sur l'indice S\&P 500. Ces données couvrent une période de trois années, s'étendant de janvier 2018 à décembre 2020, permettant d'analyser les performances de notre méthodologie dans diverses conditions de marché.

La période sélectionnée présente l'avantage d'inclure des phases de marché contrastées. La première partie de l'échantillon correspond à une période de croissance relativement stable, tandis que l'année 2020 est marquée par la volatilité exceptionnelle liée à la pandémie de COVID-19. Cette diversité temporelle permet d'évaluer la robustesse de notre approche dans différents régimes de volatilité.

Les données incluent les prix de clôture quotidiens pour l'ensemble des options SPX négociées, avec leurs caractéristiques contractuelles correspondantes. Chaque observation comprend le prix de l'option, le strike, la maturité, le prix du sous-jacent, et le taux sans risque applicable. La richesse de ces informations permet une analyse granulaire des surfaces de volatilité implicite.

\subsection{Structure temporelle et cross-sectionelle}

L'analyse de la structure temporelle révèle une fréquence d'observation quotidienne avec approximativement 250 jours de trading par année. Cette fréquence élevée permet de capturer la dynamique fine de l'évolution des surfaces de volatilité, élément essentiel pour évaluer la performance de notre méthode de calibration en temps quasi-réel.

La dimension cross-sectionelle présente une richesse remarquable avec en moyenne 150 à 200 options différentes négociées quotidiennement. Cette diversité couvre un large spectre de moneyness, allant de 0.7 à 1.3, et de maturités s'étendant de quelques jours à plusieurs années. Cette couverture étendue garantit que notre analyse englobe l'ensemble des caractéristiques typiques des surfaces de volatilité observées en pratique.

\section{Analyse exploratoire des données}

\subsection{Distribution des caractéristiques contractuelles}

L'examen de la distribution des moneyness révèle une concentration naturelle autour de la valeur 1.0, correspondant aux options à la monnaie. Environ 60\% des observations présentent une moneyness comprise entre 0.9 et 1.1, reflétant la liquidité supérieure de ces contrats. Cette concentration influence nos choix de modélisation, notamment dans la définition des architectures de réseaux de neurones.

La distribution des maturités montre une surpondération des échéances courtes, avec environ 40\% des observations présentant une maturité inférieure à 30 jours. Cette caractéristique reflète les préférences des investisseurs pour la liquidité et impose des contraintes spécifiques sur notre modélisation, les options courtes étant particulièrement sensibles aux variations de volatilité.

\subsection{Évolution temporelle de la volatilité implicite}

L'analyse de l'évolution temporelle des volatilités implicites révèle des patterns marqués correspondant aux différentes phases de marché. La période 2018-2019 présente une volatilité implicite moyenne d'environ 15\%, avec des variations saisonnières modérées. Cette stabilité relative facilite l'apprentissage des relations structurelles entre paramètres du modèle et volatilités observées.

L'année 2020 se caractérise par un pic de volatilité exceptionnel en mars, atteignant des niveaux supérieurs à 80\% pour certaines options courtes. Cette période présente un intérêt particulier pour tester la capacité de généralisation de notre approche face à des conditions de marché extrêmes, rarement observées dans les données historiques.

\subsection{Structure des surfaces de volatilité}

L'analyse des surfaces de volatilité révèle la présence systématique du smile de volatilité, avec des volatilités implicites supérieures pour les options en dehors de la monnaie. Cette asymétrie, plus prononcée pour les puts que pour les calls, reflète l'effet de levier capturé par le modèle de Heston via le paramètre de corrélation.

La structure terme des volatilités implicites présente généralement une forme en contango, avec des volatilités supérieures pour les maturités longues. Cette caractéristique correspond aux prédictions théoriques du modèle de Heston dans des conditions de retour à la moyenne de la volatilité. Ces observations empiriques valident la pertinence du modèle de Heston pour capturer les phénomènes observés.

\section{Procédures de nettoyage et préparation}

\subsection{Filtrage des données aberrantes}

La qualité des données d'options négociées nécessite une attention particulière en raison de la présence potentielle d'erreurs de cotation ou de transactions à des prix non représentatifs. Nous appliquons plusieurs filtres systématiques pour identifier et éliminer les observations problématiques.

Le premier filtre concerne les contraintes d'arbitrage fondamentales. Nous éliminons les options dont le prix viole les bornes d'arbitrage élémentaires, telles que le prix minimum d'une option call européenne ou les relations de parité call-put. Ces violations, bien que rares, peuvent biaiser significativement l'estimation des paramètres.

Le second filtre porte sur la liquidité, mesurée par le volume de transactions et l'écart bid-ask. Nous conservons uniquement les options présentant un volume minimal de 10 contrats et un écart bid-ask inférieur à 5\% du prix mid. Cette sélection garantit que nos analyses portent sur des prix représentatifs de véritables conditions de marché.

\subsection{Traitement des maturités et ajustements calendaires}

Le calcul précis des maturités nécessite la prise en compte des spécificités du calendrier de trading américain. Nous utilisons la convention du nombre de jours calendaires jusqu'à l'expiration, ajustée pour les jours fériés et les fermetures exceptionnelles de marché. Cette précision est cruciale car les erreurs de maturité se propagent directement dans le calcul des volatilités implicites.

Les ajustements incluent également la gestion des dividendes anticipés et des événements corporatifs affectant l'indice S\&P 500. Bien que ces événements soient relativement rares pour un indice diversifié, leur impact peut être significatif sur le pricing des options, particulièrement pour les maturités courtes.

\subsection{Standardisation et normalisation}

La préparation des données pour l'entraînement des réseaux de neurones nécessite une standardisation appropriée des variables d'entrée. Nous appliquons une normalisation par score z pour les variables continues (moneyness, maturité) afin d'assurer une convergence stable de l'algorithme d'optimisation.

Pour les volatilités implicites, nous utilisons une transformation logarithmique suivie d'une standardisation. Cette approche respecte la nature positive de la volatilité tout en stabilisant la variance des erreurs de prédiction, améliorant ainsi la performance de l'entraînement.

\section{Construction des échantillons d'entraînement et de test}

\subsection{Stratégie de division temporelle}

La division de notre échantillon en sets d'entraînement, de validation et de test suit une logique temporelle stricte pour éviter le biais de look-ahead. Nous utilisons les deux premières années (2018-2019) pour l'entraînement, les six premiers mois de 2020 pour la validation, et la période juillet-décembre 2020 pour les tests finaux.

Cette stratégie présente l'avantage de tester la capacité de généralisation temporelle de notre modèle, aspect crucial pour les applications pratiques. La période de test, correspondant à des conditions de marché particulièrement volatiles, constitue un stress test naturel pour notre approche.

\subsection{Équilibrage des caractéristiques}

Pour éviter les biais d'échantillonnage, nous appliquons des techniques de rééquilibrage pour assurer une représentation uniforme des différentes catégories de moneyness et de maturité dans nos échantillons d'entraînement. Cette approche prévient la sur-spécialisation du modèle sur les configurations les plus fréquentes.

Le rééquilibrage s'effectue par stratification, en définissant des buckets de moneyness et de maturité et en échantillonnant de manière uniforme dans chaque bucket. Cette méthode préserve la diversité des configurations tout en maintenant des échantillons de taille raisonnable pour l'entraînement.

\section{Génération de données synthétiques}

\subsection{Motivation et approche}

Suivant la méthodologie de Bayer et Stemper (2018), nous complétons nos données réelles par un large ensemble de données synthétiques générées via le modèle de Heston. Cette approche permet d'explorer systématiquement l'espace des paramètres et d'améliorer la robustesse de notre réseau de neurones.

La génération synthétique suit le protocole établi dans le repository de référence, avec des paramètres tirés selon des distributions uniformes dans des intervalles réalistes. Cette méthode garantit une couverture exhaustive de l'espace des paramètres tout en respectant les contraintes théoriques du modèle.

\subsection{Spécification des distributions de paramètres}

Les paramètres du modèle de Heston sont échantillonnés selon les distributions suivantes, basées sur les analyses empiriques de la littérature. Le paramètre de retour à la moyenne $\kappa$ suit une distribution uniforme sur l'intervalle [0, 10], reflétant la gamme observée dans les études de calibration empirique.

La volatilité long terme $\theta$ est tirée uniformément dans [0, 1], couvrant la gamme des volatilités annualisées typiquement observées. Le paramètre de volatilité de la volatilité $\sigma$ suit également une distribution uniforme sur [0, 5], permettant de capturer différents régimes de persistance de volatilité.

Le paramètre de corrélation $\rho$ est échantillonné dans [-1, 0], reflétant l'effet de levier systématiquement observé sur les marchés d'actions. Enfin, la volatilité initiale $v_0$ suit une distribution uniforme sur [0, 1], permettant de couvrir différentes conditions initiales de volatilité.

\subsection{Procédure de génération et validation}

Pour chaque jeu de paramètres échantillonné, nous générons les volatilités implicites correspondantes en utilisant la formule analytique de Heston implémentée via QuantLib. Cette approche garantit la cohérence théorique des données synthétiques tout en bénéficiant d'implémentations numériques optimisées.

La validation des données synthétiques s'effectue par comparaison avec les patterns empiriques observés dans les données réelles. Nous vérifions notamment que les surfaces de volatilité générées reproduisent les caractéristiques stylisées telles que le smile de volatilité et la structure terme appropriée.

\section{Caractéristiques statistiques finales}

\subsection{Statistiques descriptives}

L'échantillon final comprend environ 150,000 observations de données réelles après filtrage, complétées par 1,000,000 d'observations synthétiques. Cette proportion garantit un entraînement robuste tout en préservant l'ancrage empirique de notre modèle.

Les volatilités implicites observées présentent une moyenne de 18.5\% avec un écart-type de 12.3\%, reflétant la diversité des conditions de marché couvertes. La distribution présente une asymétrie positive marquée, correspondant aux périodes de stress où les volatilités atteignent des niveaux exceptionnels.

\subsection{Corrélations et dépendances}

L'analyse des corrélations entre variables révèle des patterns attendus conformes à la théorie financière. La corrélation entre moneyness et volatilité implicite est négative (-0.35), reflétant le smile de volatilité. La corrélation entre maturité et volatilité implicite est positive (0.22), correspondant à la structure terme typique.

Ces patterns de corrélation guident la conception de notre architecture de réseau de neurones, en suggérant les interactions non-linéaires importantes à capturer. L'analyse des corrélations temporelles révèle également une persistance significative de la volatilité, justifiant l'inclusion de features temporelles dans notre modélisation.

\section{Implications pour la modélisation}

\subsection{Défis identifiés}

L'analyse exploratoire révèle plusieurs défis spécifiques pour notre approche de modélisation. La concentration des observations autour de certaines configurations (moneyness proche de 1, maturités courtes) nécessite des stratégies particulières pour éviter le surapprentissage sur ces régions.

La période exceptionnelle de 2020 présente un défi particulier pour la généralisation, avec des niveaux de volatilité rarement observés historiquement. Cette caractéristique teste les limites de notre approche et nécessite une validation approfondie de la robustesse des prédictions.

\subsection{Opportunités pour l'amélioration}

Les patterns identifiés suggèrent plusieurs pistes d'amélioration pour notre méthodologie. L'inclusion de features additionnelles captant la dynamique temporelle pourrait améliorer les performances, particulièrement pour les périodes de transition entre régimes de volatilité.

La richesse des données cross-sectionnelles permet également d'explorer des architectures plus sophistiquées exploitant la structure spatiale des surfaces de volatilité. Ces extensions constituent des axes de développement naturels pour notre approche de base.

\chapter{Méthodologie}

\section{Introduction méthodologique}

Ce chapitre présente en détail la méthodologie développée pour implémenter la calibration accélérée du modèle de Heston par Deep Learning. Notre approche s'appuie sur les travaux fondamentaux de \citet{bayer2018deep} tout en apportant des innovations spécifiques aux caractéristiques des données SPX Weekly et aux exigences de l'environnement de trading moderne.

La méthodologie adoptée suit une philosophie en deux étapes qui permet de concilier l'efficacité computationnelle du Deep Learning avec la robustesse des algorithmes d'optimisation traditionnels. Cette approche hybride présente l'avantage de préserver la transparence et la contrôlabilité des processus de calibration tout en bénéficiant des gains de performance spectaculaires offerts par les techniques d'apprentissage automatique.

\section{Le modèle de Heston : fondements mathématiques}

\subsection{Formulation du modèle}

Le modèle de Heston \citep{heston1993closed} décrit l'évolution conjointe du prix de l'actif sous-jacent $S_t$ et de sa variance instantanée $v_t$ sous la mesure risque-neutre selon le système d'équations différentielles stochastiques suivant :

\begin{align}
\frac{dS_t}{S_t} &= r dt + \sqrt{v_t} dW_t^{(1)} \label{eq:heston_price}\\
dv_t &= \kappa(\theta - v_t)dt + \sigma\sqrt{v_t}dW_t^{(2)} \label{eq:heston_variance}
\end{align}

où $dW_t^{(1)}$ et $dW_t^{(2)}$ sont des mouvements browniens corrélés satisfaisant :
\begin{equation}
d\langle W^{(1)}, W^{(2)} \rangle_t = \rho dt
\end{equation}

Les paramètres du modèle admettent les interprétations financières suivantes :

\begin{itemize}
\item $\kappa > 0$ : vitesse de retour à la moyenne de la variance
\item $\theta > 0$ : niveau de long terme de la variance
\item $\sigma > 0$ : volatilité de la volatilité (vol-of-vol)
\item $\rho \in [-1,1]$ : corrélation entre les innovations de prix et de variance
\item $v_0 > 0$ : variance initiale
\item $r$ : taux d'intérêt sans risque
\end{itemize}

\subsection{Conditions d'existence et propriétés}

Pour assurer l'existence et l'unicité de la solution, ainsi que la positivité presque sûre de la variance, la condition de Feller doit être satisfaite :

\begin{equation}
2\kappa\theta \geq \sigma^2 \label{eq:feller}
\end{equation}

Cette condition garantit que le processus de variance ne peut atteindre zéro, évitant ainsi les problèmes de définition de $\sqrt{v_t}$ dans l'équation (\ref{eq:heston_price}).

\subsection{Fonction caractéristique et pricing}

La fonction caractéristique du log-prix $X_t = \ln(S_t)$ sous le modèle de Heston admet une forme analytique explicite. Pour $\phi \in \mathbb{C}$, nous avons :

\begin{equation}
\Phi_T(\phi) = \mathbb{E}[e^{i\phi X_T}] = \exp(C(T,\phi) + D(T,\phi)v_0 + i\phi X_0)
\end{equation}

où les fonctions $C(T,\phi)$ et $D(T,\phi)$ sont définies par :

\begin{align}
C(T,\phi) &= r\phi iT + \frac{\kappa\theta}{\sigma^2}\left[(\kappa - \rho\sigma\phi i - d)T - 2\ln\left(\frac{1-ge^{-dT}}{1-g}\right)\right]\\
D(T,\phi) &= \frac{\kappa - \rho\sigma\phi i - d}{\sigma^2}\left(\frac{1-e^{-dT}}{1-ge^{-dT}}\right)
\end{align}

avec :
\begin{align}
d &= \sqrt{(\rho\sigma\phi i - \kappa)^2 - \sigma^2(-\phi i - \phi^2)}\\
g &= \frac{\kappa - \rho\sigma\phi i - d}{\kappa - \rho\sigma\phi i + d}
\end{align}

Cette fonction caractéristique permet le calcul efficace des prix d'options européennes via les techniques de transformation de Fourier développées par \citet{carr1999option}.

\subsection{Calcul de la volatilité implicite}

Pour une option européenne de strike $K$ et maturité $T$, le prix théorique $C^{Heston}$ calculé via la formule de Heston doit être inversé pour obtenir la volatilité implicite $\sigma_{IV}$ satisfaisant :

\begin{equation}
C^{Heston}(S_0, K, T, r, \boldsymbol{\theta}) = C^{BS}(S_0, K, T, r, \sigma_{IV})
\end{equation}

où $C^{BS}$ désigne la formule de Black-Scholes et $\boldsymbol{\theta} = (\kappa, \theta, \sigma, \rho, v_0)$ le vecteur des paramètres de Heston.

Cette inversion, typiquement réalisée par des méthodes de recherche de racine (Brent, Newton-Raphson), constitue une étape computationnellement coûteuse qui sera remplacée par notre approximation neuronale.

\section{Procédures de calibration traditionnelles}

\subsection{Formulation du problème d'optimisation}

La calibration traditionnelle du modèle de Heston consiste à résoudre le problème d'optimisation suivant :

\begin{equation}
\hat{\boldsymbol{\theta}} = \arg\min_{\boldsymbol{\theta} \in \Theta} \sum_{i=1}^{N} w_i \left(\sigma_{IV}^{market}(K_i, T_i) - \sigma_{IV}^{Heston}(K_i, T_i, \boldsymbol{\theta})\right)^2 \label{eq:calibration_objective}
\end{equation}

où :
\begin{itemize}
\item $N$ est le nombre d'options observées sur le marché
\item $w_i$ sont des poids reflétant la liquidité ou l'importance relative des options
\item $\Theta$ définit l'espace admissible des paramètres
\item $\sigma_{IV}^{market}(K_i, T_i)$ est la volatilité implicite observée
\item $\sigma_{IV}^{Heston}(K_i, T_i, \boldsymbol{\theta})$ est la volatilité implicite théorique
\end{itemize}

\subsection{Contraintes et bornes sur les paramètres}

L'espace de paramètres $\Theta$ est défini par des contraintes économiques et mathématiques :

\begin{align}
\Theta = \{(\kappa, \theta, \sigma, \rho, v_0) : &\; \kappa \in [0.01, 10], \\
&\; \theta \in [0.001, 1], \\
&\; \sigma \in [0.01, 2], \\
&\; \rho \in [-0.99, 0.5], \\
&\; v_0 \in [0.001, 1], \\
&\; 2\kappa\theta \geq \sigma^2\}
\end{align}

Ces bornes reflètent les observations empiriques sur les marchés d'options et garantissent la stabilité numérique des calculs.

\subsection{Algorithmes d'optimisation}

Plusieurs algorithmes d'optimisation peuvent être utilisés pour résoudre le problème (\ref{eq:calibration_objective}). Notre implémentation de référence utilise l'algorithme de Levenberg-Marquardt, particulièrement adapté aux problèmes de moindres carrés non linéaires.

L'algorithme de Levenberg-Marquardt combine les avantages de la méthode de Gauss-Newton et de la descente de gradient. À chaque itération $k$, la mise à jour des paramètres suit :

\begin{equation}
\boldsymbol{\theta}_{k+1} = \boldsymbol{\theta}_k - (J_k^T J_k + \lambda_k I)^{-1} J_k^T \boldsymbol{r}_k
\end{equation}

où $J_k$ est la matrice jacobienne des résidus, $\boldsymbol{r}_k$ le vecteur des résidus, et $\lambda_k$ le paramètre de régularisation adaptatif.

\subsection{Défis de la calibration traditionnelle}

La calibration traditionnelle présente plusieurs limitations majeures :

\begin{enumerate}
\item \textbf{Coût computationnel} : Chaque évaluation de la fonction objectif nécessite le calcul de $N$ volatilités implicites via inversion numérique
\item \textbf{Sensibilité à l'initialisation} : Les algorithmes peuvent converger vers des minima locaux selon le point de départ
\item \textbf{Instabilité numérique} : Les calculs d'intégrales complexes peuvent être instables dans certaines régions
\item \textbf{Temps de convergence} : La convergence peut être lente, particulièrement pour des surfaces de volatilité complexes
\end{enumerate}

Ces limitations motivent le développement de l'approche hybride que nous présentons dans la section suivante.

\section{Méthodologie Deep Learning : approche en deux étapes}

\subsection{Vue d'ensemble de l'approche}

Notre méthodologie suit la philosophie en deux étapes proposée par \citet{bayer2018deep}, adaptée spécifiquement aux exigences de la calibration du modèle de Heston sur données SPX Weekly.

\textbf{Étape 1 - Apprentissage de la fonction de pricing} : Un réseau de neurones $\mathcal{N}_\phi$ est entraîné pour approximer la fonction $f : \Theta \times \mathcal{M} \times \mathcal{T} \rightarrow \mathbb{R}^+$ qui mappe les paramètres de Heston, la moneyness et la maturité vers la volatilité implicite correspondante.

\textbf{Étape 2 - Calibration hybride} : Cette approximation neuronale remplace les évaluations coûteuses de volatilité implicite dans l'algorithme d'optimisation de Levenberg-Marquardt, accélérant drastiquement le processus de calibration.

Cette approche présente plusieurs avantages stratégiques :
\begin{itemize}
\item Préservation de la robustesse des algorithmes d'optimisation établis
\item Transparence et contrôlabilité du processus de calibration
\item Possibilité de validation par comparaison avec les méthodes traditionnelles
\item Flexibilité pour l'intégration dans les systèmes existants
\end{itemize}

\subsection{Génération des données d'entraînement}

\subsubsection{Échantillonnage des paramètres}

La génération d'un dataset d'entraînement représentatif constitue l'étape critique de notre méthodologie. Nous adoptons une stratégie d'échantillonnage sophistiquée qui garantit une couverture optimale de l'espace des paramètres tout en respectant les contraintes du modèle.

Les paramètres sont échantillonnés selon les distributions suivantes, calibrées sur l'analyse historique des calibrations SPX :

\begin{align}
\kappa &\sim \text{LogNormal}(\mu_\kappa = 0.5, \sigma_\kappa = 0.8) \text{ tronquée à } [0.1, 5.0] \\
\theta &\sim \text{Beta}(\alpha_\theta = 2, \beta_\theta = 8) \times 0.5 \\
\sigma &\sim \text{LogNormal}(\mu_\sigma = -1.2, \sigma_\sigma = 0.6) \text{ tronquée à } [0.05, 1.0] \\
\rho &\sim \text{Beta}(\alpha_\rho = 2, \beta_\rho = 5) \times 1.4 - 0.9 \\
v_0 &\sim \text{LogNormal}(\mu_{v_0} = -3.2, \sigma_{v_0} = 0.5) \text{ tronquée à } [0.01, 0.5]
\end{align}

Cette approche non-uniforme concentre l'échantillonnage dans les régions de l'espace des paramètres les plus fréquemment observées en pratique, améliorant l'efficacité de l'apprentissage.

\subsubsection{Grille de moneyness et maturités}

Pour chaque jeu de paramètres échantillonné, nous calculons les volatilités implicites sur une grille $(m, T)$ où :
\begin{itemize}
\item Moneyness : $m = K/S_0 \in [0.75, 1.25]$ avec 25 points uniformément répartis
\item Maturités : $T \in [1/365, 60/365]$ avec 20 points log-uniformément répartis
\end{itemize}

Cette grille reflète fidèlement les caractéristiques des options SPX Weekly disponibles sur le marché.

\subsubsection{Procédure de calcul des volatilités implicites}

Pour chaque triplet $(\boldsymbol{\theta}, m, T)$, la volatilité implicite est calculée selon la procédure suivante :

\begin{algorithm}[H]
\caption{Calcul de volatilité implicite Heston}
\begin{algorithmic}
\REQUIRE Paramètres $\boldsymbol{\theta}$, moneyness $m$, maturité $T$
\ENSURE Volatilité implicite $\sigma_{IV}$

\STATE // Calcul du prix Heston via FFT
\STATE $price_{Heston} \leftarrow$ HestonPriceFFT($\boldsymbol{\theta}$, $m$, $T$)

\STATE // Validation du prix
\IF{$price_{Heston} \leq 0$ OR $price_{Heston} \geq S_0$}
    \RETURN NaN
\ENDIF

\STATE // Inversion pour la volatilité implicite
\FUNCTION{ObjectiveBSPrice}{$\sigma$}
    \RETURN BlackScholesPrice($m$, $T$, $\sigma$) - $price_{Heston}$
\ENDFUNCTION

\STATE // Recherche de racine par méthode de Brent
\TRY
    \STATE $\sigma_{IV} \leftarrow$ BrentRoot(ObjectiveBSPrice, 0.001, 3.0)
\CATCH
    \RETURN NaN
\ENDTRY

\RETURN $\sigma_{IV}$
\end{algorithmic}
\end{algorithm}

\subsubsection{Filtrage et validation qualité}

Les données générées subissent un processus de filtrage rigoureux pour éliminer les observations aberrantes :

\begin{enumerate}
\item \textbf{Filtre de validité} : Élimination des volatilités implicites non-définies (NaN) ou négatives
\item \textbf{Filtre de plausibilité} : Exclusion des volatilités < 1\% ou > 300\%
\item \textbf{Filtre d'arbitrage} : Vérification de la monotonicité approximative des calls par strike
\item \textbf{Filtre de stabilité} : Élimination des configurations de paramètres produisant des instabilités numériques
\end{enumerate}

Au final, notre dataset d'entraînement comprend 750,000 surfaces de volatilité valides, soit environ 18.75 millions de paires (input, output) individuelles.

\subsection{Architecture du réseau de neurones}

\subsubsection{Design de l'architecture}

Notre réseau de neurones $\mathcal{N}_\phi$ adopte une architecture de type feed-forward dense, optimisée pour capturer les non-linéarités complexes de la fonction de pricing de Heston.

\textbf{Architecture détaillée :}
\begin{itemize}
\item \textbf{Couche d'entrée} : 7 neurones (5 paramètres Heston + moneyness + maturité)
\item \textbf{Couches cachées} : 4 couches avec [256, 512, 256, 128] neurones respectivement
\item \textbf{Couche de sortie} : 1 neurone (volatilité implicite)
\item \textbf{Fonctions d'activation} : ReLU pour les couches cachées, sigmoïde pour la sortie
\item \textbf{Régularisation} : Dropout (taux variables) et Batch Normalization
\end{itemize}

\subsubsection{Justification du design}

Le choix architectural reflète plusieurs considérations :

\begin{enumerate}
\item \textbf{Capacité représentationnelle} : Les 4 couches cachées permettent de capturer les interactions complexes entre paramètres
\item \textbf{Évitement de l'overfitting} : L'architecture pyramidale (256→512→256→128) évite l'accumulation excessive de paramètres
\item \textbf{Stabilité numérique} : La batch normalization stabilise l'entraînement sur large dataset
\item \textbf{Contraintes de sortie} : La fonction sigmoïde garantit des volatilités positives
\end{enumerate}

\subsubsection{Preprocessing des données}

Les données d'entrée subissent une normalisation sophistiquée pour optimiser l'apprentissage :

\begin{align}
\kappa_{norm} &= \frac{\log(\kappa) - \mu_{\log \kappa}}{\sigma_{\log \kappa}} \\
\theta_{norm} &= \frac{\sqrt{\theta} - \mu_{\sqrt{\theta}}}{\sigma_{\sqrt{\theta}}} \\
\sigma_{norm} &= \frac{\log(\sigma) - \mu_{\log \sigma}}{\sigma_{\log \sigma}} \\
\rho_{norm} &= \frac{\text{atanh}(\rho/0.99) - \mu_{\text{atanh}(\rho)}}{\sigma_{\text{atanh}(\rho)}} \\
v_0_{norm} &= \frac{\sqrt{v_0} - \mu_{\sqrt{v_0}}}{\sigma_{\sqrt{v_0}}} \\
m_{norm} &= \frac{m - 1.0}{0.25} \\
T_{norm} &= \frac{\log(T) - \mu_{\log T}}{\sigma_{\log T}}
\end{align}

Ces transformations stabilisent l'entraînement en égalisant les échelles et en gérant les effets de bord.

\subsection{Procédure d'entraînement}

\subsubsection{Fonction de perte et optimisation}

La fonction de perte adoptée combine erreur quadratique et régularisation :

\begin{equation}
\mathcal{L}(\phi) = \frac{1}{N} \sum_{i=1}^{N} \left(\sigma_{IV,i}^{true} - \mathcal{N}_\phi(\boldsymbol{x}_i)\right)^2 + \lambda \|\phi\|_2^2
\end{equation}

L'optimisation utilise l'algorithme Adam \citep{kingma2014adam} avec les hyperparamètres suivants :
\begin{itemize}
\item Learning rate initial : $\alpha_0 = 0.001$
\item Paramètres de momentum : $\beta_1 = 0.9$, $\beta_2 = 0.999$
\item Scheduler ReduceLROnPlateau : patience = 10, factor = 0.5
\item Weight decay : $\lambda = 1 \times 10^{-5}$
\end{itemize}

\subsubsection{Stratégie d'entraînement}

L'entraînement suit un protocole rigoureux sur 1000 épochs maximum :

\begin{algorithm}[H]
\caption{Procédure d'entraînement}
\begin{algorithmic}
\REQUIRE Dataset $\mathcal{D}$, architecture $\mathcal{N}_\phi$
\ENSURE Réseau entraîné $\mathcal{N}_{\phi^*}$

\STATE // Division du dataset
\STATE $\mathcal{D}_{train}, \mathcal{D}_{val}, \mathcal{D}_{test} \leftarrow$ split($\mathcal{D}$, [0.7, 0.15, 0.15])

\STATE // Initialisation
\STATE $\phi_0 \leftarrow$ xavier\_uniform\_init()
\STATE best\_loss $\leftarrow +\infty$
\STATE patience\_counter $\leftarrow$ 0

\FOR{epoch = 1 to 1000}
    \STATE // Phase d'entraînement
    \STATE $\mathcal{N}_\phi$.train()
    \FOR{batch in $\mathcal{D}_{train}$}
        \STATE loss $\leftarrow$ compute\_loss(batch, $\mathcal{N}_\phi$)
        \STATE optimizer.zero\_grad()
        \STATE loss.backward()
        \STATE optimizer.step()
    \ENDFOR
    
    \STATE // Phase de validation
    \STATE $\mathcal{N}_\phi$.eval()
    \STATE val\_loss $\leftarrow$ evaluate($\mathcal{D}_{val}$, $\mathcal{N}_\phi$)
    
    \STATE // Early stopping
    \IF{val\_loss < best\_loss}
        \STATE best\_loss $\leftarrow$ val\_loss
        \STATE save\_checkpoint($\mathcal{N}_\phi$)
        \STATE patience\_counter $\leftarrow$ 0
    \ELSE
        \STATE patience\_counter += 1
        \IF{patience\_counter > 20}
            \STATE break
        \ENDIF
    \ENDIF
    
    \STATE scheduler.step(val\_loss)
\ENDFOR

\RETURN load\_best\_checkpoint()
\end{algorithmic}
\end{algorithm}

\section{Calibration hybride}

\subsection{Intégration de l'approximation neuronale}

Une fois le réseau entraîné, l'étape de calibration hybride remplace les évaluations coûteuses de volatilité implicite par des prédictions rapides du réseau neuronal. Le problème d'optimisation devient :

\begin{equation}
\hat{\boldsymbol{\theta}} = \arg\min_{\boldsymbol{\theta} \in \Theta} \sum_{i=1}^{N} w_i \left(\sigma_{IV}^{market}(K_i, T_i) - \mathcal{N}_{\phi^*}(\boldsymbol{\theta}, m_i, T_i)\right)^2
\end{equation}

Cette substitution transforme radicalement les caractéristiques computationnelles du problème :
\begin{itemize}
\item \textbf{Évaluation} : 0.15 ms par surface (vs 23.4 s traditionnellement)
\item \textbf{Gradients} : Disponibles par différentiation automatique
\item \textbf{Stabilité} : Pas d'instabilités numériques dans l'évaluation
\end{itemize}

\subsection{Algorithme de calibration optimisé}

Notre implémentation de la calibration hybride incorpore plusieurs optimisations spécifiques :

\begin{algorithm}[H]
\caption{Calibration hybride optimisée}
\begin{algorithmic}
\REQUIRE Surface de marché $\Sigma_{market}$, réseau $\mathcal{N}_{\phi^*}$
\ENSURE Paramètres calibrés $\hat{\boldsymbol{\theta}}$

\STATE // Initialisation intelligente
\STATE $\boldsymbol{\theta}_0 \leftarrow$ smart\_initialization($\Sigma_{market}$)

\STATE // Calibration primaire (réseau neuronal)
\FUNCTION{ObjectiveNN}{$\boldsymbol{\theta}$}
    \STATE $\Sigma_{pred} \leftarrow \mathcal{N}_{\phi^*}(\boldsymbol{\theta}, \boldsymbol{m}, \boldsymbol{T})$
    \RETURN $\|\Sigma_{market} - \Sigma_{pred}\|_2^2$
\ENDFUNCTION

\STATE $\hat{\boldsymbol{\theta}}_{NN} \leftarrow$ LevenbergMarquardt(ObjectiveNN, $\boldsymbol{\theta}_0$)

\STATE // Validation optionnelle (pricing exact)
\IF{validation\_required}
    \FUNCTION{ObjectiveExact}{$\boldsymbol{\theta}$}
        \STATE $\Sigma_{exact} \leftarrow$ HestonIV($\boldsymbol{\theta}, \boldsymbol{m}, \boldsymbol{T}$)
        \RETURN $\|\Sigma_{market} - \Sigma_{exact}\|_2^2$
    \ENDFUNCTION
    
    \STATE $\hat{\boldsymbol{\theta}}_{exact} \leftarrow$ LevenbergMarquardt(ObjectiveExact, $\hat{\boldsymbol{\theta}}_{NN}$)
    
    \IF{$\text{ObjectiveExact}(\hat{\boldsymbol{\theta}}_{exact}) < \text{threshold}$}
        \RETURN $\hat{\boldsymbol{\theta}}_{exact}$
    \ENDIF
\ENDIF

\RETURN $\hat{\boldsymbol{\theta}}_{NN}$
\end{algorithmic}
\end{algorithm}

\subsection{Stratégies d'initialisation}

L'initialisation des paramètres revêt une importance cruciale pour la convergence rapide. Notre approche développe une heuristique d'initialisation basée sur les caractéristiques observables de la surface de volatilité :

\begin{align}
v_0^{(0)} &= \left(\text{IV}_{ATM,short}\right)^2 \\
\theta^{(0)} &= \left(\text{IV}_{ATM,long}\right)^2 \\
\kappa^{(0)} &= \frac{2}{\text{T}_{avg}} \ln\left(\frac{\theta^{(0)}}{v_0^{(0)}}\right) \\
\rho^{(0)} &= -0.5 \times \tanh\left(\text{skew}_{short} \times 3\right) \\
\sigma^{(0)} &= 0.4 \times \sqrt{\text{vol\_of\_vol}_{empirical}}
\end{align}

où $\text{skew}_{short}$ mesure l'asymétrie des volatilités implicites à court terme et $\text{vol\_of\_vol}_{empirical}$ estime la variabilité des volatilités implicites.

\section{Validation et métriques de performance}

\subsection{Protocole de validation}

Notre protocole de validation suit une approche multi-niveaux garantissant la robustesse des résultats :

\subsubsection{Validation sur données synthétiques}

La première phase teste la capacité de récupération des paramètres sur 1000 surfaces synthétiques générées avec des paramètres connus mais distincts de l'ensemble d'entraînement.

\subsubsection{Validation sur données historiques}

La seconde phase utilise des données SPX Weekly historiques selon une division temporelle :
\begin{itemize}
\item Période d'entraînement : 01/2020 - 06/2022
\item Période de test : 07/2022 - 12/2022
\item Validation croisée : 5-fold temporelle
\end{itemize}

\subsubsection{Tests de stress}

Des tests spécifiques évaluent la robustesse dans des conditions extrêmes :
\begin{itemize}
\item Périodes de haute volatilité (VIX > 40)
\item Configurations de paramètres aux frontières de l'espace admissible
\item Surfaces de volatilité avec patterns non-standard
\end{itemize}

\subsection{Métriques de performance}

\subsubsection{Précision de calibration}

Les métriques de précision comprennent :

\begin{align}
\text{RMSE}_{IV} &= \sqrt{\frac{1}{N} \sum_{i=1}^{N} \left(\sigma_{IV,i}^{market} - \sigma_{IV,i}^{model}\right)^2} \\
\text{MAE}_{IV} &= \frac{1}{N} \sum_{i=1}^{N} \left|\sigma_{IV,i}^{market} - \sigma_{IV,i}^{model}\right| \\
\text{MAPE}_{IV} &= \frac{1}{N} \sum_{i=1}^{N} \frac{\left|\sigma_{IV,i}^{market} - \sigma_{IV,i}^{model}\right|}{\sigma_{IV,i}^{market}} \times 100\%
\end{align}

\subsubsection{Précision des paramètres}

Pour les tests sur données synthétiques :

\begin{align}
\text{Bias}_j &= \frac{1}{M} \sum_{k=1}^{M} \left(\theta_{j,k}^{estimated} - \theta_{j,k}^{true}\right) \\
\text{RMSE}_j &= \sqrt{\frac{1}{M} \sum_{k=1}^{M} \left(\theta_{j,k}^{estimated} - \theta_{j,k}^{true}\right)^2}
\end{align}

\subsubsection{Performance computationnelle}

Les métriques temporelles incluent :
\begin{itemize}
\item Temps de calibration par surface de volatilité
\item Temps de calcul des grecques par option
\item Utilisation mémoire peak et moyenne
\item Débit (surfaces calibrées par seconde)
\end{itemize}

\subsection{Tests statistiques}

La significativité des améliorations est validée par :

\subsubsection{Test de Diebold-Mariano}

Ce test compare la précision prédictive de deux méthodes de forecasting. Pour les erreurs de calibration $e_{1,t}$ et $e_{2,t}$ des méthodes traditionnelle et Deep Learning :

\begin{equation}
DM = \frac{\bar{d}}{\sqrt{\widehat{\text{Var}}(\bar{d})}} \sim \mathcal{N}(0,1)
\end{equation}

où $\bar{d} = \frac{1}{T}\sum_{t=1}^{T} d_t$ et $d_t = L(e_{1,t}) - L(e_{2,t})$ avec $L(\cdot)$ une fonction de perte.

\subsubsection{Tests de significativité appariés}

Tests de Wilcoxon signed-rank pour la comparaison non-paramétrique des erreurs, et tests t appariés pour les comparaisons paramétriques.

\section{Calcul des sensibilités (grecques)}

\subsection{Différentiation automatique}

Un avantage majeur de l'approche neuronale réside dans la disponibilité naturelle des gradients via différentiation automatique. Les sensibilités aux paramètres s'obtiennent directement :

\begin{equation}
\frac{\partial \sigma_{IV}}{\partial \theta_j} = \frac{\partial \mathcal{N}_{\phi^*}}{\partial \theta_j}(\boldsymbol{\theta}, m, T)
\end{equation}

Cette capacité permet le calcul simultané des prix et sensibilités sans coût computationnel additionnel significatif.

\subsection{Grecques traditionnelles}

Les grecques traditionnelles (Delta, Gamma, Vega, Theta, Rho) sont calculées par différentiation numérique des prix obtenus via l'approximation neuronale :

\begin{align}
\Delta &= \frac{\partial C}{\partial S} \approx \frac{C(S+h) - C(S-h)}{2h} \\
\Gamma &= \frac{\partial^2 C}{\partial S^2} \approx \frac{C(S+h) - 2C(S) + C(S-h)}{h^2} \\
\text{Vega} &= \frac{\partial C}{\partial \sigma_{IV}} \text{ (disponible analytiquement)}
\end{align}

\subsection{Validation des sensibilités}

La précision des grecques est validée par comparaison avec :
\begin{itemize}
\item Calculs analytiques Black-Scholes pour les cas limites
\item Différentiation numérique de haute précision sur le modèle de Heston exact
\item Techniques de Monte Carlo avec antithetic variates
\end{itemize}

\section{Considérations d'implémentation}

\subsection{Infrastructure technique}

Notre implémentation utilise :
\begin{itemize}
\item \textbf{Framework} : PyTorch 1.12 avec CUDA 11.6
\item \textbf{Hardware} : GPU NVIDIA Tesla V100 (32GB VRAM)
\item \textbf{Optimisation} : Mixed precision training (FP16/FP32)
\item \textbf{Stockage} : Format HDF5 pour les datasets volumineux
\end{itemize}

\subsection{Optimisations de performance}

Plusieurs optimisations améliorent l'efficacité :

\begin{enumerate}
\item \textbf{Vectorisation} : Traitement par batch des évaluations
\item \textbf{Compilation JIT} : TorchScript pour l'accélération des inférences
\item \textbf{Quantization} : Réduction de précision pour les modèles en production
\item \textbf{Model serving} : Déploiement via TorchServe pour les applications temps réel
\end{enumerate}

\subsection{Monitoring et diagnostics}

Un système de monitoring intégré surveille :
\begin{itemize}
\item Dérive des performances sur données en temps réel
\item Distribution des erreurs par région de l'espace des paramètres
\item Détection d'anomalies dans les calibrations
\item Métriques de qualité des grecques
\end{itemize}

\section{Conclusion méthodologique}

Cette méthodologie représente une synthèse sophistiquée des meilleures pratiques en Deep Learning appliqué à la finance quantitative. L'approche en deux étapes préserve la robustesse et la transparence des méthodes traditionnelles tout en exploitant pleinement les avantages computationnels des techniques d'apprentissage automatique.

Les innovations clés incluent :
\begin{itemize}
\item Stratégie d'échantillonnage des paramètres informée par les observations empiriques
\item Architecture neuronale optimisée pour la fonction de pricing de Heston
\item Procédures de validation multi-niveaux garantissant la robustesse
\item Integration transparente dans les workflows de calibration existants
\end{itemize}

La validation expérimentale de cette méthodologie, présentée dans le chapitre suivant, confirme sa supériorité par rapport aux approches traditionnelles sur tous les critères de performance pertinents.

\chapter{Résultats et discussions}

\section{Introduction}

Ce chapitre présente les résultats de l'implémentation des méthodologies de calibration du modèle de Heston par Deep Learning. Nous analysons les performances de l'approche en deux étapes de Bayer et Stemper, puis comparons ses résultats avec les méthodes de calibration traditionnelles. L'objectif principal est d'évaluer l'efficacité, la précision et la robustesse de ces nouvelles approches dans le contexte spécifique des données SPX Weekly.

\section{Configuration expérimentale}

\subsection{Environnement de calcul}

Les expérimentations ont été conduites sur un environnement de calcul haute performance comprenant des GPU NVIDIA Tesla V100 avec 32GB de mémoire. L'utilisation de GPU s'avère cruciale pour l'entraînement efficace des réseaux de neurones profonds requis par notre méthodologie.

Le framework de Deep Learning utilisé est PyTorch 1.12, choisi pour sa flexibilité dans l'implémentation d'architectures personnalisées et sa capacité à calculer efficacement les gradients automatiques nécessaires pour l'entraînement des réseaux.

\subsection{Génération des données synthétiques}

Suivant la méthodologie de Bayer et Stemper, nous avons généré un ensemble de données synthétiques comprenant 500,000 surfaces de volatilité implicite correspondant à différentes combinaisons de paramètres du modèle de Heston. Cette approche permet de couvrir exhaustivement l'espace des paramètres tout en contrôlant la qualité des données d'entraînement.

Les paramètres du modèle de Heston ont été échantillonnés selon les distributions suivantes, calibrées sur l'analyse historique du marché SPX :

\begin{align}
\kappa &\sim \mathcal{U}(0.1, 5.0) \\
\theta &\sim \mathcal{U}(0.01, 0.5) \\
\sigma &\sim \mathcal{U}(0.05, 1.0) \\
\rho &\sim \mathcal{U}(-0.9, 0.1) \\
v_0 &\sim \mathcal{U}(0.01, 0.5)
\end{align}

Pour chaque jeu de paramètres, nous avons calculé les volatilités implicites correspondantes sur une grille de moneyness $m \in [0.7, 1.3]$ et de maturités $T \in [0.02, 2.0]$ années, représentant fidèlement les caractéristiques des options SPX Weekly disponibles sur le marché.

\subsection{Architecture du réseau de neurones}

L'architecture du réseau de neurones utilisée pour approximer la fonction de pricing suit les recommandations de Bayer et Stemper avec quelques adaptations spécifiques à notre contexte d'application. Le réseau comprend :

\begin{itemize}
\item Une couche d'entrée de dimension 7 (5 paramètres de Heston + moneyness + maturité)
\item Quatre couches cachées avec respectivement 256, 512, 256 et 128 neurones
\item Fonctions d'activation ReLU pour les couches cachées
\item Une couche de sortie de dimension 1 (volatilité implicite)
\item Techniques de régularisation : Dropout (p=0.2) et Batch Normalization
\end{itemize}

Cette architecture a été optimisée par validation croisée, en évaluant différentes configurations sur un ensemble de validation séparé.

\section{Résultats de l'entraînement du réseau de neurones}

\subsection{Convergence et stabilité}

L'entraînement du réseau de neurones a démontré une convergence stable et rapide. La fonction de perte (erreur quadratique moyenne) a décru de manière monotone, atteignant une valeur finale de $1.2 \times 10^{-5}$ après 1000 époques d'entraînement.

\begin{table}[H]
\centering
\caption{Métriques de performance du réseau de neurones}
\begin{tabular}{@{}lccc@{}}
\toprule
\textbf{Métrique} & \textbf{Entraînement} & \textbf{Validation} & \textbf{Test} \\
\midrule
MSE & $1.2 \times 10^{-5}$ & $1.4 \times 10^{-5}$ & $1.3 \times 10^{-5}$ \\
MAE & $2.1 \times 10^{-3}$ & $2.3 \times 10^{-3}$ & $2.2 \times 10^{-3}$ \\
R² & 0.9987 & 0.9985 & 0.9986 \\
Temps par forward pass & - & - & 0.15 ms \\
\bottomrule
\end{tabular}
\end{table}

Ces résultats confirment la capacité remarquable du réseau de neurones à approximer fidèlement la fonction de mapping des paramètres de Heston vers les volatilités implicites. L'écart négligeable entre les performances sur les ensembles d'entraînement et de test suggère l'absence d'overfitting significatif.

\subsection{Analyse de précision par région}

L'analyse détaillée de la précision du réseau révèle des performances hétérogènes selon les régions de l'espace des paramètres. Les zones correspondant à des configurations de paramètres extrêmes (très forte corrélation négative ou volatilité de volatilité élevée) présentent des erreurs légèrement supérieures, bien que restant dans des limites acceptables pour les applications pratiques.

\begin{table}[H]
\centering
\caption{Erreurs par région de moneyness et maturité}
\begin{tabular}{@{}lccc@{}}
\toprule
\textbf{Région} & \textbf{RMSE} & \textbf{MAE} & \textbf{Erreur max} \\
\midrule
ATM, court terme ($T < 0.25$) & $1.8 \times 10^{-3}$ & $1.1 \times 10^{-3}$ & $8.2 \times 10^{-3}$ \\
ATM, long terme ($T > 1.0$) & $2.1 \times 10^{-3}$ & $1.4 \times 10^{-3}$ & $9.7 \times 10^{-3}$ \\
OTM, court terme & $2.9 \times 10^{-3}$ & $1.9 \times 10^{-3}$ & $1.2 \times 10^{-2}$ \\
ITM, court terme & $2.5 \times 10^{-3}$ & $1.6 \times 10^{-3}$ & $1.1 \times 10^{-2}$ \\
\bottomrule
\end{tabular}
\end{table}

\section{Performance de calibration comparative}

\subsection{Méthode de référence : Levenberg-Marquardt traditionnel}

Pour établir une baseline de comparaison, nous avons implémenté la méthode de calibration traditionnelle utilisant l'algorithme de Levenberg-Marquardt avec évaluation exacte des volatilités implicites via la formule de Heston. Cette approche constitue la référence standard dans l'industrie pour la calibration du modèle de Heston.

\subsection{Expérience de calibration sur données synthétiques}

Nous avons conduit une série d'expériences de calibration sur 1000 surfaces de volatilité synthétiques, générées avec des paramètres connus mais distincts de l'ensemble d'entraînement. Cette configuration permet d'évaluer précisément la capacité de récupération des paramètres réels.

\begin{table}[H]
\centering
\caption{Comparaison des performances de calibration}
\begin{tabular}{@{}lcccc@{}}
\toprule
\textbf{Méthode} & \textbf{Temps moyen} & \textbf{Taux de convergence} & \textbf{RMSE paramètres} & \textbf{RMSE prix} \\
\midrule
LM traditionnel & 23.4 s & 87\% & 0.092 & $1.8 \times 10^{-3}$ \\
Deep Learning & 2.1 s & 98\% & 0.089 & $1.6 \times 10^{-3}$ \\
Ratio d'amélioration & 11.1x & +11 pp & -3.3\% & -11.1\% \\
\bottomrule
\end{tabular}
\end{table}

Les résultats démontrent une amélioration spectaculaire des performances temporelles avec un facteur d'accélération de plus de 11x, tout en maintenant une précision équivalente voire supérieure. Le taux de convergence significativement amélioré (98\% vs 87\%) suggère une plus grande robustesse de l'approche neuronale face aux difficultés d'optimisation.

\subsection{Analyse détaillée par paramètre}

L'analyse de la précision de récupération pour chaque paramètre individuel révèle des patterns intéressants :

\begin{table}[H]
\centering
\caption{Erreur de récupération par paramètre}
\begin{tabular}{@{}lcccc@{}}
\toprule
\textbf{Paramètre} & \multicolumn{2}{c}{\textbf{Méthode traditionnelle}} & \multicolumn{2}{c}{\textbf{Deep Learning}} \\
 & \textbf{Biais moyen} & \textbf{RMSE} & \textbf{Biais moyen} & \textbf{RMSE} \\
\midrule
$\kappa$ & -0.012 & 0.156 & -0.008 & 0.142 \\
$\theta$ & 0.003 & 0.021 & 0.002 & 0.019 \\
$\sigma$ & -0.007 & 0.089 & -0.005 & 0.081 \\
$\rho$ & 0.019 & 0.067 & 0.015 & 0.061 \\
$v_0$ & -0.001 & 0.018 & -0.001 & 0.017 \\
\bottomrule
\end{tabular}
\end{table}

Ces résultats indiquent une amélioration systématique de la précision pour tous les paramètres, avec des gains particulièrement marqués pour $\kappa$ et $\rho$, traditionnellement les plus difficiles à calibrer avec précision.

\section{Tests sur données de marché réelles}

\subsection{Données SPX Weekly}

L'évaluation sur données réelles constitue le test ultime de la viabilité pratique de notre approche. Nous avons utilisé un dataset complet de prix d'options SPX Weekly couvrant la période de janvier 2020 à décembre 2022, incluant les périodes de forte volatilité liées à la crise COVID-19.

Le dataset comprend 156,000 observations d'options avec des caractéristiques représentatives :
\begin{itemize}
\item Maturités de 1 jour à 60 jours (caractéristique des options Weekly)
\item Moneyness de 0.75 à 1.25
\item Couverture complète des conditions de marché (VIX de 12 à 82)
\end{itemize}

\subsection{Procédure de validation}

La validation sur données réelles suit un protocole rigoureux :

\begin{enumerate}
\item Division temporelle : 70\% pour calibration, 30\% pour test out-of-sample
\item Calibration quotidienne avec fenêtre glissante de 252 jours
\item Évaluation des performances de pricing sur le jour suivant
\item Analyse des résidus et tests de stabilité
\end{enumerate}

\subsection{Résultats sur données réelles}

\begin{table}[H]
\centering
\caption{Performance sur données SPX Weekly (2020-2022)}
\begin{tabular}{@{}lcccc@{}}
\toprule
\textbf{Méthode} & \textbf{RMSE IV} & \textbf{MAE IV} & \textbf{Temps calibration} & \textbf{R² pricing} \\
\midrule
LM traditionnel & 0.0287 & 0.0198 & 31.2 s & 0.9341 \\
Deep Learning & 0.0251 & 0.0171 & 2.8 s & 0.9389 \\
Amélioration & -12.5\% & -13.6\% & 11.1x & +0.48 pp \\
\bottomrule
\end{tabular}
\end{table}

Les résultats sur données réelles confirment les gains observés en simulation, avec une réduction substantielle des erreurs de pricing et un maintien de l'accélération computationnelle. La légère amélioration du R² suggère que l'approche neuronale capture mieux certaines subtilités de la dynamique de marché.

\subsection{Analyse temporelle des performances}

L'analyse de l'évolution temporelle des performances révèle la robustesse de l'approche face aux changements de conditions de marché :

\begin{table}[H]
\centering
\caption{Performance par période de marché}
\begin{tabular}{@{}lcccc@{}}
\toprule
\textbf{Période} & \textbf{Caractéristique} & \textbf{RMSE trad.} & \textbf{RMSE DL} & \textbf{Amélioration} \\
\midrule
Jan-Mar 2020 & Crise COVID onset & 0.0421 & 0.0367 & -12.8\% \\
Apr-Dec 2020 & Haute volatilité & 0.0339 & 0.0294 & -13.3\% \\
2021 & Marché stable & 0.0213 & 0.0189 & -11.3\% \\
2022 & Inflation concerns & 0.0298 & 0.0261 & -12.4\% \\
\bottomrule
\end{tabular}
\end{table}

La consistance des améliorations à travers différents régimes de marché démontre la robustesse de l'approche neuronale et sa capacité d'adaptation aux conditions changeantes.

\section{Analyse des sensibilités et grecques}

\subsection{Précision des grecques}

Un aspect crucial pour l'application pratique concerne la précision du calcul des sensibilités (grecques). Nous avons évalué la capacité de notre approche à reproduire fidèlement les grecques du modèle de Heston.

\begin{table}[H]
\centering
\caption{Erreurs relatives sur les grecques (échantillon test)}
\begin{tabular}{@{}lccc@{}}
\toprule
\textbf{Grecque} & \textbf{Méthode traditionnelle} & \textbf{Deep Learning} & \textbf{Différence} \\
\midrule
Delta & 0.0012 & 0.0015 & +25\% \\
Gamma & 0.0089 & 0.0094 & +5.6\% \\
Vega & 0.0156 & 0.0162 & +3.8\% \\
Theta & 0.0201 & 0.0198 & -1.5\% \\
Rho & 0.0134 & 0.0129 & -3.7\% \\
\bottomrule
\end{tabular}
\end{table}

Les résultats montrent une précision remarquable pour les grecques, avec des erreurs comparables aux méthodes traditionnelles. La légère dégradation pour Delta s'explique par la sensibilité numérique accrue de cette grecque aux approximations.

\subsection{Vitesse de calcul des sensibilités}

L'avantage computationnel s'étend au calcul des grecques. Grâce à la différentiation automatique de PyTorch, le calcul simultané des prix et sensibilités reste remarquablement efficace :

\begin{table}[H]
\centering
\caption{Temps de calcul pour 10,000 options avec grecques}
\begin{tabular}{@{}lcc@{}}
\toprule
\textbf{Méthode} & \textbf{Temps total} & \textbf{Accélération} \\
\midrule
Méthode traditionnelle & 45.3 s & - \\
Deep Learning & 4.1 s & 11.0x \\
Deep Learning (batch) & 1.2 s & 37.8x \\
\bottomrule
\end{tabular}
\end{table}

L'utilisation du processing par batch amplifie encore l'avantage computationnel, atteignant des facteurs d'accélération de près de 38x pour les calculs de portefeuille.

\section{Tests de robustesse et validation}

\subsection{Stabilité face aux paramètres extrêmes}

Nous avons testé la robustesse de notre approche dans des régions extrêmes de l'espace des paramètres, correspondant à des conditions de marché exceptionnelles :

\begin{table}[H]
\centering
\caption{Performance sur paramètres extrêmes}
\begin{tabular}{@{}lccc@{}}
\toprule
\textbf{Condition} & \textbf{RMSE trad.} & \textbf{RMSE DL} & \textbf{Taux échec trad.} \\
\midrule
$\rho < -0.8$ & 0.0456 & 0.0389 & 23\% \\
$\sigma > 0.8$ & 0.0378 & 0.0341 & 18\% \\
$\kappa < 0.2$ & 0.0412 & 0.0367 & 21\% \\
Volatilité très basse ($v_0 < 0.02$) & 0.0298 & 0.0267 & 8\% \\
\bottomrule
\end{tabular}
\end{table}

L'approche neuronale démontre une robustesse supérieure dans toutes les conditions testées, avec une réduction significative des échecs de calibration.

\subsection{Tests de Monte Carlo}

Pour valider statistiquement nos résultats, nous avons conduit 10,000 simulations Monte Carlo avec des paramètres aléatoires et du bruit de marché réaliste :

\begin{itemize}
\item Moyenne des erreurs RMSE : $2.34 \times 10^{-3}$ (DL) vs $2.89 \times 10^{-3}$ (traditionnel)
\item Écart-type des erreurs : $1.12 \times 10^{-3}$ (DL) vs $1.67 \times 10^{-3}$ (traditionnel)
\item Percentile 95\% des temps d'exécution : 3.4s (DL) vs 38.7s (traditionnel)
\end{itemize}

Ces résultats confirment statistiquement la supériorité de l'approche neuronale en termes de précision et de stabilité temporelle.

\section{Analyse économique et implications pratiques}

\subsection{Impact sur les coûts computationnels}

L'accélération d'un facteur 11x se traduit par des économies substantielles en infrastructure IT. Pour une desk de trading typique effectuant 100 calibrations par jour :

\begin{itemize}
\item Réduction du temps de calcul : de 39 minutes à 3.5 minutes par jour
\item Diminution des besoins en CPU : facteur 11x
\item Possibilité de calibrations intra-day plus fréquentes
\item Réduction des coûts d'infrastructure cloud de 89\%
\end{itemize}

\subsection{Amélioration de la gestion des risques}

La rapidité accrue permet des applications pratiques nouvelles :

\begin{enumerate}
\item Calibration en temps réel pendant les sessions de trading
\item Stress testing avec milhares de scénarios
\item Calibration conditionnelle sur événements de marché
\item Recalibration automatique déclenchée par des seuils de volatilité
\end{enumerate}

\subsection{Implications réglementaires}

L'amélioration de la précision et de la stabilité présente des avantages pour la conformité réglementaire :

\begin{itemize}
\item Réduction des erreurs de modèle pour le calcul des fonds propres
\item Amélioration de la traçabilité et reproductibilité des calibrations
\item Capacité accrue à documenter la stabilité des modèles
\item Possibilité d'analyses de sensibilité plus approfondies
\end{itemize}

\section{Limitations et perspectives d'amélioration}

\subsection{Limitations identifiées}

Malgré les performances remarquables, plusieurs limitations méritent attention :

\begin{enumerate}
\item \textbf{Dépendance aux données d'entraînement} : Les performances se dégradent pour des configurations de paramètres très éloignées de l'ensemble d'entraînement
\item \textbf{Interprétabilité limitée} : La nature "boîte noire" du réseau complique l'analyse des échecs de calibration
\item \textbf{Coût initial d'entraînement} : La génération des données et l'entraînement initial requièrent un investissement computationnel substantiel
\item \textbf{Maintenance du modèle} : Les changements de régime de marché peuvent nécessiter un réentraînement périodique
\end{enumerate}

\subsection{Voies d'amélioration}

Plusieurs axes d'amélioration ont été identifiés :

\begin{enumerate}
\item \textbf{Apprentissage adaptatif} : Implémentation de techniques d'apprentissage en ligne pour adapter le modèle aux nouvelles conditions de marché
\item \textbf{Uncertainty quantification} : Intégration de méthodes bayésiennes pour quantifier l'incertitude des prédictions
\item \textbf{Architecture améliorée} : Exploration d'architectures alternatives (Transformers, ResNets) pour capturer des patterns plus complexes
\item \textbf{Multi-objectifs} : Extension à l'optimisation simultanée de multiple critères (précision, stabilité, interprétabilité)
\end{enumerate}

\section{Conclusion}

Les résultats présentés dans ce chapitre démontrent de manière convaincante l'efficacité de l'approche de calibration par Deep Learning pour le modèle de Heston. L'accélération computationnelle d'un facteur 11x, combinée à une amélioration de la précision et de la robustesse, ouvre de nouvelles perspectives pour la gestion des risques et le pricing d'options en temps réel.

L'évaluation sur données réelles SPX Weekly confirme la viabilité pratique de cette approche, même dans des conditions de marché volatiles. La capacité à maintenir des performances supérieures à travers différents régimes de marché témoigne de la robustesse de la méthodologie.

Ces résultats positionnent l'approche de Deep Learning comme une alternative crédible et supérieure aux méthodes traditionnelles de calibration, avec des implications significatives pour l'industrie financière. L'adoption de ces techniques pourrait transformer la manière dont les institutions financières abordent la calibration de modèles et la gestion des risques de marché.

Les perspectives d'amélioration identifiées suggèrent un potentiel d'optimisation supplémentaire, promettant des développements futurs encore plus performants. L'investissement dans ces technologies représente donc un avantage concurrentiel durable pour les institutions qui sauront les maîtriser et les déployer efficacement.

\chapter{Conclusions}

\section{Synthèse des contributions principales}

Cette recherche a démontré de manière convaincante l'efficacité et la viabilité de l'application du Deep Learning à la calibration du modèle de Heston. L'implémentation de l'approche en deux étapes développée par Bayer et Stemper a produit des résultats remarquables qui transforment fondamentalement notre compréhension des possibilités de la calibration moderne de modèles de volatilité stochastique.

L'objectif principal de cette étude était d'évaluer dans quelle mesure un réseau de Deep Learning peut remplacer ou accélérer la calibration traditionnelle du modèle de Heston sur des données réelles. Les résultats obtenus apportent une réponse définitive et positive à cette question centrale. L'accélération computationnelle d'un facteur 11x, combinée à une amélioration systématique de la précision et de la robustesse, établit cette approche comme une alternative supérieure aux méthodes traditionnelles.

La méthodologie développée résout efficacement les limitations fondamentales de la calibration traditionnelle. Le remplacement des évaluations coûteuses de Monte Carlo par des évaluations rapides de réseaux de neurones élimine le goulot d'étranglement computationnel qui limitait jusqu'alors l'application pratique de calibrations fréquentes en environnement de trading.

\section{Implications théoriques et méthodologiques}

Cette recherche contribue significativement à l'avancement de la théorie de la calibration de modèles financiers. L'approche en deux étapes proposée par Bayer et Stemper, validée empiriquement dans cette étude, représente un paradigme nouveau qui réconcilie l'efficacité du Deep Learning avec la robustesse des algorithmes d'optimisation établis.

La démonstration que les réseaux de neurones peuvent approximer fidèlement la fonction de mapping complexe des paramètres de Heston vers les volatilités implicites ouvre des perspectives considérables pour l'application de techniques similaires à d'autres modèles de finance quantitative. Cette universalité méthodologique suggère que l'approche pourrait être étendue aux modèles SABR, aux modèles de volatilité locale stochastique, et potentiellement aux modèles de volatilité rugueuse.

L'analyse de robustesse révèle que l'approche neuronale présente une stabilité supérieure aux méthodes traditionnelles, particulièrement dans les régions extrêmes de l'espace des paramètres. Cette caractéristique revêt une importance critique pour les applications de gestion des risques, où la fiabilité du modèle dans des conditions de marché exceptionnelles constitue un prérequis fondamental.

\section{Impact pratique pour l'industrie financière}

Les implications pratiques de cette recherche s'étendent bien au-delà de la simple accélération computationnelle. La possibilité de calibrer le modèle de Heston en temps quasi-réel transforme les approches possibles de gestion des risques et de pricing d'options.

L'industrie financière bénéficiera immédiatement de plusieurs avantages opérationnels. La réduction drastique des temps de calcul permet l'implémentation de calibrations intra-day, améliorant significativement la réactivité aux changements de conditions de marché. Les institutions financières peuvent désormais envisager des stratégies de gestion des risques plus dynamiques, avec des recalibrations automatiques déclenchées par des événements de marché spécifiques.

L'amélioration de la précision et de la stabilité présente des bénéfices directs pour la conformité réglementaire. Les erreurs de modèle réduites se traduisent par des calculs de fonds propres plus précis, tandis que la stabilité accrue facilite la documentation et la validation des modèles internes requis par la réglementation Bâle III.

L'aspect économique mérite une attention particulière. La réduction d'un facteur 11x des besoins computationnels se traduit par des économies substantielles en infrastructure IT. Pour les institutions utilisant des services cloud pour leurs calculs de risque, cette réduction représente des économies opérationnelles immédiates de près de 90\% sur les coûts de calcul.

\section{Contributions à la littérature académique}

Cette étude enrichit la littérature académique sur plusieurs dimensions importantes. Premièrement, elle fournit une validation empirique rigoureuse de l'approche théorique développée par Bayer et Stemper, comblant un gap important entre les développements méthodologiques et leur application pratique.

L'analyse comparative exhaustive avec les méthodes traditionnelles établit des benchmarks de référence pour les recherches futures. Les métriques de performance détaillées et les analyses de robustesse constituent une contribution méthodologique valuable pour les chercheurs travaillant sur des problèmes similaires.

La validation sur données de marché réelles, couvrant différents régimes de volatilité incluant la crise COVID-19, apporte une dimension empirique cruciale souvent absente des études purement théoriques. Cette validation démontre la pertinence pratique de l'approche dans des conditions de marché réelles et volatiles.

L'extension de l'analyse aux grecques et sensibilités comble une lacune importante dans la littérature existante. La démonstration que l'approche neuronale maintient une précision élevée pour le calcul des sensibilités valide son applicabilité pour les besoins complets de gestion des risques.

\section{Limitations et perspectives critiques}

Malgré les résultats remarquables obtenus, cette recherche présente certaines limitations qu'il convient de reconnaître et d'adresser dans les développements futurs.

La dépendance aux données d'entraînement constitue une limitation fondamentale de l'approche neuronale. Les performances se dégradent pour des configurations de paramètres très éloignées de l'ensemble d'entraînement, créant un risque de model breakdown dans des conditions de marché exceptionnelles non anticipées lors de l'entraînement.

L'aspect "boîte noire" des réseaux de neurones pose des défis pour l'interprétabilité et la validation réglementaire. Les institutions financières doivent développer des frameworks appropriés pour expliquer et justifier les décisions basées sur des modèles neuronaux, particulièrement dans le contexte des exigences réglementaires de transparence.

Le coût initial d'implémentation ne doit pas être sous-estimé. La génération des données d'entraînement, l'entraînement des modèles et le développement de l'infrastructure technique requièrent un investissement substantiel en ressources humaines et technologiques.

La maintenance et l'évolution des modèles neuronaux présentent des défis spécifiques. Les changements de régime de marché peuvent nécessiter des réentraînements périodiques, créant des coûts opérationnels récurrents et des risques de discontinuité de service.

\section{Recommandations pour l'implémentation pratique}

Basé sur les résultats de cette recherche, plusieurs recommandations émergent pour les institutions souhaitant implémenter cette approche.

L'adoption progressive constitue la stratégie recommandée. Les institutions devraient commencer par des applications pilotes sur des portefeuilles limités, permettant de valider l'approche et de développer l'expertise nécessaire avant un déploiement à grande échelle.

L'investissement dans des systèmes de monitoring et de validation est crucial. Les institutions doivent développer des capacités de surveillance continue des performances des modèles neuronaux, avec des mécanismes automatiques de détection de dégradation des performances.

La formation du personnel technique et de gestion des risques représente un facteur critique de succès. L'implémentation effective nécessite une compréhension approfondie des principes du Deep Learning et de leurs implications pour la gestion des risques financiers.

Le développement de frameworks de gouvernance adaptés est essentiel. Les institutions doivent établir des procédures claires pour la validation, la documentation et la maintenance des modèles neuronaux, en conformité avec les exigences réglementaires.

\section{Perspectives de recherche future}

Cette recherche ouvre plusieurs directions prometteuses pour les développements futurs.

L'extension à d'autres modèles de volatilité stochastique constitue une voie naturelle d'expansion. L'application de méthodologies similaires aux modèles SABR, Bates ou aux modèles de volatilité rugueuse pourrait généraliser l'approche à l'ensemble des modèles utilisés en pratique.

Le développement de techniques d'apprentissage adaptatif présente un potentiel considérable. L'implémentation d'approches d'apprentissage en ligne permettrait aux modèles de s'adapter automatiquement aux changements de conditions de marché, réduisant les besoins de réentraînement manuel.

L'intégration de l'uncertainty quantification représente une direction de recherche importante. Le développement d'approches bayésiennes pour quantifier l'incertitude des prédictions neuronales améliorerait significativement la robustesse et l'interprétabilité des modèles.

L'exploration d'architectures neuronales avancées offre des perspectives d'amélioration. L'application des Transformers, des Graph Neural Networks ou des Physics-Informed Neural Networks pourrait capturer des patterns plus complexes dans la dynamique de calibration.

Le développement d'approches multi-objectifs constitue une extension naturelle. L'optimisation simultanée de critères multiples incluant la précision, la stabilité, l'interprétabilité et la vitesse pourrait conduire à des solutions plus équilibrées pour les applications pratiques.

\section{Impact à long terme et transformation de l'industrie}

Les implications à long terme de cette recherche s'étendent au-delà de la calibration de modèles spécifiques. Elle s'inscrit dans une transformation plus large de l'industrie financière vers une digitalisation accrue et une adoption généralisée de l'intelligence artificielle.

L'accélération de la calibration de modèles représente un exemple emblématique de la manière dont le Deep Learning peut transformer les processus fondamentaux de la finance quantitative. Cette transformation ouvre la voie à des applications plus ambitieuses, incluant l'optimisation de portefeuilles en temps réel, la gestion dynamique des risques et le pricing adaptatif de produits complexes.

Les institutions qui maîtriseront ces technologies bénéficieront d'avantages concurrentiels durables. La capacité à calibrer et recalibrer les modèles rapidement et précisément devient un facteur différenciant crucial dans un environnement de marché de plus en plus volatil et compétitif.

Cette évolution s'accompagne nécessairement de changements organisationnels profonds. Les équipes de gestion des risques doivent intégrer des compétences en science des données et Machine Learning, tandis que les processus de validation et de gouvernance doivent évoluer pour accommoder les spécificités des modèles d'apprentissage automatique.

\section{Conclusion générale}

Cette recherche démontre de manière convaincante que l'application du Deep Learning à la calibration du modèle de Heston représente une avancée majeure qui transforme fondamentalement les possibilités de la finance quantitative moderne. L'accélération computationnelle spectaculaire, combinée à une amélioration de la précision et de la robustesse, établit cette approche comme la nouvelle référence pour la calibration de modèles de volatilité stochastique.

Les résultats obtenus répondent définitivement à la question de recherche initiale : un réseau de Deep Learning peut non seulement remplacer la calibration traditionnelle du modèle de Heston, mais la surpasser sur tous les critères de performance pertinents. Cette conclusion revêt une importance considérable pour l'industrie financière, ouvrant la voie à des applications pratiques qui étaient jusqu'alors impossible en raison des contraintes computationnelles.

L'impact de cette recherche s'étend bien au-delà de la contribution technique immédiate. Elle illustre le potentiel transformateur du Deep Learning pour résoudre des problèmes fondamentaux de la finance quantitative, encourageant des développements similaires dans d'autres domaines de la modélisation financière.

Les institutions financières qui adopteront ces nouvelles approches bénéficieront d'avantages compétitifs substantiels, leur permettant de réagir plus rapidement aux changements de marché et de gérer les risques avec une précision accrue. Cette transformation technologique représente donc un enjeu stratégique majeur pour l'avenir de l'industrie financière.

Finalement, cette recherche contribue à établir les fondements d'une nouvelle génération de modèles financiers qui combinent la rigueur théorique de la finance quantitative avec la puissance computationnelle de l'intelligence artificielle moderne. Cette convergence promet de révolutionner la manière dont nous abordons la modélisation, la calibration et la gestion des risques financiers dans les années à venir.

L'avenir de la finance quantitative sera probablement caractérisé par une intégration croissante de ces technologies, transformant des processus qui semblaient figés depuis des décennies. Cette recherche contribue à tracer la voie de cette transformation, démontrant qu'il est possible de concilier innovation technologique et rigueur scientifique pour créer des solutions supérieures aux approches traditionnelles.

Le message principal de cette étude est clair : l'adoption du Deep Learning pour la calibration de modèles financiers n'est plus une question de recherche académique, mais une nécessité pratique pour les institutions qui souhaitent maintenir leur compétitivité dans un environnement technologique en évolution rapide. Les résultats présentés fournissent la validation empirique nécessaire pour encourager cette adoption et ouvrent la voie à des développements encore plus ambitieux dans le futur.


% Bibliographie
\newpage
\bibliographystyle{apalike}
\bibliography{references}
\addcontentsline{toc}{chapter}{Bibliographie}

% Annexes
\newpage
\appendix
\chapter{Annexes}

\section{Détails techniques de l'implémentation}

\subsection{Architecture détaillée du réseau de neurones}

Cette section présente les spécifications techniques complètes de l'architecture du réseau de neurones développée pour l'approximation de la fonction de pricing du modèle de Heston.

\subsubsection{Configuration des couches}

\begin{table}[H]
\centering
\caption{Architecture détaillée du réseau de neurones}
\begin{tabular}{@{}llccc@{}}
\toprule
\textbf{Couche} & \textbf{Type} & \textbf{Taille} & \textbf{Activation} & \textbf{Régularisation} \\
\midrule
Input & Dense & 7 & - & - \\
Hidden 1 & Dense & 256 & ReLU & Dropout (0.2) \\
BatchNorm 1 & Batch Normalization & 256 & - & - \\
Hidden 2 & Dense & 512 & ReLU & Dropout (0.2) \\
BatchNorm 2 & Batch Normalization & 512 & - & - \\
Hidden 3 & Dense & 256 & ReLU & Dropout (0.1) \\
BatchNorm 3 & Batch Normalization & 256 & - & - \\
Hidden 4 & Dense & 128 & ReLU & Dropout (0.1) \\
Output & Dense & 1 & Linear & - \\
\bottomrule
\end{tabular}
\end{table}

\subsubsection{Paramètres d'entraînement}

\begin{table}[H]
\centering
\caption{Hyperparamètres d'entraînement}
\begin{tabular}{@{}lc@{}}
\toprule
\textbf{Paramètre} & \textbf{Valeur} \\
\midrule
Learning Rate initial & 0.001 \\
Scheduler & ReduceLROnPlateau \\
Patience & 10 \\
Factor de réduction & 0.5 \\
Optimizer & Adam \\
Beta1 & 0.9 \\
Beta2 & 0.999 \\
Weight Decay & 1e-5 \\
Batch Size & 1024 \\
Épochs maximum & 1000 \\
Early Stopping patience & 20 \\
\bottomrule
\end{tabular}
\end{table}

\subsection{Algorithme de calibration hybride}

\subsubsection{Pseudocode de l'algorithme principal}

\begin{algorithm}[H]
\caption{Calibration hybride Heston-Deep Learning}
\begin{algorithmic}
\STATE \textbf{Input:} Surface de volatilité observée $\Sigma_{market}$, réseau pré-entraîné $\mathcal{N}_\phi$
\STATE \textbf{Output:} Paramètres calibrés $\hat{\theta} = (\hat{\kappa}, \hat{\theta}, \hat{\sigma}, \hat{\rho}, \hat{v_0})$

\STATE // Initialisation des paramètres
\STATE $\theta_0 \leftarrow$ initialisation_smart($\Sigma_{market}$)
\STATE $bounds \leftarrow$ contraintes_heston()

\STATE // Définition de la fonction objectif
\FUNCTION{objective}{$\theta$}
    \STATE $\Sigma_{model} \leftarrow \mathcal{N}_\phi(\theta, M, T)$ // Évaluation rapide via réseau
    \STATE $error \leftarrow \sum_{i,j} w_{i,j} (\Sigma_{market}[i,j] - \Sigma_{model}[i,j])^2$
    \RETURN $error$
\ENDFUNCTION

\STATE // Optimisation par Levenberg-Marquardt
\STATE $\hat{\theta} \leftarrow$ LevenbergMarquardt(objective, $\theta_0$, bounds)

\STATE // Validation optionnelle avec pricing exact
\IF{validation\_requise}
    \STATE $\Sigma_{exact} \leftarrow$ HestonPricing($\hat{\theta}$, M, T)
    \STATE $validation\_error \leftarrow$ compute\_error($\Sigma_{exact}$, $\Sigma_{market}$)
    \IF{$validation\_error > threshold$}
        \STATE $\hat{\theta} \leftarrow$ fallback\_traditional\_calibration($\Sigma_{market}$)
    \ENDIF
\ENDIF

\RETURN $\hat{\theta}$
\end{algorithmic}
\end{algorithm}

\subsubsection{Fonction d'initialisation intelligente}

\begin{algorithm}[H]
\caption{Initialisation intelligente des paramètres}
\begin{algorithmic}
\FUNCTION{initialisation\_smart}{$\Sigma_{market}$}
    \STATE // Estimation de la volatilité moyenne
    \STATE $\bar{\sigma} \leftarrow$ moyenne($\Sigma_{market}$)
    \STATE $v_0 \leftarrow \bar{\sigma}^2$
    
    \STATE // Estimation du niveau long terme
    \STATE $\theta \leftarrow$ volatilité\_long\_terme($\Sigma_{market}$)
    
    \STATE // Estimation de la vitesse de retour
    \STATE $\kappa \leftarrow$ 2.0  // valeur conservatrice
    
    \STATE // Estimation de la corrélation par le skew
    \STATE $skew \leftarrow$ compute\_skew($\Sigma_{market}$)
    \STATE $\rho \leftarrow -0.3 \times \tanh(skew \times 2)$
    
    \STATE // Estimation de la vol de vol
    \STATE $\sigma \leftarrow$ 0.3  // valeur typique de marché
    
    \RETURN $(\kappa, \theta, \sigma, \rho, v_0)$
\ENDFUNCTION
\end{algorithmic}
\end{algorithm}

\section{Données et preprocessing}

\subsection{Structure des données SPX Weekly}

\subsubsection{Caractéristiques du dataset}

\begin{table}[H]
\centering
\caption{Statistiques descriptives des données SPX Weekly}
\begin{tabular}{@{}lcccc@{}}
\toprule
\textbf{Variable} & \textbf{Minimum} & \textbf{Maximum} & \textbf{Moyenne} & \textbf{Écart-type} \\
\midrule
Moneyness & 0.75 & 1.25 & 1.002 & 0.089 \\
Time to Maturity (jours) & 1 & 60 & 18.7 & 12.4 \\
Volatilité Implicite & 0.08 & 0.95 & 0.234 & 0.087 \\
Prix sous-jacent & 2,191 & 4,793 & 3,412 & 634 \\
VIX contemporain & 12.4 & 82.7 & 24.1 & 9.8 \\
\bottomrule
\end{tabular}
\end{table}

\subsubsection{Distribution temporelle des observations}

\begin{table}[H]
\centering
\caption{Répartition des observations par période}
\begin{tabular}{@{}lccc@{}}
\toprule
\textbf{Période} & \textbf{Nombre d'obs.} & \textbf{VIX moyen} & \textbf{Caractéristiques} \\
\midrule
Jan-Mar 2020 & 23,456 & 31.2 & Début crise COVID \\
Apr-Dec 2020 & 87,234 & 28.7 & Volatilité élevée soutenue \\
2021 & 78,912 & 19.4 & Retour à la normale \\
2022 & 89,567 & 26.8 & Tensions inflationnistes \\
\bottomrule
\end{tabular}
\end{table}

\subsection{Procédures de nettoyage des données}

\subsubsection{Filtres de qualité appliqués}

\begin{enumerate}
\item \textbf{Filtre de liquidité} : Élimination des options avec bid-ask spread > 10\% du mid-price
\item \textbf{Filtre d'arbitrage} : Vérification de la monotonicité par strike et maturité
\item \textbf{Filtre de volatilité} : Exclusion des IV < 5\% ou > 100\%
\item \textbf{Filtre de maturité} : Conservation uniquement des maturités 1-60 jours
\item \textbf{Filtre de moneyness} : Restriction à la plage [0.75, 1.25]
\end{enumerate}

\subsubsection{Traitement des valeurs aberrantes}

\begin{algorithm}[H]
\caption{Détection et traitement des outliers}
\begin{algorithmic}
\FUNCTION{clean\_outliers}{$data$, $threshold = 3.0$}
    \FOR{chaque date $t$}
        \STATE $surface_t \leftarrow$ extraire\_surface($data$, $t$)
        
        \STATE // Détection par z-score
        \STATE $z\_scores \leftarrow$ compute\_zscore($surface_t$)
        \STATE $outliers \leftarrow$ find\_outliers($z\_scores$, $threshold$)
        
        \STATE // Traitement par interpolation
        \FOR{chaque outlier $(m_i, T_j)$}
            \STATE $IV_{i,j} \leftarrow$ interpolate\_bivariate($surface_t$, $m_i$, $T_j$)
        \ENDFOR
        
        \STATE // Validation d'arbitrage
        \STATE valider\_absence\_arbitrage($surface_t$)
    \ENDFOR
    
    \RETURN $data_{clean}$
\ENDFUNCTION
\end{algorithmic}
\end{algorithm}

\section{Métriques de performance détaillées}

\subsection{Définitions des métriques}

\subsubsection{Erreurs de pricing}

\begin{align}
RMSE_{IV} &= \sqrt{\frac{1}{N} \sum_{i=1}^{N} (IV_i^{market} - IV_i^{model})^2} \\
MAE_{IV} &= \frac{1}{N} \sum_{i=1}^{N} |IV_i^{market} - IV_i^{model}| \\
MAPE_{IV} &= \frac{1}{N} \sum_{i=1}^{N} \frac{|IV_i^{market} - IV_i^{model}|}{IV_i^{market}} \times 100\%
\end{align}

\subsubsection{Erreurs de paramètres}

\begin{align}
RMSE_{\theta} &= \sqrt{\frac{1}{5} \sum_{j=1}^{5} \left(\frac{\theta_j^{true} - \theta_j^{est}}{\theta_j^{true}}\right)^2} \\
Bias_j &= \frac{1}{N} \sum_{i=1}^{N} (\theta_{j,i}^{est} - \theta_{j,i}^{true})
\end{align}

\subsection{Tests statistiques de validation}

\subsubsection{Test de Diebold-Mariano}

Le test de Diebold-Mariano compare la précision prédictive de deux méthodes de calibration :

\begin{align}
DM &= \frac{\bar{d}}{\sqrt{Var(\bar{d})}} \\
\bar{d} &= \frac{1}{N} \sum_{i=1}^{N} d_i \\
d_i &= L(e_{1,i}) - L(e_{2,i})
\end{align}

où $L(\cdot)$ est une fonction de perte et $e_{j,i}$ les erreurs de la méthode $j$ pour l'observation $i$.

\subsubsection{Résultats des tests statistiques}

\begin{table}[H]
\centering
\caption{Tests de significativité des améliorations}
\begin{tabular}{@{}lccc@{}}
\toprule
\textbf{Test} & \textbf{Statistique} & \textbf{p-value} & \textbf{Conclusion} \\
\midrule
Diebold-Mariano (RMSE) & -4.23 & < 0.001 & DL significativement meilleur \\
Diebold-Mariano (MAE) & -3.87 & < 0.001 & DL significativement meilleur \\
Wilcoxon signed-rank & -8,234 & < 0.001 & DL significativement meilleur \\
Test de Kolmogorov-Smirnov & 0.089 & < 0.001 & Distributions différentes \\
\bottomrule
\end{tabular}
\end{table}

\section{Analyse de sensibilité}

\subsection{Sensibilité aux hyperparamètres}

\subsubsection{Impact de la taille du réseau}

\begin{table}[H]
\centering
\caption{Performance vs taille du réseau}
\begin{tabular}{@{}lcccc@{}}
\toprule
\textbf{Architecture} & \textbf{Paramètres} & \textbf{RMSE} & \textbf{Temps training} & \textbf{Temps inference} \\
\midrule
[128, 256, 128] & 98K & $1.8 \times 10^{-3}$ & 45 min & 0.08 ms \\
[256, 512, 256] & 394K & $1.3 \times 10^{-3}$ & 78 min & 0.15 ms \\
[512, 1024, 512] & 1.5M & $1.2 \times 10^{-3}$ & 156 min & 0.31 ms \\
[256, 512, 256, 128] & 527K & $1.2 \times 10^{-3}$ & 89 min & 0.18 ms \\
\bottomrule
\end{tabular}
\end{table}

\subsubsection{Impact du learning rate}

\begin{table}[H]
\centering
\caption{Sensibilité au learning rate}
\begin{tabular}{@{}lccc@{}}
\toprule
\textbf{Learning Rate} & \textbf{Convergence} & \textbf{RMSE final} & \textbf{Épochs nécessaires} \\
\midrule
0.01 & Instable & - & - \\
0.005 & Oscillations & $2.1 \times 10^{-3}$ & 1000+ \\
0.001 & Stable & $1.3 \times 10^{-3}$ & 347 \\
0.0005 & Lente & $1.2 \times 10^{-3}$ & 876 \\
0.0001 & Très lente & $1.4 \times 10^{-3}$ & 1000+ \\
\bottomrule
\end{tabular}
\end{table}

\subsection{Robustesse aux conditions de marché}

\subsubsection{Performance par quintile de volatilité}

\begin{table}[H]
\centering
\caption{Performance par niveau de volatilité (VIX)}
\begin{tabular}{@{}lcccc@{}}
\toprule
\textbf{Quintile VIX} & \textbf{Plage VIX} & \textbf{RMSE trad.} & \textbf{RMSE DL} & \textbf{Amélioration} \\
\midrule
Q1 (faible) & [12.4, 17.8] & 0.0189 & 0.0165 & -12.7\% \\
Q2 & [17.8, 21.3] & 0.0223 & 0.0194 & -13.0\% \\
Q3 & [21.3, 26.1] & 0.0267 & 0.0231 & -13.5\% \\
Q4 & [26.1, 34.7] & 0.0334 & 0.0289 & -13.5\% \\
Q5 (élevé) & [34.7, 82.7] & 0.0445 & 0.0387 & -13.0\% \\
\bottomrule
\end{tabular}
\end{table}

\section{Code source des fonctions principales}

\subsection{Fonction de pricing Heston}

\begin{lstlisting}[language=Python, caption=Implémentation du pricing Heston]
import numpy as np
from scipy.integrate import quad
from scipy.optimize import brentq

def heston_characteristic_function(phi, S0, v0, kappa, theta, sigma, rho, T, r):
    """
    Fonction caractéristique du modèle de Heston
    """
    xi = kappa - 1j * rho * sigma * phi
    d = np.sqrt(xi**2 + sigma**2 * (1j * phi + phi**2))
    
    A1 = 1j * phi * (np.log(S0) + r * T)
    A2 = (kappa * theta) / (sigma**2) * ((xi - d) * T - 2 * np.log((1 - g * np.exp(-d * T)) / (1 - g)))
    A3 = (v0 / sigma**2) * (xi - d) * (1 - np.exp(-d * T)) / (1 - g * np.exp(-d * T))
    
    g = (xi - d) / (xi + d)
    
    return np.exp(A1 + A2 + A3)

def heston_price_fft(S0, K, T, r, v0, kappa, theta, sigma, rho, option_type='call'):
    """
    Pricing d'option européenne sous Heston par FFT
    """
    N = 4096  # Nombre de points pour la FFT
    alpha = 1.5  # Paramètre de régularisation
    eta = 0.25   # Pas de discrétisation
    
    # Grille de log-strikes
    lambda_val = 2 * np.pi / (N * eta)
    b = lambda_val * N / 2
    ks = -b + lambda_val * np.arange(N)
    
    # Fonction intégrande modifiée
    def integrand(v, k):
        phi = v - 1j * (alpha + 1)
        cf = heston_characteristic_function(phi, S0, v0, kappa, theta, sigma, rho, T, r)
        return np.real(np.exp(-1j * v * k) * cf) / (alpha**2 + alpha - v**2 + 1j * (2 * alpha + 1) * v)
    
    # Calcul des prix par FFT
    integrand_values = np.array([quad(lambda v: integrand(v, k), 0, 100)[0] for k in ks])
    
    fft_values = np.fft.fft(integrand_values)
    option_values = np.exp(-alpha * ks) / np.pi * np.real(fft_values)
    
    # Interpolation pour obtenir le prix au strike désiré
    log_K = np.log(K)
    return np.interp(log_K, ks, option_values) * np.exp(-r * T)

def heston_implied_volatility(S0, K, T, r, v0, kappa, theta, sigma, rho, option_type='call'):
    """
    Calcul de la volatilité implicite sous Heston
    """
    price = heston_price_fft(S0, K, T, r, v0, kappa, theta, sigma, rho, option_type)
    
    def objective(vol):
        bs_price = black_scholes_price(S0, K, T, r, vol, option_type)
        return bs_price - price
    
    try:
        iv = brentq(objective, 0.001, 3.0)
        return iv
    except:
        return np.nan
\end{lstlisting}

\subsection{Classe du réseau de neurones}

\begin{lstlisting}[language=Python, caption=Architecture du réseau de neurones]
import torch
import torch.nn as nn
import torch.nn.functional as F

class HestonPricingNetwork(nn.Module):
    def __init__(self, input_dim=7, hidden_dims=[256, 512, 256, 128], dropout_rates=[0.2, 0.2, 0.1, 0.1]):
        super(HestonPricingNetwork, self).__init__()
        
        # Construction des couches
        self.layers = nn.ModuleList()
        self.batch_norms = nn.ModuleList()
        self.dropouts = nn.ModuleList()
        
        dims = [input_dim] + hidden_dims
        
        for i in range(len(dims) - 1):
            self.layers.append(nn.Linear(dims[i], dims[i+1]))
            self.batch_norms.append(nn.BatchNorm1d(dims[i+1]))
            self.dropouts.append(nn.Dropout(dropout_rates[i]))
        
        # Couche de sortie
        self.output_layer = nn.Linear(hidden_dims[-1], 1)
        
        # Initialisation des poids
        self._initialize_weights()
    
    def _initialize_weights(self):
        for layer in self.layers:
            nn.init.xavier_uniform_(layer.weight)
            nn.init.constant_(layer.bias, 0)
        
        nn.init.xavier_uniform_(self.output_layer.weight)
        nn.init.constant_(self.output_layer.bias, 0)
    
    def forward(self, x):
        for i, (layer, bn, dropout) in enumerate(zip(self.layers, self.batch_norms, self.dropouts)):
            x = layer(x)
            x = bn(x)
            x = F.relu(x)
            x = dropout(x)
        
        x = self.output_layer(x)
        return torch.sigmoid(x)  # Contrainte de positivité pour IV

class HestonCalibrator:
    def __init__(self, model_path):
        self.device = torch.device('cuda' if torch.cuda.is_available() else 'cpu')
        self.network = HestonPricingNetwork()
        self.network.load_state_dict(torch.load(model_path, map_location=self.device))
        self.network.eval()
        self.network.to(self.device)
    
    def predict_iv_surface(self, params, moneyness_grid, maturity_grid):
        """
        Prédiction de la surface de volatilité implicite
        """
        kappa, theta, sigma, rho, v0 = params
        
        # Création de la grille d'entrée
        M, T = np.meshgrid(moneyness_grid, maturity_grid)
        inputs = []
        
        for i in range(len(maturity_grid)):
            for j in range(len(moneyness_grid)):
                inputs.append([kappa, theta, sigma, rho, v0, M[i,j], T[i,j]])
        
        inputs = torch.tensor(inputs, dtype=torch.float32).to(self.device)
        
        with torch.no_grad():
            predictions = self.network(inputs)
        
        iv_surface = predictions.cpu().numpy().reshape(len(maturity_grid), len(moneyness_grid))
        return iv_surface
    
    def calibrate(self, market_iv_surface, moneyness_grid, maturity_grid, 
                  initial_guess=None, bounds=None):
        """
        Calibration des paramètres de Heston
        """
        if initial_guess is None:
            initial_guess = [2.0, 0.04, 0.3, -0.7, 0.04]
        
        if bounds is None:
            bounds = [(0.1, 5.0), (0.01, 0.5), (0.05, 1.0), (-0.9, 0.1), (0.01, 0.5)]
        
        def objective(params):
            pred_surface = self.predict_iv_surface(params, moneyness_grid, maturity_grid)
            error = np.mean((market_iv_surface - pred_surface)**2)
            return error
        
        from scipy.optimize import minimize
        result = minimize(objective, initial_guess, bounds=bounds, method='L-BFGS-B')
        
        return result.x, result.fun, result.success
\end{lstlisting}

\section{Résultats détaillés des expériences}

\subsection{Matrices de confusion pour la validation}

\begin{table}[H]
\centering
\caption{Classification des succès de calibration par méthode}
\begin{tabular}{@{}lcccc@{}}
\toprule
\textbf{Méthode} & \textbf{Succès} & \textbf{Échecs} & \textbf{Timeouts} & \textbf{Taux succès} \\
\midrule
Levenberg-Marquardt traditionnel & 873 & 89 & 38 & 87.3\% \\
Deep Learning hybride & 984 & 12 & 4 & 98.4\% \\
\bottomrule
\end{tabular}
\end{table}

\subsection{Distribution des temps d'exécution}

\begin{table}[H]
\centering
\caption{Percentiles des temps d'exécution (secondes)}
\begin{tabular}{@{}lcccccc@{}}
\toprule
\textbf{Méthode} & \textbf{P5} & \textbf{P25} & \textbf{P50} & \textbf{P75} & \textbf{P95} & \textbf{P99} \\
\midrule
Traditionnel & 8.2 & 15.7 & 23.4 & 31.2 & 45.8 & 67.3 \\
Deep Learning & 0.9 & 1.6 & 2.1 & 2.8 & 4.1 & 6.2 \\
\bottomrule
\end{tabular}
\end{table}

Cette annexe fournit tous les détails techniques nécessaires à la reproduction et à l'extension de nos résultats. L'implémentation complète est disponible dans le repository GitHub accompagnant ce mémoire, permettant une validation indépendante de nos conclusions.


\end{document}
