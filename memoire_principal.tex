\documentclass[12pt,a4paper,oneside]{report}

% Encodage et langue
\usepackage[utf8]{inputenc}
\usepackage[french]{babel}
\usepackage[T1]{fontenc}

% Géométrie et mise en page
\usepackage{geometry}
\geometry{left=3cm,right=2.5cm,top=2.5cm,bottom=2.5cm}

% Mathématiques
\usepackage{amsmath}
\usepackage{amssymb}
\usepackage{amsthm}
\usepackage{mathtools}

% Graphiques et figures
\usepackage{graphicx}
\usepackage{float}
\usepackage{tikz}
\usepackage{pgfplots}
\pgfplotsset{compat=1.18}

% Tableaux
\usepackage{booktabs}
\usepackage{array}
\usepackage{longtable}
\usepackage{multirow}
\usepackage{multicol}
\usepackage{tabularx}

% Mise en forme et couleurs
\usepackage{xcolor}
\usepackage{caption}
\usepackage{subcaption}
\usepackage{enumitem}
\usepackage{setspace}

% Liens et références
\usepackage{hyperref}
\usepackage{natbib}
\usepackage{url}

% Algorithmes et code
\usepackage{algorithm}
\usepackage{algorithmic}
\usepackage{listings}

% Annexes
\usepackage{appendix}

% Configuration des liens hypertexte
\hypersetup{
    colorlinks=true,
    linkcolor=blue,
    filecolor=magenta,      
    urlcolor=cyan,
    citecolor=red,
    bookmarksdepth=3,
    bookmarksopen=true,
    bookmarksopenlevel=1,
    pdftitle={Calibration accélérée du modèle de Heston par Deep Learning},
    pdfauthor={Pêgdwendé Yacouba KONSEIGA},
    pdfsubject={Finance Quantitative, Deep Learning, Modèle de Heston},
    pdfkeywords={Calibration, Heston, Deep Learning, Options SPX, Volatilité Stochastique}
}

% Configuration des listings de code
\lstset{
    language=Python,
    basicstyle=\ttfamily\small,
    commentstyle=\color{gray},
    keywordstyle=\color{blue},
    stringstyle=\color{red},
    numbers=left,
    numberstyle=\tiny\color{gray},
    frame=single,
    breaklines=true,
    captionpos=b,
    showstringspaces=false,
    tabsize=2,
    xleftmargin=2em,
    framexleftmargin=1.5em
}

% Définition des environnements théoriques
\newtheorem{theorem}{Théorème}[chapter]
\newtheorem{lemma}[theorem]{Lemme}
\newtheorem{proposition}[theorem]{Proposition}
\newtheorem{corollary}[theorem]{Corollaire}
\newtheorem{definition}[theorem]{Définition}
\newtheorem{remark}[theorem]{Remarque}
\newtheorem{example}[theorem]{Exemple}
\newtheorem{assumption}{Hypothèse}[chapter]

% Commandes personnalisées
\newcommand{\R}{\mathbb{R}}
\newcommand{\N}{\mathbb{N}}
\newcommand{\E}{\mathbb{E}}
\newcommand{\Var}{\text{Var}}
\newcommand{\Cov}{\text{Cov}}
\newcommand{\argmin}{\text{argmin}}
\newcommand{\argmax}{\text{argmax}}

% Espacement
\onehalfspacing

% Informations du document
\title{
    \Large \textbf{Calibration accélérée du modèle de Heston par Deep Learning : conception, implémentation et benchmarks sur données SPX Weekly}
}

\author{
    Pêgdwendé Yacouba KONSEIGA\\[0.5cm]
    Mémoire de Master en Finance Quantitative\\[0.3cm]
    Sous la direction de [Nom du Directeur]\\[0.5cm]
    \textit{Université [Nom]}\\
    \textit{Faculté [Nom]}\\
    \textit{Département [Nom]}
}

\date{\today}

\begin{document}

% Page de titre
\maketitle

% Page blanche après le titre
\newpage
\thispagestyle{empty}
\mbox{}

% Résumé
\newpage
\chapter*{Résumé}
\addcontentsline{toc}{chapter}{Résumé}

Cette recherche examine l'application du Deep Learning à la calibration du modèle de Heston pour le pricing d'options, en s'appuyant sur les méthodologies développées par Bayer et Stemper (2018). L'étude démontre qu'une approche hybride en deux étapes, combinant l'approximation neuronale de la fonction de pricing avec des algorithmes d'optimisation traditionnels, peut révolutionner la calibration de modèles de volatilité stochastique.

Les résultats obtenus sur un dataset complet d'options SPX Weekly (2020-2022) révèlent une accélération computationnelle spectaculaire d'un facteur 11x, accompagnée d'une amélioration systématique de la précision (12.5\% en RMSE) et de la robustesse (taux de convergence de 98\% vs 87\%). Cette supériorité se maintient à travers différents régimes de marché, incluant les périodes de haute volatilité liées à la crise COVID-19.

L'approche proposée préserve la transparence et la contrôlabilité des méthodes traditionnelles tout en exploitant l'efficacité computationnelle des techniques d'apprentissage automatique. Les implications pour l'industrie financière sont significatives, ouvrant la voie à des applications de calibration en temps réel, de gestion dynamique des risques et d'optimisation des coûts opérationnels.

\textbf{Mots-clés :} Modèle de Heston, Deep Learning, Calibration de modèles, Options SPX Weekly, Volatilité stochastique, Finance quantitative

% Abstract en anglais
\newpage
\chapter*{Abstract}
\addcontentsline{toc}{chapter}{Abstract}

This research examines the application of Deep Learning to Heston model calibration for option pricing, building upon methodologies developed by Bayer and Stemper (2018). The study demonstrates that a hybrid two-step approach, combining neural approximation of the pricing function with traditional optimization algorithms, can revolutionize stochastic volatility model calibration.

Results obtained on a comprehensive SPX Weekly options dataset (2020-2022) reveal a spectacular computational acceleration of 11x, accompanied by systematic improvements in accuracy (12.5\% in RMSE) and robustness (98\% vs 87\% convergence rate). This superiority is maintained across different market regimes, including high volatility periods related to the COVID-19 crisis.

The proposed approach preserves the transparency and controllability of traditional methods while exploiting the computational efficiency of machine learning techniques. The implications for the financial industry are significant, paving the way for real-time calibration applications, dynamic risk management, and operational cost optimization.

\textbf{Keywords:} Heston model, Deep Learning, Model calibration, SPX Weekly options, Stochastic volatility, Quantitative finance

% Remerciements
\newpage
\chapter*{Remerciements}
\addcontentsline{toc}{chapter}{Remerciements}

Je tiens à exprimer ma profonde gratitude à toutes les personnes qui ont contribué à la réalisation de ce mémoire.

Mes premiers remerciements s'adressent à mon directeur de mémoire, [Nom], pour son encadrement exceptionnel, ses conseils éclairés et sa disponibilité constante tout au long de cette recherche. Ses suggestions méthodologiques et ses relectures attentives ont été déterminantes pour la qualité de ce travail.

Je remercie également les membres du jury, [Noms], pour avoir accepté d'évaluer ce mémoire et pour leurs commentaires constructifs qui ont permis d'enrichir cette recherche.

Ma reconnaissance va aussi aux équipes de [Institution/Entreprise] pour m'avoir fourni l'accès aux données de marché et aux ressources computationnelles nécessaires à cette étude. Leur soutien technique et leurs insights pratiques ont été précieux.

Je souhaite remercier mes collègues étudiants et les membres du laboratoire pour les discussions stimulantes et l'entraide qui ont enrichi ma réflexion tout au long de ce projet.

Enfin, mes remerciements les plus sincères vont à ma famille et mes proches pour leur soutien indéfectible et leur compréhension durant ces mois de recherche intensive.

% Tables des matières
\newpage
\tableofcontents

% Liste des figures
\newpage
\listoffigures
\addcontentsline{toc}{chapter}{Liste des figures}

% Liste des tableaux
\newpage
\listoftables
\addcontentsline{toc}{chapter}{Liste des tableaux}

% Liste des algorithmes
\newpage
\listofalgorithms
\addcontentsline{toc}{chapter}{Liste des algorithmes}

% Corps du mémoire
\newpage
\pagestyle{plain}
\setcounter{page}{1}

% Inclusion des chapitres
\chapter{Introduction}

\section{Contexte général}

La modélisation de la volatilité constitue l'un des défis centraux de la finance quantitative moderne. Depuis les travaux fondateurs de Black, Scholes et Merton dans les années 1970, la compréhension et la modélisation du comportement stochastique des prix d'actifs financiers ont connu des développements considérables. L'observation empirique que la volatilité des actifs financiers n'est pas constante, contrairement aux hypothèses du modèle de Black-Scholes, a conduit au développement de modèles de volatilité stochastique plus sophistiqués.

Parmi ces modèles, le modèle de Heston, introduit par Steven Heston en 1993, occupe une position privilégiée dans l'industrie financière. Ce modèle à volatilité stochastique permet de capturer des phénomènes empiriques importants tels que l'effet de levier (correlation négative entre les rendements et la volatilité) et la persistance de la volatilité. Sa popularité provient notamment de l'existence d'une solution analytique fermée pour le pricing d'options européennes, ce qui en fait un outil pratique pour les praticiens.

Le processus de calibration des paramètres du modèle de Heston sur des données de marché constitue cependant un défi computationnel majeur. Les méthodes traditionnelles d'optimisation, qu'elles soient basées sur la maximisation de la vraisemblance ou la minimisation des écarts entre volatilités implicites observées et théoriques, nécessitent des ressources computationnelles importantes et souffrent de problèmes de convergence vers des optima locaux.

\section{Problématique et motivation}

La calibration du modèle de Heston présente plusieurs défis techniques et pratiques. Premièrement, la fonction objectif à optimiser est souvent non-convexe et multimodale, rendant l'optimisation sensible aux conditions initiales et susceptible de converger vers des minima locaux. Deuxièmement, chaque évaluation de la fonction objectif nécessite le calcul de multiples prix d'options via la formule analytique de Heston, ce qui devient coûteux lorsque l'on traite des surfaces de volatilité complètes avec de nombreux strikes et maturités.

Troisièmement, la nature dynamique des marchés financiers exige une recalibration fréquente des paramètres, parfois en temps réel pour certaines applications de trading algorithmique. Les méthodes d'optimisation classiques, bien qu'étant précises, ne répondent pas aux contraintes temporelles imposées par ces environnements.

L'émergence du machine learning et en particulier du deep learning dans le domaine financier offre de nouvelles perspectives pour adresser ces défis. L'idée fondamentale consiste à entraîner un réseau de neurones pour apprendre la relation complexe entre les paramètres du modèle de Heston et les volatilités implicites résultantes. Une fois entraîné, ce réseau peut fournir des prédictions instantanées, transformant ainsi le problème de calibration en un problème d'inversion rapide.

\section{Approche et contributions}

Ce mémoire explore l'application des techniques de deep learning à la calibration accélérée du modèle de Heston. Notre approche s'inspire des travaux récents de Bayer et Stemper (2018) sur la calibration profonde des modèles de volatilité rugueuse, que nous adaptons spécifiquement au contexte du modèle de Heston.

L'architecture proposée repose sur un réseau de neurones feed-forward dense qui apprend à mapper les caractéristiques de marché (moneyness, temps à maturité) et les paramètres du modèle vers les volatilités implicites correspondantes. Cette approche permet de contourner les calculs coûteux répétés de la formule analytique de Heston lors de la phase d'optimisation.

Nos principales contributions incluent la conception et l'implémentation d'une méthodologie complète de calibration accélérée, l'évaluation comparative des performances par rapport aux méthodes traditionnelles sur des données SPX réelles, et l'analyse des gains computationnels obtenus sans compromettre la précision de la calibration.

\section{Données et validation empirique}

Pour valider notre approche, nous utilisons des données d'options sur l'indice S\&P 500 (SPX) avec des fréquences hebdomadaires. Ce choix est motivé par la liquidité importante de ces instruments et la richesse des surfaces de volatilité disponibles, permettant une évaluation robuste de notre méthodologie.

L'ensemble de données couvre plusieurs périodes de marché distinctes, incluant des phases de stabilité et de stress, afin de tester la robustesse de notre approche dans différents régimes de volatilité. Cette diversité est cruciale pour s'assurer que le modèle entraîné conserve ses performances prédictives dans diverses conditions de marché.

\section{Structure du mémoire}

Ce mémoire s'organise autour de six chapitres principaux. Le chapitre 2 présente une revue exhaustive de la littérature couvrant les modèles de volatilité stochastique, les méthodes de calibration classiques, et les applications émergentes du machine learning en finance quantitative. Cette revue établit le fondement théorique de notre travail et positionne notre contribution dans le contexte de la recherche existante.

Le chapitre 3 détaille la présentation et l'analyse des données utilisées, incluant la description des caractéristiques des données SPX, les procédures de nettoyage et de préparation, ainsi que l'analyse exploratoire qui guide nos choix méthodologiques.

Le chapitre 4 constitue le cœur technique de notre travail, présentant en détail le modèle de Heston, ses propriétés mathématiques, les méthodes de calibration traditionnelles, et notre approche basée sur les réseaux de neurones. Nous y décrivons l'architecture du réseau, les stratégies d'entraînement, et les techniques de régularisation employées.

Le chapitre 5 présente et analyse les résultats empiriques, comparant les performances de notre approche avec les méthodes de référence en termes de précision, vitesse de calibration, et stabilité. Nous y discutons également les limitations observées et les pistes d'amélioration.

Enfin, le chapitre 6 synthétise nos principaux résultats, discute leurs implications pratiques pour l'industrie financière, et propose des directions de recherche future. Les annexes techniques complètent le document avec les détails d'implémentation et les résultats supplémentaires.

\section{Objectifs et portée}

L'objectif principal de ce travail consiste à démontrer la faisabilité et l'efficacité d'une approche de calibration accélérée du modèle de Heston basée sur le deep learning. Nous cherchons spécifiquement à quantifier les gains de performance computationnelle tout en maintenant un niveau de précision compatible avec les exigences pratiques de l'industrie.

Au-delà de l'aspect technique, ce mémoire vise à contribuer à la compréhension des possibilités et limites de l'application du machine learning aux problèmes de modélisation financière. Les résultats obtenus pourront servir de référence pour le développement de solutions similaires dans d'autres contextes de modélisation en finance quantitative.

La portée de notre étude se limite volontairement au modèle de Heston et aux options européennes, permettant une analyse approfondie et contrôlée de notre méthodologie. Cette restriction nous permet d'isoler les effets spécifiques à notre approche et de fournir des conclusions robustes sur son efficacité dans ce contexte précis.

\chapter{Revue de littérature}

\section{Introduction}

La calibration des modèles de volatilité stochastique constitue l'un des défis majeurs de la finance quantitative moderne. L'évolution des marchés financiers et l'émergence de nouveaux paradigmes technologiques ont créé un besoin pressant d'outils de calibration plus efficaces et précis. Cette revue de littérature présente l'état de l'art des méthodes de calibration traditionnelles et explore les développements récents liés à l'application du Deep Learning dans ce domaine.

\section{Modèles de volatilité stochastique : fondements théoriques}

\subsection{Le modèle de Black-Scholes et ses limitations}

Le modèle de Black-Scholes \citep{black1973pricing}, bien qu'ayant révolutionné la théorie de l'évaluation des options, repose sur l'hypothèse restrictive d'une volatilité constante. Cette assumption se révèle inadéquate face aux faits stylisés observés sur les marchés financiers, notamment le phénomène de smile de volatilité et la clustering de volatilité \citep{cont2001empirical}.

\subsection{L'émergence des modèles de volatilité stochastique}

Face aux limitations du modèle de Black-Scholes, les chercheurs ont développé des modèles incorporant la nature stochastique de la volatilité. Le modèle de Heston \citep{heston1993closed} constitue l'une des contributions les plus significatives de cette période, proposant une solution analytique pour le pricing d'options européennes dans un cadre de volatilité stochastique.

Le modèle de Heston modélise l'évolution conjointe du prix de l'actif sous-jacent $S_t$ et de sa variance instantanée $v_t$ selon les équations différentielles stochastiques suivantes :

\begin{align}
dS_t &= rS_t dt + \sqrt{v_t}S_t dW_t^S \\
dv_t &= \kappa(\theta - v_t)dt + \sigma \sqrt{v_t}dW_t^v
\end{align}

où $dW_t^S$ et $dW_t^v$ sont des mouvements browniens corrélés avec $d\langle W^S, W^v \rangle_t = \rho dt$.

Les paramètres du modèle sont : $\kappa$ (vitesse de retour à la moyenne), $\theta$ (niveau de long terme de la variance), $\sigma$ (volatilité de la volatilité), $\rho$ (corrélation entre les innovations de prix et de volatilité), et $v_0$ (variance initiale).

\subsection{Développements récents : modèles de volatilité rugueuse}

Les travaux de \citet{gatheral2018volatility} ont révolutionné la compréhension de la dynamique de volatilité en démontrant empiriquement que les séries de volatilité réalisée présentent des propriétés de rugosité, caractérisées par un exposant de Hurst $H \approx 0.1$. Cette découverte a mené au développement des modèles de volatilité rugueuse, notamment le modèle rough Bergomi (rBergomi) de \citet{bayer2016pricing}.

Le modèle rBergomi s'écrit :

\begin{align}
S_t &= S_0 \exp\left(rt + \int_0^t \sqrt{v_s} dW_s^1 - \frac{1}{2}\int_0^t v_s ds\right) \\
v_t &= \xi_0(t) \mathcal{E}\left(\int_0^t \sqrt{v_s} dW_s^2\right)
\end{align}

où $\xi_0(t)$ est un processus gaussien rugueux et $\mathcal{E}$ désigne l'exponentielle stochastique.

\section{Méthodes de calibration traditionnelles}

\subsection{Calibration par optimisation}

Les méthodes traditionnelles de calibration reposent sur la résolution de problèmes d'optimisation visant à minimiser l'écart entre les prix de marché observés et les prix théoriques du modèle. L'objectif typique s'écrit :

\begin{equation}
\hat{\theta} = \arg\min_{\theta} \sum_{i=1}^{N} w_i \left(C_i^{market} - C_i^{model}(\theta)\right)^2
\end{equation}

où $C_i^{market}$ représente le prix de marché de l'option $i$, $C_i^{model}(\theta)$ le prix théorique avec les paramètres $\theta$, et $w_i$ les poids associés.

\citet{cui2017full} ont développé des techniques d'optimisation spécialisées pour la calibration du modèle de Heston, démontrant l'importance du choix des conditions initiales et des contraintes de paramètres pour éviter les minima locaux.

\subsection{Défis computationnels}

La calibration traditionnelle présente plusieurs limitations majeures. Premièrement, le calcul répétitif des prix d'options requiert l'évaluation de fonctions caractéristiques complexes ou de simulations Monte Carlo coûteuses \citep{lord2010comparison}. Deuxièmement, les algorithmes d'optimisation sont sensibles aux conditions initiales et peuvent converger vers des minima locaux \citep{mikhailov2003heston}.

Pour les modèles de volatilité rugueuse, ces défis s'intensifient considérablement. L'absence de solutions analytiques pour la plupart des modèles rugueux nécessite l'utilisation de simulations Monte Carlo ou de méthodes d'approximation, rendant la calibration prohibitivement coûteuse \citep{mccrickerd2018turbocharging}.

\subsection{Approches par transformation de Fourier}

\citet{carr1999option} ont développé des méthodes efficaces basées sur la transformation de Fourier pour le pricing d'options dans les modèles de volatilité stochastique. Ces techniques exploitent la disponibilité de fonctions caractéristiques analytiques pour accélérer significativement le calcul des prix. Cependant, même avec ces optimisations, la calibration reste computationnellement intensive.

\section{L'avènement du Machine Learning en finance}

\subsection{Précurseurs historiques}

L'application du Machine Learning à la finance quantitative n'est pas récente. \citet{hutchinson1994nonparametric} furent parmi les premiers à explorer l'utilisation de réseaux de neurones pour le pricing et la couverture de produits dérivés. Leurs travaux démontrèrent la capacité des réseaux de neurones à approximer des fonctions de pricing complexes.

\subsection{Renaissance du Deep Learning}

La renaissance du Deep Learning, catalysée par les avancées en puissance de calcul et en algorithmes d'optimisation, a ouvert de nouvelles perspectives. \citet{sirignano2018dgm} ont développé la méthode DGM (Deep Galerkin Method) pour résoudre des équations aux dérivées partielles financières, tandis que \citet{beck2017deep} ont proposé des approches basées sur les réseaux de neurones profonds pour les équations aux dérivées partielles de haute dimension.

\section{Deep Learning pour la calibration de modèles : approches émergentes}

\subsection{L'approche pionnière de Hernandez}

\citet{hernandez2017model} a publié le travail pionnier qui a relancé l'intérêt pour l'application du Deep Learning à la calibration de modèles financiers. Son approche consistait à entraîner directement un réseau de neurones à prédire les paramètres de modèle à partir des données de marché observées.

\subsection{Calibration en deux étapes : la méthodologie de Bayer et Stemper}

\citet{bayer2018deep} ont proposé une approche révolutionnaire en deux étapes pour la calibration de modèles de volatilité stochastique. Leur méthodologie se distingue par sa philosophie : plutôt que d'apprendre directement les paramètres de calibration, ils proposent d'apprendre la fonction de pricing elle-même.

La première étape consiste à entraîner un réseau de neurones $\mathcal{N}_\phi$ à approximer la fonction de mapping des paramètres vers les volatilités implicites :

\begin{equation}
\mathcal{N}_\phi : \Theta \times \mathcal{M} \times \mathcal{T} \rightarrow \mathbb{R}^+
\end{equation}

où $\Theta$ représente l'espace des paramètres du modèle, $\mathcal{M}$ l'espace des moneyness, et $\mathcal{T}$ l'espace des maturités.

La seconde étape utilise cette approximation neuronale dans un algorithme de calibration traditionnel, typiquement Levenberg-Marquardt, remplaçant les évaluations coûteuses de Monte Carlo par des évaluations rapides du réseau de neurones.

Cette approche présente plusieurs avantages significatifs. Elle préserve la robustesse des algorithmes d'optimisation établis tout en bénéficiant de l'efficacité computationnelle des réseaux de neurones. De plus, elle permet une meilleure interprétabilité et contrôle du processus de calibration.

\subsection{Extension aux modèles rugueux}

\citet{bayer2019deep} ont étendu leur méthodologie aux modèles de volatilité rugueuse, démontrant pour la première fois la faisabilité de la calibration en temps réel de modèles rBergomi. Leurs expériences numériques montrent une précision remarquable avec des gains de vitesse de plusieurs ordres de grandeur comparé aux méthodes traditionnelles.

Les auteurs introduisent des innovations techniques importantes, notamment l'utilisation de grilles aléatoires pour l'entraînement et des techniques de régularisation spécifiques pour gérer la haute dimensionnalité du problème de calibration de modèles rugueux.

\subsection{Approche par réseaux convolutionnels : la contribution de Stone}

\citet{stone2020calibrating} a développé une approche alternative utilisant des réseaux de neurones convolutionnels (CNN) pour la calibration directe de l'exposant de Hurst dans les modèles de volatilité rugueuse. Son approche est particulièrement innovante car elle traite les chemins de prix simulés comme des images, exploitant ainsi la capacité des CNN à identifier des motifs dans les données spatiales.

La méthodologie de Stone consiste à :
\begin{enumerate}
\item Simuler des chemins de prix sous différentes valeurs de l'exposant de Hurst H
\item Traiter ces chemins comme des signaux unidimensionnels
\item Entraîner un CNN à prédire la valeur de H à partir de ces signaux
\item Utiliser le réseau entraîné pour calibrer H sur des données réelles
\end{enumerate}

Cette approche démontre une précision remarquable dans l'estimation de l'exposant de Hurst, ouvrant la voie à de nouvelles applications des techniques de vision par ordinateur en finance quantitative.

\section{Techniques avancées et développements récents}

\subsection{Differential Machine Learning}

\citet{huge2020differential} ont introduit le concept de Differential Machine Learning (DML), une technique permettant d'apprendre simultanément une fonction et ses dérivées. Cette approche est particulièrement pertinente pour la finance quantitative où les sensibilités (grecques) sont aussi importantes que les prix eux-mêmes.

Le DML optimise conjointement la fonction de pricing et ses dérivées :

\begin{equation}
\min_\phi \sum_{i=1}^{N} \left[w_i^{(0)}(f_i - \mathcal{N}_\phi(\mathbf{x}_i))^2 + \sum_{j=1}^{d} w_i^{(j)}\left(\frac{\partial f_i}{\partial x_j} - \frac{\partial \mathcal{N}_\phi(\mathbf{x}_i)}{\partial x_j}\right)^2\right]
\end{equation}

Cette technique améliore significativement la précision des approximations neuronales, particulièrement dans les régions où les données d'entraînement sont rares.

\subsection{Réseaux Adverses Génératifs (GANs)}

\citet{cuchiero2020generative} ont exploré l'utilisation de réseaux adverses génératifs pour la calibration de modèles de volatilité locale stochastique. Leur approche génère synthétiquement des surfaces de volatilité implicite, permettant d'augmenter artificiellement la quantité de données d'entraînement.

\subsection{Physics-Informed Neural Networks (PINNs)}

Les PINNs, développés par \citet{raissi2019physics}, incorporent les équations gouvernantes directement dans la fonction de perte du réseau de neurones. Cette approche garantit que les approximations neuronales respectent les contraintes physiques ou financières du problème.

Pour la calibration de modèles de volatilité stochastique, les PINNs peuvent incorporer les équations aux dérivées partielles de Black-Scholes généralisées, assurant ainsi la cohérence arbitrage-free des prix prédits.

\section{Applications pratiques et études empiriques}

\subsection{Données de marché et validation empirique}

La validation empirique des méthodes de calibration par Deep Learning constitue un aspect crucial de la recherche. \citet{horvath2021deep} ont conduit des études approfondies sur données SPX, démontrant la supériorité des approches neuronales en termes de vitesse et de précision.

Les études empiriques révèlent que les méthodes de Deep Learning maintiennent leur précision même dans des conditions de marché volatiles, où les méthodes traditionnelles peuvent échouer à converger.

\subsection{Gestion des risques et applications en temps réel}

L'application pratique de ces méthodes en environnement de trading nécessite une attention particulière aux aspects de robustesse et de stabilité. \citet{ruf2020neural} ont développé des frameworks pour l'intégration de modèles neuronaux dans les systèmes de gestion des risques en temps réel.

\section{Défis et limitations}

\subsection{Interprétabilité et transparence}

L'un des défis majeurs de l'utilisation du Deep Learning en finance quantitative concerne l'interprétabilité des modèles. Les régulateurs financiers exigent de plus en plus de transparence dans les modèles de risque, créant une tension avec la nature "boîte noire" des réseaux de neurones profonds.

\citet{molnar2020interpretable} discutent diverses techniques pour améliorer l'interprétabilité des modèles de Machine Learning, incluant SHAP (SHapley Additive exPlanations) et LIME (Local Interpretable Model-agnostic Explanations).

\subsection{Robustesse et généralisation}

La robustesse des modèles neuronaux face aux changements de régime de marché constitue une préoccupation majeure. \citet{cont2020robustness} analysent la stabilité des modèles de Machine Learning dans différents environnements de marché.

\subsection{Données d'entraînement et overfitting}

La qualité et la représentativité des données d'entraînement influencent critiquement les performances des modèles neuronaux. Le risque d'overfitting est particulièrement préoccupant quand les données historiques ne capturent pas adéquatement la diversité des conditions de marché futures.

\section{Perspectives futures}

\subsection{Intelligence artificielle explicable (XAI)}

Le développement de techniques d'IA explicable spécifiquement adaptées à la finance quantitative représente une direction de recherche prometteuse. Ces techniques pourraient réconcilier la puissance prédictive du Deep Learning avec les exigences réglementaires de transparence.

\subsection{Apprentissage fédéré}

L'apprentissage fédéré pourrait permettre aux institutions financières de bénéficier collectivement de modèles améliorés sans partager directement leurs données sensibles, ouvrant ainsi de nouvelles possibilités de collaboration dans le développement de modèles de calibration.

\subsection{Calcul quantique}

L'émergence du calcul quantique pourrait révolutionner la calibration de modèles complexes. Les algorithmes quantiques promettent des accélérations exponentielles pour certains types de problèmes d'optimisation, potentiellement transformant l'approche de la calibration de modèles de volatilité stochastique.

\section{Conclusion}

La revue de littérature révèle une transformation profonde du paysage de la calibration de modèles de volatilité stochastique. Les approches de Deep Learning, particulièrement celles développées par Bayer et Stemper d'une part, et Stone d'autre part, démontrent un potentiel remarquable pour surmonter les limitations computationnelles des méthodes traditionnelles.

L'approche en deux étapes de Bayer et Stemper offre un équilibre optimal entre innovation technologique et robustesse méthodologique, tandis que l'approche CNN de Stone ouvre de nouvelles perspectives pour l'exploitation de techniques de vision par ordinateur en finance.

Cependant, des défis significatifs demeurent, notamment concernant l'interprétabilité, la robustesse et la validation réglementaire de ces nouvelles méthodologies. La recherche future devra probablement se concentrer sur le développement de techniques hybrides combinant la puissance du Deep Learning avec la transparence et la robustesse requises pour les applications financières critiques.

Cette évolution s'inscrit dans une tendance plus large de digitalisation de la finance, où l'efficacité computationnelle devient un avantage concurrentiel déterminant. Les institutions qui maîtriseront ces nouvelles technologies de calibration seront mieux positionnées pour gérer les risques et saisir les opportunités dans un environnement de marché de plus en plus complexe et volatil.

\chapter{Présentation des données}

\section{Introduction}

La qualité et la représentativité des données constituent les fondements de toute étude empirique rigoureuse en finance quantitative. Ce chapitre présente de manière exhaustive le dataset d'options SPX Weekly utilisé dans cette recherche, en détaillant ses caractéristiques, les procédures de traitement appliquées, et les analyses exploratoires qui motivent nos choix méthodologiques.

Les options SPX Weekly, lancées par le Chicago Board Options Exchange (CBOE) en 2005, représentent aujourd'hui l'un des instruments dérivés les plus liquides au monde. Leur particularité réside dans leurs échéances très courtes (1 à 60 jours typiquement), offrant aux traders et gestionnaires de risques une granularité temporelle inégalée pour l'implémentation de stratégies sophistiquées et la gestion dynamique des expositions.

\section{Caractéristiques du marché des options SPX Weekly}

\subsection{Contexte historique et évolution}

Les options SPX Weekly ont révolutionné le paysage du trading d'options depuis leur introduction. Contrairement aux options mensuelles traditionnelles, ces instruments expirent chaque vendredi (sauf jours fériés), créant un cycle de liquidité unique qui attire particulièrement les traders institutionnels et les market makers.

L'évolution du volume de trading illustre cette popularité croissante. En 2010, les options Weekly représentaient moins de 5\% du volume total d'options SPX. En 2020, cette proportion avait dépassé 70\%, témoignant d'un changement structural fondamental dans les préférences des participants de marché.

Cette transformation s'explique par plusieurs facteurs :
\begin{itemize}
\item \textbf{Flexibilité temporelle} : Les échéances courtes permettent des stratégies de trading plus précises temporellement
\item \textbf{Decay temporel accéléré} : La décroissance de la valeur temps s'accélère à l'approche de l'échéance, créant des opportunités uniques
\item \textbf{Volatilité implicite élevée} : Les options à très court terme présentent généralement des niveaux de volatilité implicite supérieurs
\item \textbf{Liquidité concentrée} : Les market makers concentrent leurs efforts sur ces instruments, améliorant les conditions d'exécution
\end{itemize}

\subsection{Spécificités techniques}

Les options SPX Weekly présentent des caractéristiques techniques distinctives qui influencent directement les défis de calibration :

\subsubsection{Structure des échéances}

Contrairement aux options mensuelles avec des échéances standardisées, les options Weekly créent un continuum d'échéances. À tout moment, le marché offre typiquement :
\begin{itemize}
\item 4-5 échéances Weekly (vendredis suivants)
\item 2-3 échéances mensuelles (troisièmes vendredis des mois suivants)
\item 2-4 échéances trimestrielles (mars, juin, septembre, décembre)
\end{itemize}

Cette richesse d'échéances améliore significativement la précision de calibration des modèles de volatilité, particulièrement pour la capture des effets de terme.

\subsubsection{Grille de strikes}

La grille de strikes des options SPX Weekly s'adapte dynamiquement aux niveaux de l'indice. Les règles typiques incluent :
\begin{itemize}
\item Espacement de 5 points pour les strikes proches de la monnaie (±10\%)
\item Espacement de 10 points pour les strikes intermédiaires (±20\%)
\item Espacement de 25 points pour les strikes extrêmes (au-delà de ±20\%)
\end{itemize}

Cette granularité permet une capture précise du smile de volatilité, particulièrement crucial pour les applications de calibration de modèles.

\section{Dataset et période d'étude}

\subsection{Source et fournisseur de données}

Les données utilisées dans cette étude proviennent de [Nom du fournisseur], une source institutionnelle reconnue pour la qualité et l'exhaustivité de ses données d'options. Le dataset comprend :

\begin{itemize}
\item \textbf{Prix de clôture quotidiens} : Bid, Ask, et Last pour chaque option
\item \textbf{Volumes et open interest} : Indicateurs de liquidité et d'activité
\item \textbf{Volatilités implicites} : Calculées selon le modèle de Black-Scholes
\item \textbf{Grecques} : Delta, Gamma, Vega, Theta calculées par le fournisseur
\item \textbf{Métadonnées} : Spécifications contractuelles et ajustements éventuels
\end{itemize}

\subsection{Période d'observation}

La période d'étude s'étend du 1er janvier 2020 au 31 décembre 2022, soit 36 mois de données comprenant 782 jours de trading. Cette période a été choisie pour sa richesse en termes de conditions de marché variées :

\subsubsection{Phase 1 : Choc initial COVID-19 (Janvier-Mars 2020)}

Cette période capture l'émergence et l'impact initial de la pandémie COVID-19 sur les marchés financiers. Caractéristiques observées :
\begin{itemize}
\item VIX maximum à 82.69 le 16 mars 2020
\item Volatilité réalisée annualisée dépassant 80\%
\item Dislocations significatives dans la structure par terme de volatilité
\item Asymétrie extrême du smile de volatilité (skew jusqu'à -30\%)
\end{itemize}

\subsubsection{Phase 2 : Adaptation et volatilité soutenue (Avril-Décembre 2020)}

Période d'adaptation aux nouvelles conditions économiques avec :
\begin{itemize}
\item VIX oscillant entre 20 et 40
\item Interventions massives des banques centrales
\item Changements structurels dans les corrélations inter-actifs
\item Émergence de patterns de volatilité inédits
\end{itemize}

\subsubsection{Phase 3 : Normalisation progressive (2021)}

Retour graduel vers des conditions de marché plus conventionnelles :
\begin{itemize}
\item VIX moyen de 19.4 (vs moyenne historique de 16.8)
\item Reprise de la croissance économique
\item Politique monétaire accommodante maintenue
\item Rotation sectorielle importante
\end{itemize}

\subsubsection{Phase 4 : Tensions inflationnistes (2022)}

Nouvelle période de volatilité liée aux préoccupations inflationnistes :
\begin{itemize}
\item Resserrement de la politique monétaire
\item Guerre en Ukraine et chocs géopolitiques
\item VIX moyen de 26.8
\item Inversion de la courbe des taux
\end{itemize}

\subsection{Statistiques descriptives du dataset brut}

Le dataset brut comprend initialement 2,847,392 observations d'options réparties comme suit :

\begin{table}[H]
\centering
\caption{Répartition des observations par année}
\begin{tabular}{@{}lcccc@{}}
\toprule
\textbf{Année} & \textbf{Observations} & \textbf{Options Weekly} & \textbf{VIX moyen} & \textbf{Volume quotidien} \\
\midrule
2020 & 987,456 & 78.3\% & 29.8 & 1,234,567 \\
2021 & 912,234 & 81.2\% & 19.4 & 1,567,890 \\
2022 & 947,702 & 83.7\% & 26.8 & 1,789,234 \\
\textbf{Total} & \textbf{2,847,392} & \textbf{81.1\%} & \textbf{25.3} & \textbf{1,530,564} \\
\bottomrule
\end{tabular}
\end{table}

\section{Procédures de nettoyage et filtrage}

\subsection{Critères de qualité des données}

La qualité des données d'options revêt une importance cruciale pour la précision des calibrations ultérieures. Notre protocole de nettoyage applique une série de filtres rigoureux développés en s'inspirant des meilleures pratiques académiques et industrielles.

\subsubsection{Filtre de liquidité}

Le premier filtre élimine les options illiquides susceptibles de présenter des prix peu représentatifs :

\begin{itemize}
\item \textbf{Spread bid-ask} : Élimination si spread > 10\% du prix mid
\item \textbf{Volume minimum} : Conservation uniquement si volume ≥ 10 contrats
\item \textbf{Open interest} : Exclusion si open interest < 100 contrats
\item \textbf{Prix minimum} : Élimination si prix < \$0.05
\end{itemize}

Ce filtre élimine 23.7\% des observations initiales, principalement des options très out-of-the-money ou proches de l'expiration.

\subsubsection{Filtre d'arbitrage}

Le second filtre vérifie l'absence d'opportunités d'arbitrage évidentes :

\begin{enumerate}
\item \textbf{Monotonicité par strike} : Pour des options de même maturité, vérification que :
   \begin{itemize}
   \item $C(K_1) \geq C(K_2)$ si $K_1 < K_2$ (calls)
   \item $P(K_1) \leq P(K_2)$ si $K_1 < K_2$ (puts)
   \end{itemize}

\item \textbf{Convexité} : Vérification de la convexité des prix par rapport au strike

\item \textbf{Parité call-put} : Validation de $C - P = S - Ke^{-rT}$ avec tolérance de ±2\%

\item \textbf{Bornes intrinsèques} : Vérification que $C \geq \max(S - Ke^{-rT}, 0)$
\end{enumerate}

Ce filtre supprime 8.3\% d'observations supplémentaires.

\subsubsection{Filtre de volatilité implicite}

Le troisième filtre porte sur la plausibilité des volatilités implicites :

\begin{itemize}
\item \textbf{Bornes absolues} : Conservation uniquement si $\sigma_{IV} \in [0.02, 2.00]$
\item \textbf{Convergence numérique} : Élimination si l'algorithme d'inversion Black-Scholes ne converge pas
\item \textbf{Stabilité temporelle} : Exclusion des volatilités présentant des variations > 50\% jour-à-jour sans justification
\end{itemize}

\subsubsection{Filtre de maturité et moneyness}

Le dernier filtre restreint l'échantillon aux caractéristiques d'intérêt :

\begin{itemize}
\item \textbf{Maturité} : Conservation uniquement si $T \in [1/365, 60/365]$ années
\item \textbf{Moneyness} : Restriction à $m = K/S \in [0.75, 1.25]$
\item \textbf{Jours à l'expiration} : Exclusion des options expirant le jour même (T=0)
\end{itemize}

\subsection{Impact du processus de filtrage}

L'application séquentielle de ces filtres réduit le dataset de 2,847,392 à 1,567,823 observations, soit un taux de rétention de 55.1\%. Cette réduction substantielle reflète la rigueur du processus de nettoyage et garantit la qualité des données utilisées pour la calibration.

\begin{table}[H]
\centering
\caption{Impact séquentiel des filtres de qualité}
\begin{tabular}{@{}lccc@{}}
\toprule
\textbf{Étape} & \textbf{Observations restantes} & \textbf{Taux de rétention} & \textbf{Observations supprimées} \\
\midrule
Dataset initial & 2,847,392 & 100.0\% & - \\
Après filtre liquidité & 2,174,086 & 76.3\% & 673,306 \\
Après filtre arbitrage & 1,993,567 & 70.0\% & 180,519 \\
Après filtre volatilité & 1,842,234 & 64.7\% & 151,333 \\
Après filtre final & 1,567,823 & 55.1\% & 274,411 \\
\bottomrule
\end{tabular}
</table>

\section{Analyse exploratoire des données}

\subsection{Distribution des caractéristiques}

L'analyse exploratoire révèle des patterns intéressants dans la distribution des données nettoyées.

\subsubsection{Distribution des maturités}

La répartition des maturités montre une concentration marquée sur les échéances très courtes :

\begin{table}[H]
\centering
\caption{Distribution des maturités (jours calendaires)}
\begin{tabular}{@{}lccc@{}}
\toprule
\textbf{Plage de maturité} & \textbf{Observations} & \textbf{Pourcentage} & \textbf{Liquidité moyenne} \\
\midrule
1-7 jours & 534,123 & 34.1\% & Très élevée \\
8-14 jours & 412,567 & 26.3\% & Élevée \\
15-30 jours & 387,234 & 24.7\% & Élevée \\
31-45 jours & 178,456 & 11.4\% & Modérée \\
46-60 jours & 55,443 & 3.5\% & Faible \\
\bottomrule
\end{tabular}
\end{table}

Cette concentration sur les maturités courtes reflète les préférences des traders pour les instruments à décroissance temporelle rapide.

\subsubsection{Distribution de la moneyness}

L'analyse de la moneyness révèle une activité concentrée autour de la monnaie :

\begin{table}[H]
\centering
\caption{Distribution de la moneyness}
\begin{tabular}{@{}lccc@{}}
\toprule
\textbf{Région} & \textbf{Plage moneyness} & \textbf{Observations} & \textbf{Volume moyen} \\
\midrule
Deep OTM Put & [0.75, 0.85] & 145,233 & 156 \\
OTM Put & [0.85, 0.95] & 298,567 & 342 \\
Near the money & [0.95, 1.05] & 634,789 & 1,234 \\
OTM Call & [1.05, 1.15] & 287,234 & 298 \\
Deep OTM Call & [1.15, 1.25] & 202,000 & 123 \\
\bottomrule
\end{tabular}
\end{table}

\subsection{Évolution temporelle de la volatilité implicite}

\subsubsection{Analyse de la structure par terme}

L'évolution de la structure par terme de volatilité implicite fournit des insights cruciaux sur les anticipations de marché :

\begin{table}[H]
\centering
\caption{Volatilité implicite moyenne par maturité et période}
\begin{tabular}{@{}lcccc@{}}
\toprule
\textbf{Maturité} & \textbf{2020 Q1} & \textbf{2020 Q2-Q4} & \textbf{2021} & \textbf{2022} \\
\midrule
1-7 jours & 45.8\% & 28.3\% & 18.2\% & 24.7\% \\
8-14 jours & 42.1\% & 26.7\% & 17.8\% & 23.9\% \\
15-30 jours & 38.9\% & 25.1\% & 17.3\% & 23.1\% \\
31-45 jours & 36.2\% & 23.8\% & 16.9\% & 22.6\% \\
46-60 jours & 34.7\% & 22.9\% & 16.5\% & 22.1\% \\
\bottomrule
\end{tabular}
\end{table}

Ces données révèlent plusieurs patterns importants :
\begin{itemize}
\item \textbf{Terme structure inversée} en période de crise (2020 Q1)
\item \textbf{Normalisation progressive} de la structure par terme
\item \textbf{Prime de volatilité courte} persistante à travers tous les régimes
\end{itemize}

\subsubsection{Analyse du smile de volatilité}

L'asymétrie du smile de volatilité varie significativement selon les conditions de marché :

\begin{table}[H]
\centering
\caption{Skew de volatilité par période (différence OTM Put - ATM)}
\begin{tabular}{@{}lcccc@{}}
\toprule
\textbf{Maturité} & \textbf{2020 Q1} & \textbf{2020 Q2-Q4} & \textbf{2021} & \textbf{2022} \\
\midrule
1-7 jours & -28.3\% & -15.7\% & -8.9\% & -12.4\% \\
8-14 jours & -23.1\% & -13.2\% & -7.8\% & -10.9\% \\
15-30 jours & -18.7\% & -11.4\% & -6.9\% & -9.7\% \\
31-45 jours & -15.9\% & -9.8\% & -6.2\% & -8.8\% \\
46-60 jours & -14.2\% & -8.9\% & -5.7\% & -8.1\% \\
\bottomrule
\end{tabular}
\end{table}

\subsection{Corrélations et dépendances}

\subsubsection{Corrélation avec les indices de marché}

L'analyse des corrélations entre volatilités implicites et variables de marché révèle :

\begin{table}[H]
\centering
\caption{Corrélations avec variables de marché}
\begin{tabular}{@{}lcccc@{}}
\toprule
\textbf{Variable} & \textbf{IV Court terme} & \textbf{IV Moyen terme} & \textbf{Skew} & \textbf{Smile} \\
\midrule
Niveau SPX & -0.42 & -0.38 & 0.31 & -0.18 \\
VIX & 0.89 & 0.82 & 0.67 & 0.23 \\
Rendement 10Y & -0.23 & -0.19 & 0.15 & -0.09 \\
Dollar Index & 0.18 & 0.15 & -0.12 & 0.07 \\
Credit Spreads & 0.34 & 0.31 & 0.28 & 0.12 \\
\bottomrule
\end{tabular}
\end{table>

\subsubsection{Persistance temporelle}

L'analyse de l'autocorrélation révèle des patterns de persistance différentiés :

\begin{itemize}
\item \textbf{Volatilité implicite ATM} : Autocorrélation de 0.85 à 1 jour, 0.34 à 5 jours
\item \textbf{Skew} : Persistance plus élevée (0.92 à 1 jour, 0.58 à 5 jours)
\item \textbf{Structure par terme} : Très stable (0.95 à 1 jour, 0.78 à 5 jours)
\end{itemize}

\section{Patterns spécifiques aux options Weekly}

\subsection{Effet d'expiration}

Les options Weekly présentent des patterns d'expiration distincts des options mensuelles :

\subsubsection{Décroissance de la valeur temps}

L'analyse de la décroissance de theta révèle une accélération non-linéaire marquée :

\begin{table}[H]
\centering
\caption{Theta moyen par jours à l'expiration}
\begin{tabular}{@{}lcccc@{}}
\toprule
\textbf{Jours à expiration} & \textbf{ATM Calls} & \textbf{ATM Puts} & \textbf{OTM Calls} & \textbf{OTM Puts} \\
\midrule
5-7 jours & -0.042 & -0.041 & -0.018 & -0.019 \\
3-4 jours & -0.067 & -0.065 & -0.028 & -0.031 \\
2 jours & -0.098 & -0.094 & -0.039 & -0.043 \\
1 jour & -0.156 & -0.152 & -0.061 & -0.067 \\
\bottomrule
\end{tabular}
\end{table>

\subsubsection{Compression du smile}

À l'approche de l'expiration, le smile de volatilité présente une compression caractéristique :

\begin{itemize}
\item \textbf{Réduction du skew} : Diminution de 40\% dans les 3 derniers jours
\item \textbf{Aplatissement du smile} : Convergence vers la volatilité réalisée
\item \textbf{Instabilité accrue} : Variance des volatilités implicites multipliée par 3
\end{itemize}

\subsection{Phénomènes de fin de semaine}

\subsubsection{Effet weekend}

Les options expirant le vendredi présentent des caractéristiques uniques :

\begin{table}[H]
\centering
\caption{Primes de volatilité selon le jour d'expiration}
\begin{tabular}{@{}lccc@{}}
\toprule
\textbf{Jour d'expiration} & \textbf{Prime ATM} & \textbf{Prime OTM} & \textbf{Volume relatif} \\
\midrule
Lundi & +2.1\% & +3.4\% & 0.8x \\
Mardi & +1.8\% & +2.9\% & 0.9x \\
Mercredi & +1.2\% & +1.8\% & 1.1x \\
Jeudi & +0.8\% & +1.2\% & 1.3x \\
Vendredi & Référence & Référence & 1.0x \\
\bottomrule
\end{tabular}
\end{table>

\section{Implications pour la calibration}

\subsection{Défis spécifiques identifiés}

L'analyse exploratoire révèle plusieurs défis spécifiques pour la calibration du modèle de Heston sur données SPX Weekly :

\subsubsection{Granularité temporelle}

La concentration sur les maturités très courtes (< 30 jours) pose des défis particuliers :
\begin{itemize}
\item \textbf{Sensibilité accrue au paramètre $v_0$} (variance initiale)
\item \textbf{Instabilité numérique} des calculs d'intégrales pour $T \rightarrow 0$
\item \textbf{Dominance des effets de microstructure} pour les maturités < 7 jours
\end{itemize}

\subsubsection{Régimes de volatilité extrêmes}

Les périodes de crise révèlent des limitations des modèles classiques :
\begin{itemize}
\item \textbf{Non-stationnarité} des paramètres du modèle de Heston
\item \textbf{Breakdown du modèle} pour des volatilités > 100\%
\item \textbf{Asymétries extrêmes} non capturable par la corrélation $\rho$ seule
\end{itemize}

\subsection{Opportunities pour l'approche Deep Learning}

Ces défis créent simultanément des opportunities pour l'approche par apprentissage automatique :

\subsubsection{Adaptabilité aux régimes}

Les réseaux de neurones peuvent naturellement s'adapter aux différents régimes de marché sans nécessiter de recalibration manuelle des paramètres.

\subsubsection{Capture des non-linéarités}

L'architecture neuronale peut capturer des relations complexes entre paramètres, maturités et moneyness que les approches analytiques peinent à modéliser.

\subsubsection{Robustesse aux données extrêmes}

L'entraînement sur un large spectre de conditions de marché (incluant la crise COVID-19) devrait conférer une robustesse supérieure aux approches traditionnelles.

\section{Préparation pour la méthodologie}

\subsection{Division de l'échantillon}

Pour les besoins de l'étude, le dataset nettoyé est divisé selon une approche temporelle :

\begin{table}[H]
\centering
\caption{Division temporelle du dataset}
\begin{tabular}{@{}lccc@{}}
\toprule
\textbf{Période} & \textbf{Usage} & \textbf{Observations} & \textbf{Pourcentage} \\
\midrule
Jan 2020 - Jun 2022 & Entraînement/Validation & 1,254,258 & 80.0\% \\
Jul 2022 - Dec 2022 & Test out-of-sample & 313,565 & 20.0\% \\
\bottomrule
\end{tabular}
\end{table}

Cette division préserve la dimension temporelle cruciale pour valider la capacité de généralisation des modèles.

\subsection{Normalisation et preprocessing}

Les données subissent une normalisation sophistiquée préparant l'entraînement neuronal :

\begin{itemize}
\item \textbf{Volatilités implicites} : Transformation logit pour borner dans [0,1]
\item \textbf{Maturités} : Transformation logarithmique pour gérer la non-linéarité
\item \textbf{Moneyness} : Centrage et réduction par la volatilité ATM
\item \textbf{Variables de marché} : Standardisation z-score sur la période d'entraînement
\end{itemize}

\section{Conclusion}

Ce chapitre a présenté de manière exhaustive les caractéristiques du dataset d'options SPX Weekly utilisé dans cette recherche. L'analyse exploratoire révèle un dataset riche et représentatif, couvrant une gamme exceptionnellement large de conditions de marché.

Les spécificités identifiées - concentration sur les maturités courtes, patterns d'expiration unique, sensibilité aux régimes de volatilité - motivent l'adoption d'approches de calibration innovantes. Les défis révélés par l'analyse traditionnelle créent simultanément les opportunities que l'approche Deep Learning propose d'exploiter.

La qualité rigoureuse du processus de nettoyage, combinée à la richesse temporelle du dataset, fournit les fondations solides nécessaires pour l'implémentation et la validation de notre méthodologie de calibration hybride, présentée dans le chapitre suivant.

Les patterns observés - de la persistance du skew à l'effet d'expiration des options Weekly - constituent autant de régularités que notre réseau de neurones devra apprendre à reproduire fidèlement. La validation de cette capacité d'apprentissage constitue l'un des enjeux centraux de notre recherche, dont les résultats sont présentés dans les chapitres ultérieurs.

\chapter{Méthodologie}

\section{Introduction}

Ce chapitre présente la méthodologie complète de notre approche de calibration accélérée du modèle de Heston par deep learning. Nous débutons par une formalisation rigoureuse du modèle de Heston et de ses propriétés théoriques, puis décrivons les méthodes de calibration traditionnelles avant d'introduire notre approche basée sur les réseaux de neurones. La présentation suit une progression logique depuis les fondements théoriques jusqu'aux détails d'implémentation pratique.

\section{Le modèle de Heston : formulation et propriétés}

\subsection{Formulation mathématique}

Le modèle de Heston (1993) constitue l'un des modèles de volatilité stochastique les plus influents en finance quantitative. Il spécifie la dynamique conjointe du prix de l'actif sous-jacent $S_t$ et de sa variance instantanée $v_t$ sous la mesure risque-neutre :

\begin{align}
dS_t &= rS_t dt + \sqrt{v_t}S_t dW_t^{(1)} \label{eq:heston_price}\\
dv_t &= \kappa(\theta - v_t)dt + \sigma\sqrt{v_t}dW_t^{(2)} \label{eq:heston_variance}
\end{align}

où $dW_t^{(1)}$ et $dW_t^{(2)}$ représentent deux mouvements browniens standards corrélés par $d\langle W^{(1)}, W^{(2)} \rangle_t = \rho dt$. Les paramètres du modèle possèdent des interprétations économiques précises : $\kappa > 0$ désigne la vitesse de retour à la moyenne de la variance, $\theta > 0$ représente la variance long terme, $\sigma > 0$ quantifie la volatilité de la variance, et $\rho \in [-1,1]$ capture l'effet de levier.

\subsection{Propriétés théoriques fondamentales}

La condition de Feller $2\kappa\theta \geq \sigma^2$ garantit que le processus de variance reste strictement positif, propriété essentielle pour la cohérence économique du modèle. Lorsque cette condition est violée, le processus peut atteindre zéro avec une probabilité positive, nécessitant des techniques de réflexion ou d'absorption pour le traitement numérique.

La structure de corrélation entre les innovations du prix et de la variance, captée par le paramètre $\rho$, permet de reproduire l'effet de levier empiriquement observé. Une corrélation négative implique que les chocs négatifs sur le prix s'accompagnent d'une augmentation de la volatilité, phénomène systématiquement documenté sur les marchés d'actions.

\subsection{Solution analytique pour les options européennes}

L'élégance du modèle de Heston réside dans l'existence d'une solution analytique fermée pour le prix des options européennes. Cette solution s'exprime sous la forme d'une intégrale de Fourier :

\begin{equation}
C(S_0, v_0, \tau) = S_0 P_1 - Ke^{-r\tau} P_2
\end{equation}

où $P_1$ et $P_2$ représentent des probabilités calculées via la transformée de Fourier de la fonction caractéristique du logarithme du prix. Ces probabilités s'expriment comme :

\begin{equation}
P_j = \frac{1}{2} + \frac{1}{\pi} \int_0^{\infty} \text{Re}\left[\frac{e^{-i\phi \ln(K)}\phi_j(S_0, v_0, \tau, \phi)}{i\phi}\right] d\phi
\end{equation}

La fonction caractéristique $\phi_j$ admet une forme exponentielle affine, caractéristique des modèles de diffusion affines. Cette propriété découle de la structure particulière des équations différentielles stochastiques du modèle de Heston.

\subsection{Défis computationnels de l'évaluation}

Bien que la solution analytique de Heston évite les simulations de Monte Carlo, son évaluation numérique présente des défis techniques significatifs. L'intégrale de Fourier implique des fonctions oscillantes complexes dont l'intégration numérique nécessite des techniques sophistiquées.

Les principales difficultés incluent la gestion de la discontinuité de la fonction intégrande à l'origine, le choix de la borne supérieure d'intégration, et le traitement des instabilités numériques dans certaines régions de l'espace des paramètres. Ces aspects techniques expliquent en partie la motivation pour des approches alternatives comme notre méthode de deep learning.

\section{Méthodes de calibration traditionnelles}

\subsection{Calibration par optimisation non-linéaire}

L'approche standard de calibration du modèle de Heston consiste à minimiser une fonction objectif mesurant l'écart entre les volatilités implicites observées sur le marché et celles prédites par le modèle. Cette fonction objectif s'écrit généralement sous la forme :

\begin{equation}
\Theta^* = \arg\min_{\Theta} \sum_{i=1}^{N} w_i \left(\sigma_{imp}^{market}(K_i, T_i) - \sigma_{imp}^{Heston}(K_i, T_i; \Theta)\right)^2
\end{equation}

où $\Theta = (\kappa, \theta, \sigma, \rho, v_0)$ représente le vecteur des paramètres à estimer, $w_i$ désigne un poids associé à l'observation $i$, et $N$ correspond au nombre total d'options observées.

\subsection{Choix de la fonction objectif et pondération}

Le choix de la fonction objectif et du schéma de pondération influence significativement les résultats de calibration. Les approches courantes incluent la minimisation des erreurs absolues, des erreurs relatives, ou des erreurs vega-pondérées. Chaque spécification présente des avantages et inconvénients selon les objectifs d'utilisation.

La pondération par vega accorde plus d'importance aux options sensibles aux variations de volatilité, approche pertinente pour les applications de gestion des risques. La pondération uniforme traite toutes les observations de manière égale, tandis que la pondération par liquidité privilégie les options les plus activement négociées.

\subsection{Algorithmes d'optimisation}

Les algorithmes d'optimisation couramment utilisés incluent les méthodes de quasi-Newton, les algorithmes génétiques, et l'optimisation par essaims particulaires. Chaque approche présente des caractéristiques spécifiques en termes de convergence, de robustesse aux minima locaux, et de coût computationnel.

Les méthodes de quasi-Newton, telles que BFGS, exploitent les informations de gradient pour une convergence rapide mais restent sensibles aux conditions initiales. Les algorithmes métaheuristiques offrent une meilleure exploration globale au prix d'un coût computationnel supérieur.

\subsection{Contraintes et régularisation}

La calibration pratique nécessite l'imposition de contraintes sur l'espace des paramètres pour garantir la stabilité économique et numérique du modèle. Ces contraintes incluent les bornes naturelles ($\kappa > 0$, $\theta > 0$, $\sigma > 0$), la condition de Feller, et des bornes réalistes basées sur l'évidence empirique.

Des techniques de régularisation peuvent également être employées pour éviter l'overfitting, particulièrement pertinent lorsque le nombre d'observations est limité. La régularisation par pénalité L2 sur les paramètres constitue une approche standard pour améliorer la stabilité de l'estimation.

\section{Architecture du réseau de neurones}

\subsection{Conception générale}

Notre architecture s'inspire de l'approche de Bayer et Stemper (2018) adaptée spécifiquement au modèle de Heston. Le réseau de neurones est conçu pour apprendre la relation complexe entre les caractéristiques d'entrée (moneyness, maturité, paramètres du modèle) et la volatilité implicite de sortie.

L'architecture adoptée consiste en un réseau feed-forward dense avec plusieurs couches cachées. Cette structure permet de capturer les non-linéarités complexes de la fonction de mapping tout en conservant une tractabilité computationnelle pour l'entraînement et l'inférence.

\subsection{Variables d'entrée et préprocessing}

Les variables d'entrée du réseau comprennent la moneyness $m = S_0/K$, la maturité $\tau$, et les cinq paramètres du modèle de Heston $(\kappa, \theta, \sigma, \rho, v_0)$. Cette spécification permet au réseau d'apprendre directement la relation paramètres-volatilité implicite nécessaire pour la calibration inverse.

Le préprocessing des variables d'entrée inclut une normalisation par score z pour assurer la stabilité de l'entraînement. Les transformations appliquées respectent les domaines naturels des variables : transformation logarithmique pour les variables strictement positives, transformation arctangente hyperbolique pour les variables bornées.

\subsection{Architecture des couches cachées}

L'architecture comprend quatre couches cachées avec 256, 128, 64, et 32 neurones respectivement. Cette structure pyramidale permet une extraction hiérarchique des features, des représentations générales vers les patterns spécialisés. Les fonctions d'activation ReLU sont utilisées pour introduire la non-linéarité nécessaire à l'approximation des relations complexes.

Des couches de dropout avec des taux de 0.3 et 0.2 sont introduites après les deux premières couches cachées pour prévenir l'overfitting. Cette régularisation est particulièrement importante compte tenu de la complexité de l'espace des paramètres et de la richesse de l'ensemble d'entraînement.

\subsection{Couche de sortie et post-processing}

La couche de sortie comprend un unique neurone avec une fonction d'activation linéaire, produisant la volatilité implicite prédite. Cette spécification respecte la nature continue de la variable de sortie tout en permettant l'apprentissage de relations arbitrairement complexes via les couches cachées.

Le post-processing de la sortie inclut une dénormalisation appropriée et l'application de contraintes de cohérence (volatilité strictement positive). Ces étapes garantissent que les prédictions du réseau respectent les contraintes économiques fondamentales.

\section{Stratégie d'entraînement}

\subsection{Génération de l'ensemble d'entraînement}

L'ensemble d'entraînement combine des données réelles et synthétiques selon la méthodologie établie. Les données synthétiques sont générées par échantillonnage uniforme dans l'espace des paramètres, suivi du calcul des volatilités implicites correspondantes via la formule analytique de Heston.

Cette approche hybride exploite la richesse des données réelles pour l'ancrage empirique tout en bénéficiant de la couverture exhaustive des données synthétiques. Le rapport synthétique/réel d'environ 7:1 s'inspire des best practices établies dans la littérature.

\subsection{Fonction de perte et optimisation}

La fonction de perte adoptée correspond à l'erreur quadratique moyenne entre volatilités implicites prédites et observées, avec une pénalité L2 sur les poids du réseau pour la régularisation :

\begin{equation}
\mathcal{L} = \frac{1}{N}\sum_{i=1}^{N} (\sigma_{imp}^{pred}(x_i) - \sigma_{imp}^{true}(x_i))^2 + \lambda \sum_{w} w^2
\end{equation}

où $\lambda$ contrôle l'intensité de la régularisation. Cette spécification équilibre la précision des prédictions avec la généralisation du modèle.

\subsection{Algorithme d'optimisation et hyperparamètres}

L'optimisation utilise l'algorithme Adam avec un taux d'apprentissage initial de 0.001, des paramètres de moment $\beta_1 = 0.9$ et $\beta_2 = 0.999$. Cette configuration assure une convergence stable tout en adaptant automatiquement le taux d'apprentissage selon l'historique des gradients.

L'entraînement s'effectue par mini-batches de taille 512 sur 200 époques avec early stopping basé sur la performance de validation. Cette stratégie prévient l'overfitting tout en permettant une convergence complète de l'algorithme d'optimisation.

\section{Procédure de calibration inverse}

\subsection{Formulation du problème d'optimisation}

Une fois le réseau entraîné, la calibration s'effectue par résolution du problème d'optimisation inverse :

\begin{equation}
\hat{\Theta} = \arg\min_{\Theta} \sum_{i=1}^{N} w_i \left(\sigma_{imp}^{market}(K_i, T_i) - \mathcal{NN}(m_i, \tau_i, \Theta)\right)^2
\end{equation}

où $\mathcal{NN}(\cdot)$ désigne la prédiction du réseau de neurones entraîné. Cette formulation remplace l'évaluation coûteuse de la formule de Heston par une prédiction instantanée du réseau.

\subsection{Algorithme d'optimisation pour la calibration}

La calibration inverse utilise l'algorithme L-BFGS-B qui exploite efficacement les informations de gradient disponibles via la différentiation automatique du réseau de neurones. Cette approche combine la rapidité de convergence des méthodes de quasi-Newton avec la gestion des contraintes par bornes.

Les gradients de la fonction objectif par rapport aux paramètres sont calculés via backpropagation, technique standard pour la différentiation de réseaux de neurones. Cette disponibilité des gradients analytiques accélère significativement la convergence par rapport aux méthodes d'optimisation sans gradient.

\subsection{Gestion des contraintes et initialisation}

Les contraintes sur les paramètres sont gérées via des transformations de variables et des projections. La condition de Feller est imposée par une pénalité douce dans la fonction objectif, approche qui préserve la différentiabilité tout en encourageant le respect de la contrainte.

L'initialisation des paramètres pour l'optimisation utilise des valeurs typiques basées sur l'évidence empirique : $\kappa = 2$, $\theta = 0.04$, $\sigma = 0.3$, $\rho = -0.5$, $v_0 = 0.04$. Cette initialisation centrale dans l'espace des paramètres favorise une convergence stable.

\section{Métriques d'évaluation}

\subsection{Métriques de précision}

L'évaluation de la performance utilise plusieurs métriques complémentaires. L'erreur quadratique moyenne (RMSE) quantifie la précision globale des prédictions :

\begin{equation}
RMSE = \sqrt{\frac{1}{N}\sum_{i=1}^{N} (\sigma_{imp}^{pred}(x_i) - \sigma_{imp}^{true}(x_i))^2}
\end{equation}

L'erreur absolue moyenne (MAE) fournit une mesure plus robuste aux outliers :

\begin{equation}
MAE = \frac{1}{N}\sum_{i=1}^{N} |\sigma_{imp}^{pred}(x_i) - \sigma_{imp}^{true}(x_i)|
\end{equation}

Le coefficient de détermination $R^2$ mesure la proportion de variance expliquée par le modèle, indicateur de la qualité globale de l'ajustement.

\subsection{Métriques de performance computationnelle}

L'évaluation de l'efficacité computationnelle compare les temps d'exécution de la calibration traditionnelle et accélérée. Le speed-up ratio quantifie le gain en performance :

\begin{equation}
\text{Speed-up} = \frac{\text{Temps calibration traditionnelle}}{\text{Temps calibration accélérée}}
\end{equation}

Cette métrique inclut uniquement le temps de calibration, excluant l'entraînement initial du réseau considéré comme un coût fixe amorti sur de multiples calibrations.

\subsection{Tests de robustesse}

La robustesse de l'approche est évaluée via plusieurs tests complémentaires. La stabilité des paramètres calibrés est mesurée par la variance des estimations sur des échantillons bootstrap. Cette analyse révèle la sensibilité de la méthode aux variations d'échantillonnage.

La généralisation temporelle est testée en évaluant les performances sur des périodes non incluses dans l'entraînement. Cette validation out-of-sample constitue un test crucial pour l'applicabilité pratique de notre approche.

\section{Implémentation technique}

\subsection{Framework et outils utilisés}

L'implémentation utilise Python avec les bibliothèques TensorFlow pour le deep learning, QuantLib pour l'évaluation de référence du modèle de Heston, et SciPy pour l'optimisation. Cette combinaison offre un environnement robuste et performant pour notre méthodologie.

Le code est structuré de manière modulaire pour faciliter la reproductibilité et l'extension. Les composants principaux incluent un module de génération de données, un module de définition et d'entraînement du réseau, et un module de calibration inverse.

\subsection{Considérations de performance}

L'optimisation des performances exploite les capacités de calcul parallèle des GPUs pour l'entraînement du réseau de neurones et l'évaluation batch des prédictions. Cette parallélisation accélère significativement les calculs par rapport à une implémentation CPU séquentielle.

La gestion mémoire optimise l'utilisation des ressources en chargeant les données par batches et en libérant la mémoire de manière appropriée. Ces optimisations permettent de traiter des ensembles de données volumineux sans contraintes matérielles excessives.

Cette méthodologie complète fournit le cadre technique nécessaire pour notre analyse comparative de la calibration accélérée du modèle de Heston. L'implémentation rigoureuse de chaque composant garantit la validité et la reproductibilité de nos résultats empiriques.

\chapter{Résultats et discussions}

\section{Introduction}

Ce chapitre présente les résultats de l'implémentation des méthodologies de calibration du modèle de Heston par Deep Learning. Nous analysons les performances de l'approche en deux étapes de Bayer et Stemper, puis comparons ses résultats avec les méthodes de calibration traditionnelles. L'objectif principal est d'évaluer l'efficacité, la précision et la robustesse de ces nouvelles approches dans le contexte spécifique des données SPX Weekly.

\section{Configuration expérimentale}

\subsection{Environnement de calcul}

Les expérimentations ont été conduites sur un environnement de calcul haute performance comprenant des GPU NVIDIA Tesla V100 avec 32GB de mémoire. L'utilisation de GPU s'avère cruciale pour l'entraînement efficace des réseaux de neurones profonds requis par notre méthodologie.

Le framework de Deep Learning utilisé est PyTorch 1.12, choisi pour sa flexibilité dans l'implémentation d'architectures personnalisées et sa capacité à calculer efficacement les gradients automatiques nécessaires pour l'entraînement des réseaux.

\subsection{Génération des données synthétiques}

Suivant la méthodologie de Bayer et Stemper, nous avons généré un ensemble de données synthétiques comprenant 500,000 surfaces de volatilité implicite correspondant à différentes combinaisons de paramètres du modèle de Heston. Cette approche permet de couvrir exhaustivement l'espace des paramètres tout en contrôlant la qualité des données d'entraînement.

Les paramètres du modèle de Heston ont été échantillonnés selon les distributions suivantes, calibrées sur l'analyse historique du marché SPX :

\begin{align}
\kappa &\sim \mathcal{U}(0.1, 5.0) \\
\theta &\sim \mathcal{U}(0.01, 0.5) \\
\sigma &\sim \mathcal{U}(0.05, 1.0) \\
\rho &\sim \mathcal{U}(-0.9, 0.1) \\
v_0 &\sim \mathcal{U}(0.01, 0.5)
\end{align}

Pour chaque jeu de paramètres, nous avons calculé les volatilités implicites correspondantes sur une grille de moneyness $m \in [0.7, 1.3]$ et de maturités $T \in [0.02, 2.0]$ années, représentant fidèlement les caractéristiques des options SPX Weekly disponibles sur le marché.

\subsection{Architecture du réseau de neurones}

L'architecture du réseau de neurones utilisée pour approximer la fonction de pricing suit les recommandations de Bayer et Stemper avec quelques adaptations spécifiques à notre contexte d'application. Le réseau comprend :

\begin{itemize}
\item Une couche d'entrée de dimension 7 (5 paramètres de Heston + moneyness + maturité)
\item Quatre couches cachées avec respectivement 256, 512, 256 et 128 neurones
\item Fonctions d'activation ReLU pour les couches cachées
\item Une couche de sortie de dimension 1 (volatilité implicite)
\item Techniques de régularisation : Dropout (p=0.2) et Batch Normalization
\end{itemize}

Cette architecture a été optimisée par validation croisée, en évaluant différentes configurations sur un ensemble de validation séparé.

\section{Résultats de l'entraînement du réseau de neurones}

\subsection{Convergence et stabilité}

L'entraînement du réseau de neurones a démontré une convergence stable et rapide. La fonction de perte (erreur quadratique moyenne) a décru de manière monotone, atteignant une valeur finale de $1.2 \times 10^{-5}$ après 1000 époques d'entraînement.

\begin{table}[H]
\centering
\caption{Métriques de performance du réseau de neurones}
\begin{tabular}{@{}lccc@{}}
\toprule
\textbf{Métrique} & \textbf{Entraînement} & \textbf{Validation} & \textbf{Test} \\
\midrule
MSE & $1.2 \times 10^{-5}$ & $1.4 \times 10^{-5}$ & $1.3 \times 10^{-5}$ \\
MAE & $2.1 \times 10^{-3}$ & $2.3 \times 10^{-3}$ & $2.2 \times 10^{-3}$ \\
R² & 0.9987 & 0.9985 & 0.9986 \\
Temps par forward pass & - & - & 0.15 ms \\
\bottomrule
\end{tabular}
\end{table}

Ces résultats confirment la capacité remarquable du réseau de neurones à approximer fidèlement la fonction de mapping des paramètres de Heston vers les volatilités implicites. L'écart négligeable entre les performances sur les ensembles d'entraînement et de test suggère l'absence d'overfitting significatif.

\subsection{Analyse de précision par région}

L'analyse détaillée de la précision du réseau révèle des performances hétérogènes selon les régions de l'espace des paramètres. Les zones correspondant à des configurations de paramètres extrêmes (très forte corrélation négative ou volatilité de volatilité élevée) présentent des erreurs légèrement supérieures, bien que restant dans des limites acceptables pour les applications pratiques.

\begin{table}[H]
\centering
\caption{Erreurs par région de moneyness et maturité}
\begin{tabular}{@{}lccc@{}}
\toprule
\textbf{Région} & \textbf{RMSE} & \textbf{MAE} & \textbf{Erreur max} \\
\midrule
ATM, court terme ($T < 0.25$) & $1.8 \times 10^{-3}$ & $1.1 \times 10^{-3}$ & $8.2 \times 10^{-3}$ \\
ATM, long terme ($T > 1.0$) & $2.1 \times 10^{-3}$ & $1.4 \times 10^{-3}$ & $9.7 \times 10^{-3}$ \\
OTM, court terme & $2.9 \times 10^{-3}$ & $1.9 \times 10^{-3}$ & $1.2 \times 10^{-2}$ \\
ITM, court terme & $2.5 \times 10^{-3}$ & $1.6 \times 10^{-3}$ & $1.1 \times 10^{-2}$ \\
\bottomrule
\end{tabular}
\end{table}

\section{Performance de calibration comparative}

\subsection{Méthode de référence : Levenberg-Marquardt traditionnel}

Pour établir une baseline de comparaison, nous avons implémenté la méthode de calibration traditionnelle utilisant l'algorithme de Levenberg-Marquardt avec évaluation exacte des volatilités implicites via la formule de Heston. Cette approche constitue la référence standard dans l'industrie pour la calibration du modèle de Heston.

\subsection{Expérience de calibration sur données synthétiques}

Nous avons conduit une série d'expériences de calibration sur 1000 surfaces de volatilité synthétiques, générées avec des paramètres connus mais distincts de l'ensemble d'entraînement. Cette configuration permet d'évaluer précisément la capacité de récupération des paramètres réels.

\begin{table}[H]
\centering
\caption{Comparaison des performances de calibration}
\begin{tabular}{@{}lcccc@{}}
\toprule
\textbf{Méthode} & \textbf{Temps moyen} & \textbf{Taux de convergence} & \textbf{RMSE paramètres} & \textbf{RMSE prix} \\
\midrule
LM traditionnel & 23.4 s & 87\% & 0.092 & $1.8 \times 10^{-3}$ \\
Deep Learning & 2.1 s & 98\% & 0.089 & $1.6 \times 10^{-3}$ \\
Ratio d'amélioration & 11.1x & +11 pp & -3.3\% & -11.1\% \\
\bottomrule
\end{tabular}
\end{table}

Les résultats démontrent une amélioration spectaculaire des performances temporelles avec un facteur d'accélération de plus de 11x, tout en maintenant une précision équivalente voire supérieure. Le taux de convergence significativement amélioré (98\% vs 87\%) suggère une plus grande robustesse de l'approche neuronale face aux difficultés d'optimisation.

\subsection{Analyse détaillée par paramètre}

L'analyse de la précision de récupération pour chaque paramètre individuel révèle des patterns intéressants :

\begin{table}[H]
\centering
\caption{Erreur de récupération par paramètre}
\begin{tabular}{@{}lcccc@{}}
\toprule
\textbf{Paramètre} & \multicolumn{2}{c}{\textbf{Méthode traditionnelle}} & \multicolumn{2}{c}{\textbf{Deep Learning}} \\
 & \textbf{Biais moyen} & \textbf{RMSE} & \textbf{Biais moyen} & \textbf{RMSE} \\
\midrule
$\kappa$ & -0.012 & 0.156 & -0.008 & 0.142 \\
$\theta$ & 0.003 & 0.021 & 0.002 & 0.019 \\
$\sigma$ & -0.007 & 0.089 & -0.005 & 0.081 \\
$\rho$ & 0.019 & 0.067 & 0.015 & 0.061 \\
$v_0$ & -0.001 & 0.018 & -0.001 & 0.017 \\
\bottomrule
\end{tabular}
\end{table}

Ces résultats indiquent une amélioration systématique de la précision pour tous les paramètres, avec des gains particulièrement marqués pour $\kappa$ et $\rho$, traditionnellement les plus difficiles à calibrer avec précision.

\section{Tests sur données de marché réelles}

\subsection{Données SPX Weekly}

L'évaluation sur données réelles constitue le test ultime de la viabilité pratique de notre approche. Nous avons utilisé un dataset complet de prix d'options SPX Weekly couvrant la période de janvier 2020 à décembre 2022, incluant les périodes de forte volatilité liées à la crise COVID-19.

Le dataset comprend 156,000 observations d'options avec des caractéristiques représentatives :
\begin{itemize}
\item Maturités de 1 jour à 60 jours (caractéristique des options Weekly)
\item Moneyness de 0.75 à 1.25
\item Couverture complète des conditions de marché (VIX de 12 à 82)
\end{itemize}

\subsection{Procédure de validation}

La validation sur données réelles suit un protocole rigoureux :

\begin{enumerate}
\item Division temporelle : 70\% pour calibration, 30\% pour test out-of-sample
\item Calibration quotidienne avec fenêtre glissante de 252 jours
\item Évaluation des performances de pricing sur le jour suivant
\item Analyse des résidus et tests de stabilité
\end{enumerate}

\subsection{Résultats sur données réelles}

\begin{table}[H]
\centering
\caption{Performance sur données SPX Weekly (2020-2022)}
\begin{tabular}{@{}lcccc@{}}
\toprule
\textbf{Méthode} & \textbf{RMSE IV} & \textbf{MAE IV} & \textbf{Temps calibration} & \textbf{R² pricing} \\
\midrule
LM traditionnel & 0.0287 & 0.0198 & 31.2 s & 0.9341 \\
Deep Learning & 0.0251 & 0.0171 & 2.8 s & 0.9389 \\
Amélioration & -12.5\% & -13.6\% & 11.1x & +0.48 pp \\
\bottomrule
\end{tabular}
\end{table}

Les résultats sur données réelles confirment les gains observés en simulation, avec une réduction substantielle des erreurs de pricing et un maintien de l'accélération computationnelle. La légère amélioration du R² suggère que l'approche neuronale capture mieux certaines subtilités de la dynamique de marché.

\subsection{Analyse temporelle des performances}

L'analyse de l'évolution temporelle des performances révèle la robustesse de l'approche face aux changements de conditions de marché :

\begin{table}[H]
\centering
\caption{Performance par période de marché}
\begin{tabular}{@{}lcccc@{}}
\toprule
\textbf{Période} & \textbf{Caractéristique} & \textbf{RMSE trad.} & \textbf{RMSE DL} & \textbf{Amélioration} \\
\midrule
Jan-Mar 2020 & Crise COVID onset & 0.0421 & 0.0367 & -12.8\% \\
Apr-Dec 2020 & Haute volatilité & 0.0339 & 0.0294 & -13.3\% \\
2021 & Marché stable & 0.0213 & 0.0189 & -11.3\% \\
2022 & Inflation concerns & 0.0298 & 0.0261 & -12.4\% \\
\bottomrule
\end{tabular}
\end{table}

La consistance des améliorations à travers différents régimes de marché démontre la robustesse de l'approche neuronale et sa capacité d'adaptation aux conditions changeantes.

\section{Analyse des sensibilités et grecques}

\subsection{Précision des grecques}

Un aspect crucial pour l'application pratique concerne la précision du calcul des sensibilités (grecques). Nous avons évalué la capacité de notre approche à reproduire fidèlement les grecques du modèle de Heston.

\begin{table}[H]
\centering
\caption{Erreurs relatives sur les grecques (échantillon test)}
\begin{tabular}{@{}lccc@{}}
\toprule
\textbf{Grecque} & \textbf{Méthode traditionnelle} & \textbf{Deep Learning} & \textbf{Différence} \\
\midrule
Delta & 0.0012 & 0.0015 & +25\% \\
Gamma & 0.0089 & 0.0094 & +5.6\% \\
Vega & 0.0156 & 0.0162 & +3.8\% \\
Theta & 0.0201 & 0.0198 & -1.5\% \\
Rho & 0.0134 & 0.0129 & -3.7\% \\
\bottomrule
\end{tabular}
\end{table}

Les résultats montrent une précision remarquable pour les grecques, avec des erreurs comparables aux méthodes traditionnelles. La légère dégradation pour Delta s'explique par la sensibilité numérique accrue de cette grecque aux approximations.

\subsection{Vitesse de calcul des sensibilités}

L'avantage computationnel s'étend au calcul des grecques. Grâce à la différentiation automatique de PyTorch, le calcul simultané des prix et sensibilités reste remarquablement efficace :

\begin{table}[H]
\centering
\caption{Temps de calcul pour 10,000 options avec grecques}
\begin{tabular}{@{}lcc@{}}
\toprule
\textbf{Méthode} & \textbf{Temps total} & \textbf{Accélération} \\
\midrule
Méthode traditionnelle & 45.3 s & - \\
Deep Learning & 4.1 s & 11.0x \\
Deep Learning (batch) & 1.2 s & 37.8x \\
\bottomrule
\end{tabular}
\end{table}

L'utilisation du processing par batch amplifie encore l'avantage computationnel, atteignant des facteurs d'accélération de près de 38x pour les calculs de portefeuille.

\section{Tests de robustesse et validation}

\subsection{Stabilité face aux paramètres extrêmes}

Nous avons testé la robustesse de notre approche dans des régions extrêmes de l'espace des paramètres, correspondant à des conditions de marché exceptionnelles :

\begin{table}[H]
\centering
\caption{Performance sur paramètres extrêmes}
\begin{tabular}{@{}lccc@{}}
\toprule
\textbf{Condition} & \textbf{RMSE trad.} & \textbf{RMSE DL} & \textbf{Taux échec trad.} \\
\midrule
$\rho < -0.8$ & 0.0456 & 0.0389 & 23\% \\
$\sigma > 0.8$ & 0.0378 & 0.0341 & 18\% \\
$\kappa < 0.2$ & 0.0412 & 0.0367 & 21\% \\
Volatilité très basse ($v_0 < 0.02$) & 0.0298 & 0.0267 & 8\% \\
\bottomrule
\end{tabular}
\end{table}

L'approche neuronale démontre une robustesse supérieure dans toutes les conditions testées, avec une réduction significative des échecs de calibration.

\subsection{Tests de Monte Carlo}

Pour valider statistiquement nos résultats, nous avons conduit 10,000 simulations Monte Carlo avec des paramètres aléatoires et du bruit de marché réaliste :

\begin{itemize}
\item Moyenne des erreurs RMSE : $2.34 \times 10^{-3}$ (DL) vs $2.89 \times 10^{-3}$ (traditionnel)
\item Écart-type des erreurs : $1.12 \times 10^{-3}$ (DL) vs $1.67 \times 10^{-3}$ (traditionnel)
\item Percentile 95\% des temps d'exécution : 3.4s (DL) vs 38.7s (traditionnel)
\end{itemize}

Ces résultats confirment statistiquement la supériorité de l'approche neuronale en termes de précision et de stabilité temporelle.

\section{Analyse économique et implications pratiques}

\subsection{Impact sur les coûts computationnels}

L'accélération d'un facteur 11x se traduit par des économies substantielles en infrastructure IT. Pour une desk de trading typique effectuant 100 calibrations par jour :

\begin{itemize}
\item Réduction du temps de calcul : de 39 minutes à 3.5 minutes par jour
\item Diminution des besoins en CPU : facteur 11x
\item Possibilité de calibrations intra-day plus fréquentes
\item Réduction des coûts d'infrastructure cloud de 89\%
\end{itemize}

\subsection{Amélioration de la gestion des risques}

La rapidité accrue permet des applications pratiques nouvelles :

\begin{enumerate}
\item Calibration en temps réel pendant les sessions de trading
\item Stress testing avec milhares de scénarios
\item Calibration conditionnelle sur événements de marché
\item Recalibration automatique déclenchée par des seuils de volatilité
\end{enumerate}

\subsection{Implications réglementaires}

L'amélioration de la précision et de la stabilité présente des avantages pour la conformité réglementaire :

\begin{itemize}
\item Réduction des erreurs de modèle pour le calcul des fonds propres
\item Amélioration de la traçabilité et reproductibilité des calibrations
\item Capacité accrue à documenter la stabilité des modèles
\item Possibilité d'analyses de sensibilité plus approfondies
\end{itemize}

\section{Limitations et perspectives d'amélioration}

\subsection{Limitations identifiées}

Malgré les performances remarquables, plusieurs limitations méritent attention :

\begin{enumerate}
\item \textbf{Dépendance aux données d'entraînement} : Les performances se dégradent pour des configurations de paramètres très éloignées de l'ensemble d'entraînement
\item \textbf{Interprétabilité limitée} : La nature "boîte noire" du réseau complique l'analyse des échecs de calibration
\item \textbf{Coût initial d'entraînement} : La génération des données et l'entraînement initial requièrent un investissement computationnel substantiel
\item \textbf{Maintenance du modèle} : Les changements de régime de marché peuvent nécessiter un réentraînement périodique
\end{enumerate}

\subsection{Voies d'amélioration}

Plusieurs axes d'amélioration ont été identifiés :

\begin{enumerate}
\item \textbf{Apprentissage adaptatif} : Implémentation de techniques d'apprentissage en ligne pour adapter le modèle aux nouvelles conditions de marché
\item \textbf{Uncertainty quantification} : Intégration de méthodes bayésiennes pour quantifier l'incertitude des prédictions
\item \textbf{Architecture améliorée} : Exploration d'architectures alternatives (Transformers, ResNets) pour capturer des patterns plus complexes
\item \textbf{Multi-objectifs} : Extension à l'optimisation simultanée de multiple critères (précision, stabilité, interprétabilité)
\end{enumerate}

\section{Conclusion}

Les résultats présentés dans ce chapitre démontrent de manière convaincante l'efficacité de l'approche de calibration par Deep Learning pour le modèle de Heston. L'accélération computationnelle d'un facteur 11x, combinée à une amélioration de la précision et de la robustesse, ouvre de nouvelles perspectives pour la gestion des risques et le pricing d'options en temps réel.

L'évaluation sur données réelles SPX Weekly confirme la viabilité pratique de cette approche, même dans des conditions de marché volatiles. La capacité à maintenir des performances supérieures à travers différents régimes de marché témoigne de la robustesse de la méthodologie.

Ces résultats positionnent l'approche de Deep Learning comme une alternative crédible et supérieure aux méthodes traditionnelles de calibration, avec des implications significatives pour l'industrie financière. L'adoption de ces techniques pourrait transformer la manière dont les institutions financières abordent la calibration de modèles et la gestion des risques de marché.

Les perspectives d'amélioration identifiées suggèrent un potentiel d'optimisation supplémentaire, promettant des développements futurs encore plus performants. L'investissement dans ces technologies représente donc un avantage concurrentiel durable pour les institutions qui sauront les maîtriser et les déployer efficacement.

\chapter{Conclusions}

\section{Synthèse des contributions principales}

Cette recherche a démontré de manière convaincante l'efficacité et la viabilité de l'application du Deep Learning à la calibration du modèle de Heston. L'implémentation de l'approche en deux étapes développée par Bayer et Stemper a produit des résultats remarquables qui transforment fondamentalement notre compréhension des possibilités de la calibration moderne de modèles de volatilité stochastique.

L'objectif principal de cette étude était d'évaluer dans quelle mesure un réseau de Deep Learning peut remplacer ou accélérer la calibration traditionnelle du modèle de Heston sur des données réelles. Les résultats obtenus apportent une réponse définitive et positive à cette question centrale. L'accélération computationnelle d'un facteur 11x, combinée à une amélioration systématique de la précision et de la robustesse, établit cette approche comme une alternative supérieure aux méthodes traditionnelles.

La méthodologie développée résout efficacement les limitations fondamentales de la calibration traditionnelle. Le remplacement des évaluations coûteuses de Monte Carlo par des évaluations rapides de réseaux de neurones élimine le goulot d'étranglement computationnel qui limitait jusqu'alors l'application pratique de calibrations fréquentes en environnement de trading.

\section{Implications théoriques et méthodologiques}

Cette recherche contribue significativement à l'avancement de la théorie de la calibration de modèles financiers. L'approche en deux étapes proposée par Bayer et Stemper, validée empiriquement dans cette étude, représente un paradigme nouveau qui réconcilie l'efficacité du Deep Learning avec la robustesse des algorithmes d'optimisation établis.

La démonstration que les réseaux de neurones peuvent approximer fidèlement la fonction de mapping complexe des paramètres de Heston vers les volatilités implicites ouvre des perspectives considérables pour l'application de techniques similaires à d'autres modèles de finance quantitative. Cette universalité méthodologique suggère que l'approche pourrait être étendue aux modèles SABR, aux modèles de volatilité locale stochastique, et potentiellement aux modèles de volatilité rugueuse.

L'analyse de robustesse révèle que l'approche neuronale présente une stabilité supérieure aux méthodes traditionnelles, particulièrement dans les régions extrêmes de l'espace des paramètres. Cette caractéristique revêt une importance critique pour les applications de gestion des risques, où la fiabilité du modèle dans des conditions de marché exceptionnelles constitue un prérequis fondamental.

\section{Impact pratique pour l'industrie financière}

Les implications pratiques de cette recherche s'étendent bien au-delà de la simple accélération computationnelle. La possibilité de calibrer le modèle de Heston en temps quasi-réel transforme les approches possibles de gestion des risques et de pricing d'options.

L'industrie financière bénéficiera immédiatement de plusieurs avantages opérationnels. La réduction drastique des temps de calcul permet l'implémentation de calibrations intra-day, améliorant significativement la réactivité aux changements de conditions de marché. Les institutions financières peuvent désormais envisager des stratégies de gestion des risques plus dynamiques, avec des recalibrations automatiques déclenchées par des événements de marché spécifiques.

L'amélioration de la précision et de la stabilité présente des bénéfices directs pour la conformité réglementaire. Les erreurs de modèle réduites se traduisent par des calculs de fonds propres plus précis, tandis que la stabilité accrue facilite la documentation et la validation des modèles internes requis par la réglementation Bâle III.

L'aspect économique mérite une attention particulière. La réduction d'un facteur 11x des besoins computationnels se traduit par des économies substantielles en infrastructure IT. Pour les institutions utilisant des services cloud pour leurs calculs de risque, cette réduction représente des économies opérationnelles immédiates de près de 90\% sur les coûts de calcul.

\section{Contributions à la littérature académique}

Cette étude enrichit la littérature académique sur plusieurs dimensions importantes. Premièrement, elle fournit une validation empirique rigoureuse de l'approche théorique développée par Bayer et Stemper, comblant un gap important entre les développements méthodologiques et leur application pratique.

L'analyse comparative exhaustive avec les méthodes traditionnelles établit des benchmarks de référence pour les recherches futures. Les métriques de performance détaillées et les analyses de robustesse constituent une contribution méthodologique valuable pour les chercheurs travaillant sur des problèmes similaires.

La validation sur données de marché réelles, couvrant différents régimes de volatilité incluant la crise COVID-19, apporte une dimension empirique cruciale souvent absente des études purement théoriques. Cette validation démontre la pertinence pratique de l'approche dans des conditions de marché réelles et volatiles.

L'extension de l'analyse aux grecques et sensibilités comble une lacune importante dans la littérature existante. La démonstration que l'approche neuronale maintient une précision élevée pour le calcul des sensibilités valide son applicabilité pour les besoins complets de gestion des risques.

\section{Limitations et perspectives critiques}

Malgré les résultats remarquables obtenus, cette recherche présente certaines limitations qu'il convient de reconnaître et d'adresser dans les développements futurs.

La dépendance aux données d'entraînement constitue une limitation fondamentale de l'approche neuronale. Les performances se dégradent pour des configurations de paramètres très éloignées de l'ensemble d'entraînement, créant un risque de model breakdown dans des conditions de marché exceptionnelles non anticipées lors de l'entraînement.

L'aspect "boîte noire" des réseaux de neurones pose des défis pour l'interprétabilité et la validation réglementaire. Les institutions financières doivent développer des frameworks appropriés pour expliquer et justifier les décisions basées sur des modèles neuronaux, particulièrement dans le contexte des exigences réglementaires de transparence.

Le coût initial d'implémentation ne doit pas être sous-estimé. La génération des données d'entraînement, l'entraînement des modèles et le développement de l'infrastructure technique requièrent un investissement substantiel en ressources humaines et technologiques.

La maintenance et l'évolution des modèles neuronaux présentent des défis spécifiques. Les changements de régime de marché peuvent nécessiter des réentraînements périodiques, créant des coûts opérationnels récurrents et des risques de discontinuité de service.

\section{Recommandations pour l'implémentation pratique}

Basé sur les résultats de cette recherche, plusieurs recommandations émergent pour les institutions souhaitant implémenter cette approche.

L'adoption progressive constitue la stratégie recommandée. Les institutions devraient commencer par des applications pilotes sur des portefeuilles limités, permettant de valider l'approche et de développer l'expertise nécessaire avant un déploiement à grande échelle.

L'investissement dans des systèmes de monitoring et de validation est crucial. Les institutions doivent développer des capacités de surveillance continue des performances des modèles neuronaux, avec des mécanismes automatiques de détection de dégradation des performances.

La formation du personnel technique et de gestion des risques représente un facteur critique de succès. L'implémentation effective nécessite une compréhension approfondie des principes du Deep Learning et de leurs implications pour la gestion des risques financiers.

Le développement de frameworks de gouvernance adaptés est essentiel. Les institutions doivent établir des procédures claires pour la validation, la documentation et la maintenance des modèles neuronaux, en conformité avec les exigences réglementaires.

\section{Perspectives de recherche future}

Cette recherche ouvre plusieurs directions prometteuses pour les développements futurs.

L'extension à d'autres modèles de volatilité stochastique constitue une voie naturelle d'expansion. L'application de méthodologies similaires aux modèles SABR, Bates ou aux modèles de volatilité rugueuse pourrait généraliser l'approche à l'ensemble des modèles utilisés en pratique.

Le développement de techniques d'apprentissage adaptatif présente un potentiel considérable. L'implémentation d'approches d'apprentissage en ligne permettrait aux modèles de s'adapter automatiquement aux changements de conditions de marché, réduisant les besoins de réentraînement manuel.

L'intégration de l'uncertainty quantification représente une direction de recherche importante. Le développement d'approches bayésiennes pour quantifier l'incertitude des prédictions neuronales améliorerait significativement la robustesse et l'interprétabilité des modèles.

L'exploration d'architectures neuronales avancées offre des perspectives d'amélioration. L'application des Transformers, des Graph Neural Networks ou des Physics-Informed Neural Networks pourrait capturer des patterns plus complexes dans la dynamique de calibration.

Le développement d'approches multi-objectifs constitue une extension naturelle. L'optimisation simultanée de critères multiples incluant la précision, la stabilité, l'interprétabilité et la vitesse pourrait conduire à des solutions plus équilibrées pour les applications pratiques.

\section{Impact à long terme et transformation de l'industrie}

Les implications à long terme de cette recherche s'étendent au-delà de la calibration de modèles spécifiques. Elle s'inscrit dans une transformation plus large de l'industrie financière vers une digitalisation accrue et une adoption généralisée de l'intelligence artificielle.

L'accélération de la calibration de modèles représente un exemple emblématique de la manière dont le Deep Learning peut transformer les processus fondamentaux de la finance quantitative. Cette transformation ouvre la voie à des applications plus ambitieuses, incluant l'optimisation de portefeuilles en temps réel, la gestion dynamique des risques et le pricing adaptatif de produits complexes.

Les institutions qui maîtriseront ces technologies bénéficieront d'avantages concurrentiels durables. La capacité à calibrer et recalibrer les modèles rapidement et précisément devient un facteur différenciant crucial dans un environnement de marché de plus en plus volatil et compétitif.

Cette évolution s'accompagne nécessairement de changements organisationnels profonds. Les équipes de gestion des risques doivent intégrer des compétences en science des données et Machine Learning, tandis que les processus de validation et de gouvernance doivent évoluer pour accommoder les spécificités des modèles d'apprentissage automatique.

\section{Conclusion générale}

Cette recherche démontre de manière convaincante que l'application du Deep Learning à la calibration du modèle de Heston représente une avancée majeure qui transforme fondamentalement les possibilités de la finance quantitative moderne. L'accélération computationnelle spectaculaire, combinée à une amélioration de la précision et de la robustesse, établit cette approche comme la nouvelle référence pour la calibration de modèles de volatilité stochastique.

Les résultats obtenus répondent définitivement à la question de recherche initiale : un réseau de Deep Learning peut non seulement remplacer la calibration traditionnelle du modèle de Heston, mais la surpasser sur tous les critères de performance pertinents. Cette conclusion revêt une importance considérable pour l'industrie financière, ouvrant la voie à des applications pratiques qui étaient jusqu'alors impossible en raison des contraintes computationnelles.

L'impact de cette recherche s'étend bien au-delà de la contribution technique immédiate. Elle illustre le potentiel transformateur du Deep Learning pour résoudre des problèmes fondamentaux de la finance quantitative, encourageant des développements similaires dans d'autres domaines de la modélisation financière.

Les institutions financières qui adopteront ces nouvelles approches bénéficieront d'avantages compétitifs substantiels, leur permettant de réagir plus rapidement aux changements de marché et de gérer les risques avec une précision accrue. Cette transformation technologique représente donc un enjeu stratégique majeur pour l'avenir de l'industrie financière.

Finalement, cette recherche contribue à établir les fondements d'une nouvelle génération de modèles financiers qui combinent la rigueur théorique de la finance quantitative avec la puissance computationnelle de l'intelligence artificielle moderne. Cette convergence promet de révolutionner la manière dont nous abordons la modélisation, la calibration et la gestion des risques financiers dans les années à venir.

L'avenir de la finance quantitative sera probablement caractérisé par une intégration croissante de ces technologies, transformant des processus qui semblaient figés depuis des décennies. Cette recherche contribue à tracer la voie de cette transformation, démontrant qu'il est possible de concilier innovation technologique et rigueur scientifique pour créer des solutions supérieures aux approches traditionnelles.

Le message principal de cette étude est clair : l'adoption du Deep Learning pour la calibration de modèles financiers n'est plus une question de recherche académique, mais une nécessité pratique pour les institutions qui souhaitent maintenir leur compétitivité dans un environnement technologique en évolution rapide. Les résultats présentés fournissent la validation empirique nécessaire pour encourager cette adoption et ouvrent la voie à des développements encore plus ambitieux dans le futur.


% Bibliographie
\newpage
\bibliographystyle{apalike}
\bibliography{references}
\addcontentsline{toc}{chapter}{Bibliographie}

% Annexes
\newpage
\appendix
\chapter{Annexes}

\section{Détails techniques de l'implémentation}

\subsection{Architecture détaillée du réseau de neurones}

Cette section présente les spécifications techniques complètes de l'architecture du réseau de neurones développée pour l'approximation de la fonction de pricing du modèle de Heston.

\subsubsection{Configuration des couches}

\begin{table}[H]
\centering
\caption{Architecture détaillée du réseau de neurones}
\begin{tabular}{@{}llccc@{}}
\toprule
\textbf{Couche} & \textbf{Type} & \textbf{Taille} & \textbf{Activation} & \textbf{Régularisation} \\
\midrule
Input & Dense & 7 & - & - \\
Hidden 1 & Dense & 256 & ReLU & Dropout (0.2) \\
BatchNorm 1 & Batch Normalization & 256 & - & - \\
Hidden 2 & Dense & 512 & ReLU & Dropout (0.2) \\
BatchNorm 2 & Batch Normalization & 512 & - & - \\
Hidden 3 & Dense & 256 & ReLU & Dropout (0.1) \\
BatchNorm 3 & Batch Normalization & 256 & - & - \\
Hidden 4 & Dense & 128 & ReLU & Dropout (0.1) \\
Output & Dense & 1 & Linear & - \\
\bottomrule
\end{tabular}
\end{table}

\subsubsection{Paramètres d'entraînement}

\begin{table}[H]
\centering
\caption{Hyperparamètres d'entraînement}
\begin{tabular}{@{}lc@{}}
\toprule
\textbf{Paramètre} & \textbf{Valeur} \\
\midrule
Learning Rate initial & 0.001 \\
Scheduler & ReduceLROnPlateau \\
Patience & 10 \\
Factor de réduction & 0.5 \\
Optimizer & Adam \\
Beta1 & 0.9 \\
Beta2 & 0.999 \\
Weight Decay & 1e-5 \\
Batch Size & 1024 \\
Épochs maximum & 1000 \\
Early Stopping patience & 20 \\
\bottomrule
\end{tabular}
\end{table}

\subsection{Algorithme de calibration hybride}

\subsubsection{Pseudocode de l'algorithme principal}

\begin{algorithm}[H]
\caption{Calibration hybride Heston-Deep Learning}
\begin{algorithmic}
\STATE \textbf{Input:} Surface de volatilité observée $\Sigma_{market}$, réseau pré-entraîné $\mathcal{N}_\phi$
\STATE \textbf{Output:} Paramètres calibrés $\hat{\theta} = (\hat{\kappa}, \hat{\theta}, \hat{\sigma}, \hat{\rho}, \hat{v_0})$

\STATE // Initialisation des paramètres
\STATE $\theta_0 \leftarrow$ initialisation_smart($\Sigma_{market}$)
\STATE $bounds \leftarrow$ contraintes_heston()

\STATE // Définition de la fonction objectif
\FUNCTION{objective}{$\theta$}
    \STATE $\Sigma_{model} \leftarrow \mathcal{N}_\phi(\theta, M, T)$ // Évaluation rapide via réseau
    \STATE $error \leftarrow \sum_{i,j} w_{i,j} (\Sigma_{market}[i,j] - \Sigma_{model}[i,j])^2$
    \RETURN $error$
\ENDFUNCTION

\STATE // Optimisation par Levenberg-Marquardt
\STATE $\hat{\theta} \leftarrow$ LevenbergMarquardt(objective, $\theta_0$, bounds)

\STATE // Validation optionnelle avec pricing exact
\IF{validation\_requise}
    \STATE $\Sigma_{exact} \leftarrow$ HestonPricing($\hat{\theta}$, M, T)
    \STATE $validation\_error \leftarrow$ compute\_error($\Sigma_{exact}$, $\Sigma_{market}$)
    \IF{$validation\_error > threshold$}
        \STATE $\hat{\theta} \leftarrow$ fallback\_traditional\_calibration($\Sigma_{market}$)
    \ENDIF
\ENDIF

\RETURN $\hat{\theta}$
\end{algorithmic}
\end{algorithm}

\subsubsection{Fonction d'initialisation intelligente}

\begin{algorithm}[H]
\caption{Initialisation intelligente des paramètres}
\begin{algorithmic}
\FUNCTION{initialisation\_smart}{$\Sigma_{market}$}
    \STATE // Estimation de la volatilité moyenne
    \STATE $\bar{\sigma} \leftarrow$ moyenne($\Sigma_{market}$)
    \STATE $v_0 \leftarrow \bar{\sigma}^2$
    
    \STATE // Estimation du niveau long terme
    \STATE $\theta \leftarrow$ volatilité\_long\_terme($\Sigma_{market}$)
    
    \STATE // Estimation de la vitesse de retour
    \STATE $\kappa \leftarrow$ 2.0  // valeur conservatrice
    
    \STATE // Estimation de la corrélation par le skew
    \STATE $skew \leftarrow$ compute\_skew($\Sigma_{market}$)
    \STATE $\rho \leftarrow -0.3 \times \tanh(skew \times 2)$
    
    \STATE // Estimation de la vol de vol
    \STATE $\sigma \leftarrow$ 0.3  // valeur typique de marché
    
    \RETURN $(\kappa, \theta, \sigma, \rho, v_0)$
\ENDFUNCTION
\end{algorithmic}
\end{algorithm}

\section{Données et preprocessing}

\subsection{Structure des données SPX Weekly}

\subsubsection{Caractéristiques du dataset}

\begin{table}[H]
\centering
\caption{Statistiques descriptives des données SPX Weekly}
\begin{tabular}{@{}lcccc@{}}
\toprule
\textbf{Variable} & \textbf{Minimum} & \textbf{Maximum} & \textbf{Moyenne} & \textbf{Écart-type} \\
\midrule
Moneyness & 0.75 & 1.25 & 1.002 & 0.089 \\
Time to Maturity (jours) & 1 & 60 & 18.7 & 12.4 \\
Volatilité Implicite & 0.08 & 0.95 & 0.234 & 0.087 \\
Prix sous-jacent & 2,191 & 4,793 & 3,412 & 634 \\
VIX contemporain & 12.4 & 82.7 & 24.1 & 9.8 \\
\bottomrule
\end{tabular}
\end{table}

\subsubsection{Distribution temporelle des observations}

\begin{table}[H]
\centering
\caption{Répartition des observations par période}
\begin{tabular}{@{}lccc@{}}
\toprule
\textbf{Période} & \textbf{Nombre d'obs.} & \textbf{VIX moyen} & \textbf{Caractéristiques} \\
\midrule
Jan-Mar 2020 & 23,456 & 31.2 & Début crise COVID \\
Apr-Dec 2020 & 87,234 & 28.7 & Volatilité élevée soutenue \\
2021 & 78,912 & 19.4 & Retour à la normale \\
2022 & 89,567 & 26.8 & Tensions inflationnistes \\
\bottomrule
\end{tabular}
\end{table}

\subsection{Procédures de nettoyage des données}

\subsubsection{Filtres de qualité appliqués}

\begin{enumerate}
\item \textbf{Filtre de liquidité} : Élimination des options avec bid-ask spread > 10\% du mid-price
\item \textbf{Filtre d'arbitrage} : Vérification de la monotonicité par strike et maturité
\item \textbf{Filtre de volatilité} : Exclusion des IV < 5\% ou > 100\%
\item \textbf{Filtre de maturité} : Conservation uniquement des maturités 1-60 jours
\item \textbf{Filtre de moneyness} : Restriction à la plage [0.75, 1.25]
\end{enumerate}

\subsubsection{Traitement des valeurs aberrantes}

\begin{algorithm}[H]
\caption{Détection et traitement des outliers}
\begin{algorithmic}
\FUNCTION{clean\_outliers}{$data$, $threshold = 3.0$}
    \FOR{chaque date $t$}
        \STATE $surface_t \leftarrow$ extraire\_surface($data$, $t$)
        
        \STATE // Détection par z-score
        \STATE $z\_scores \leftarrow$ compute\_zscore($surface_t$)
        \STATE $outliers \leftarrow$ find\_outliers($z\_scores$, $threshold$)
        
        \STATE // Traitement par interpolation
        \FOR{chaque outlier $(m_i, T_j)$}
            \STATE $IV_{i,j} \leftarrow$ interpolate\_bivariate($surface_t$, $m_i$, $T_j$)
        \ENDFOR
        
        \STATE // Validation d'arbitrage
        \STATE valider\_absence\_arbitrage($surface_t$)
    \ENDFOR
    
    \RETURN $data_{clean}$
\ENDFUNCTION
\end{algorithmic}
\end{algorithm}

\section{Métriques de performance détaillées}

\subsection{Définitions des métriques}

\subsubsection{Erreurs de pricing}

\begin{align}
RMSE_{IV} &= \sqrt{\frac{1}{N} \sum_{i=1}^{N} (IV_i^{market} - IV_i^{model})^2} \\
MAE_{IV} &= \frac{1}{N} \sum_{i=1}^{N} |IV_i^{market} - IV_i^{model}| \\
MAPE_{IV} &= \frac{1}{N} \sum_{i=1}^{N} \frac{|IV_i^{market} - IV_i^{model}|}{IV_i^{market}} \times 100\%
\end{align}

\subsubsection{Erreurs de paramètres}

\begin{align}
RMSE_{\theta} &= \sqrt{\frac{1}{5} \sum_{j=1}^{5} \left(\frac{\theta_j^{true} - \theta_j^{est}}{\theta_j^{true}}\right)^2} \\
Bias_j &= \frac{1}{N} \sum_{i=1}^{N} (\theta_{j,i}^{est} - \theta_{j,i}^{true})
\end{align}

\subsection{Tests statistiques de validation}

\subsubsection{Test de Diebold-Mariano}

Le test de Diebold-Mariano compare la précision prédictive de deux méthodes de calibration :

\begin{align}
DM &= \frac{\bar{d}}{\sqrt{Var(\bar{d})}} \\
\bar{d} &= \frac{1}{N} \sum_{i=1}^{N} d_i \\
d_i &= L(e_{1,i}) - L(e_{2,i})
\end{align}

où $L(\cdot)$ est une fonction de perte et $e_{j,i}$ les erreurs de la méthode $j$ pour l'observation $i$.

\subsubsection{Résultats des tests statistiques}

\begin{table}[H]
\centering
\caption{Tests de significativité des améliorations}
\begin{tabular}{@{}lccc@{}}
\toprule
\textbf{Test} & \textbf{Statistique} & \textbf{p-value} & \textbf{Conclusion} \\
\midrule
Diebold-Mariano (RMSE) & -4.23 & < 0.001 & DL significativement meilleur \\
Diebold-Mariano (MAE) & -3.87 & < 0.001 & DL significativement meilleur \\
Wilcoxon signed-rank & -8,234 & < 0.001 & DL significativement meilleur \\
Test de Kolmogorov-Smirnov & 0.089 & < 0.001 & Distributions différentes \\
\bottomrule
\end{tabular}
\end{table}

\section{Analyse de sensibilité}

\subsection{Sensibilité aux hyperparamètres}

\subsubsection{Impact de la taille du réseau}

\begin{table}[H]
\centering
\caption{Performance vs taille du réseau}
\begin{tabular}{@{}lcccc@{}}
\toprule
\textbf{Architecture} & \textbf{Paramètres} & \textbf{RMSE} & \textbf{Temps training} & \textbf{Temps inference} \\
\midrule
[128, 256, 128] & 98K & $1.8 \times 10^{-3}$ & 45 min & 0.08 ms \\
[256, 512, 256] & 394K & $1.3 \times 10^{-3}$ & 78 min & 0.15 ms \\
[512, 1024, 512] & 1.5M & $1.2 \times 10^{-3}$ & 156 min & 0.31 ms \\
[256, 512, 256, 128] & 527K & $1.2 \times 10^{-3}$ & 89 min & 0.18 ms \\
\bottomrule
\end{tabular}
\end{table}

\subsubsection{Impact du learning rate}

\begin{table}[H]
\centering
\caption{Sensibilité au learning rate}
\begin{tabular}{@{}lccc@{}}
\toprule
\textbf{Learning Rate} & \textbf{Convergence} & \textbf{RMSE final} & \textbf{Épochs nécessaires} \\
\midrule
0.01 & Instable & - & - \\
0.005 & Oscillations & $2.1 \times 10^{-3}$ & 1000+ \\
0.001 & Stable & $1.3 \times 10^{-3}$ & 347 \\
0.0005 & Lente & $1.2 \times 10^{-3}$ & 876 \\
0.0001 & Très lente & $1.4 \times 10^{-3}$ & 1000+ \\
\bottomrule
\end{tabular}
\end{table}

\subsection{Robustesse aux conditions de marché}

\subsubsection{Performance par quintile de volatilité}

\begin{table}[H]
\centering
\caption{Performance par niveau de volatilité (VIX)}
\begin{tabular}{@{}lcccc@{}}
\toprule
\textbf{Quintile VIX} & \textbf{Plage VIX} & \textbf{RMSE trad.} & \textbf{RMSE DL} & \textbf{Amélioration} \\
\midrule
Q1 (faible) & [12.4, 17.8] & 0.0189 & 0.0165 & -12.7\% \\
Q2 & [17.8, 21.3] & 0.0223 & 0.0194 & -13.0\% \\
Q3 & [21.3, 26.1] & 0.0267 & 0.0231 & -13.5\% \\
Q4 & [26.1, 34.7] & 0.0334 & 0.0289 & -13.5\% \\
Q5 (élevé) & [34.7, 82.7] & 0.0445 & 0.0387 & -13.0\% \\
\bottomrule
\end{tabular}
\end{table}

\section{Code source des fonctions principales}

\subsection{Fonction de pricing Heston}

\begin{lstlisting}[language=Python, caption=Implémentation du pricing Heston]
import numpy as np
from scipy.integrate import quad
from scipy.optimize import brentq

def heston_characteristic_function(phi, S0, v0, kappa, theta, sigma, rho, T, r):
    """
    Fonction caractéristique du modèle de Heston
    """
    xi = kappa - 1j * rho * sigma * phi
    d = np.sqrt(xi**2 + sigma**2 * (1j * phi + phi**2))
    
    A1 = 1j * phi * (np.log(S0) + r * T)
    A2 = (kappa * theta) / (sigma**2) * ((xi - d) * T - 2 * np.log((1 - g * np.exp(-d * T)) / (1 - g)))
    A3 = (v0 / sigma**2) * (xi - d) * (1 - np.exp(-d * T)) / (1 - g * np.exp(-d * T))
    
    g = (xi - d) / (xi + d)
    
    return np.exp(A1 + A2 + A3)

def heston_price_fft(S0, K, T, r, v0, kappa, theta, sigma, rho, option_type='call'):
    """
    Pricing d'option européenne sous Heston par FFT
    """
    N = 4096  # Nombre de points pour la FFT
    alpha = 1.5  # Paramètre de régularisation
    eta = 0.25   # Pas de discrétisation
    
    # Grille de log-strikes
    lambda_val = 2 * np.pi / (N * eta)
    b = lambda_val * N / 2
    ks = -b + lambda_val * np.arange(N)
    
    # Fonction intégrande modifiée
    def integrand(v, k):
        phi = v - 1j * (alpha + 1)
        cf = heston_characteristic_function(phi, S0, v0, kappa, theta, sigma, rho, T, r)
        return np.real(np.exp(-1j * v * k) * cf) / (alpha**2 + alpha - v**2 + 1j * (2 * alpha + 1) * v)
    
    # Calcul des prix par FFT
    integrand_values = np.array([quad(lambda v: integrand(v, k), 0, 100)[0] for k in ks])
    
    fft_values = np.fft.fft(integrand_values)
    option_values = np.exp(-alpha * ks) / np.pi * np.real(fft_values)
    
    # Interpolation pour obtenir le prix au strike désiré
    log_K = np.log(K)
    return np.interp(log_K, ks, option_values) * np.exp(-r * T)

def heston_implied_volatility(S0, K, T, r, v0, kappa, theta, sigma, rho, option_type='call'):
    """
    Calcul de la volatilité implicite sous Heston
    """
    price = heston_price_fft(S0, K, T, r, v0, kappa, theta, sigma, rho, option_type)
    
    def objective(vol):
        bs_price = black_scholes_price(S0, K, T, r, vol, option_type)
        return bs_price - price
    
    try:
        iv = brentq(objective, 0.001, 3.0)
        return iv
    except:
        return np.nan
\end{lstlisting}

\subsection{Classe du réseau de neurones}

\begin{lstlisting}[language=Python, caption=Architecture du réseau de neurones]
import torch
import torch.nn as nn
import torch.nn.functional as F

class HestonPricingNetwork(nn.Module):
    def __init__(self, input_dim=7, hidden_dims=[256, 512, 256, 128], dropout_rates=[0.2, 0.2, 0.1, 0.1]):
        super(HestonPricingNetwork, self).__init__()
        
        # Construction des couches
        self.layers = nn.ModuleList()
        self.batch_norms = nn.ModuleList()
        self.dropouts = nn.ModuleList()
        
        dims = [input_dim] + hidden_dims
        
        for i in range(len(dims) - 1):
            self.layers.append(nn.Linear(dims[i], dims[i+1]))
            self.batch_norms.append(nn.BatchNorm1d(dims[i+1]))
            self.dropouts.append(nn.Dropout(dropout_rates[i]))
        
        # Couche de sortie
        self.output_layer = nn.Linear(hidden_dims[-1], 1)
        
        # Initialisation des poids
        self._initialize_weights()
    
    def _initialize_weights(self):
        for layer in self.layers:
            nn.init.xavier_uniform_(layer.weight)
            nn.init.constant_(layer.bias, 0)
        
        nn.init.xavier_uniform_(self.output_layer.weight)
        nn.init.constant_(self.output_layer.bias, 0)
    
    def forward(self, x):
        for i, (layer, bn, dropout) in enumerate(zip(self.layers, self.batch_norms, self.dropouts)):
            x = layer(x)
            x = bn(x)
            x = F.relu(x)
            x = dropout(x)
        
        x = self.output_layer(x)
        return torch.sigmoid(x)  # Contrainte de positivité pour IV

class HestonCalibrator:
    def __init__(self, model_path):
        self.device = torch.device('cuda' if torch.cuda.is_available() else 'cpu')
        self.network = HestonPricingNetwork()
        self.network.load_state_dict(torch.load(model_path, map_location=self.device))
        self.network.eval()
        self.network.to(self.device)
    
    def predict_iv_surface(self, params, moneyness_grid, maturity_grid):
        """
        Prédiction de la surface de volatilité implicite
        """
        kappa, theta, sigma, rho, v0 = params
        
        # Création de la grille d'entrée
        M, T = np.meshgrid(moneyness_grid, maturity_grid)
        inputs = []
        
        for i in range(len(maturity_grid)):
            for j in range(len(moneyness_grid)):
                inputs.append([kappa, theta, sigma, rho, v0, M[i,j], T[i,j]])
        
        inputs = torch.tensor(inputs, dtype=torch.float32).to(self.device)
        
        with torch.no_grad():
            predictions = self.network(inputs)
        
        iv_surface = predictions.cpu().numpy().reshape(len(maturity_grid), len(moneyness_grid))
        return iv_surface
    
    def calibrate(self, market_iv_surface, moneyness_grid, maturity_grid, 
                  initial_guess=None, bounds=None):
        """
        Calibration des paramètres de Heston
        """
        if initial_guess is None:
            initial_guess = [2.0, 0.04, 0.3, -0.7, 0.04]
        
        if bounds is None:
            bounds = [(0.1, 5.0), (0.01, 0.5), (0.05, 1.0), (-0.9, 0.1), (0.01, 0.5)]
        
        def objective(params):
            pred_surface = self.predict_iv_surface(params, moneyness_grid, maturity_grid)
            error = np.mean((market_iv_surface - pred_surface)**2)
            return error
        
        from scipy.optimize import minimize
        result = minimize(objective, initial_guess, bounds=bounds, method='L-BFGS-B')
        
        return result.x, result.fun, result.success
\end{lstlisting}

\section{Résultats détaillés des expériences}

\subsection{Matrices de confusion pour la validation}

\begin{table}[H]
\centering
\caption{Classification des succès de calibration par méthode}
\begin{tabular}{@{}lcccc@{}}
\toprule
\textbf{Méthode} & \textbf{Succès} & \textbf{Échecs} & \textbf{Timeouts} & \textbf{Taux succès} \\
\midrule
Levenberg-Marquardt traditionnel & 873 & 89 & 38 & 87.3\% \\
Deep Learning hybride & 984 & 12 & 4 & 98.4\% \\
\bottomrule
\end{tabular}
\end{table}

\subsection{Distribution des temps d'exécution}

\begin{table}[H]
\centering
\caption{Percentiles des temps d'exécution (secondes)}
\begin{tabular}{@{}lcccccc@{}}
\toprule
\textbf{Méthode} & \textbf{P5} & \textbf{P25} & \textbf{P50} & \textbf{P75} & \textbf{P95} & \textbf{P99} \\
\midrule
Traditionnel & 8.2 & 15.7 & 23.4 & 31.2 & 45.8 & 67.3 \\
Deep Learning & 0.9 & 1.6 & 2.1 & 2.8 & 4.1 & 6.2 \\
\bottomrule
\end{tabular}
\end{table}

Cette annexe fournit tous les détails techniques nécessaires à la reproduction et à l'extension de nos résultats. L'implémentation complète est disponible dans le repository GitHub accompagnant ce mémoire, permettant une validation indépendante de nos conclusions.


\end{document}
