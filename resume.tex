Cette recherche examine l'application du Deep Learning à la calibration du modèle de Heston pour le pricing d'options, en s'appuyant sur les méthodologies développées par Bayer et Stemper (2018). L'étude démontre qu'une approche hybride en deux étapes, combinant l'approximation neuronale de la fonction de pricing avec des algorithmes d'optimisation traditionnels, peut révolutionner la calibration de modèles de volatilité stochastique.

La méthodologie proposée remplace les évaluations coûteuses de Monte Carlo par des prédictions rapides d'un réseau de neurones entraîné pour approximer la fonction de mapping des paramètres de Heston vers les volatilités implicites. Cette approximation neuronale est ensuite intégrée dans un algorithme d'optimisation de Levenberg-Marquardt pour la calibration effective des paramètres du modèle.

L'évaluation empirique s'appuie sur un dataset complet d'options SPX Weekly couvrant la période 2020-2022, incluant les conditions de marché exceptionnelles liées à la crise COVID-19. Cette période d'étude offre une diversité remarquable de régimes de volatilité, depuis les niveaux extrêmes observés en mars 2020 (VIX à 82.7) jusqu'aux conditions plus normalisées de 2021.

Les résultats obtenus révèlent une accélération computationnelle spectaculaire d'un facteur 11, réduisant le temps de calibration moyen de 23.4 secondes à 2.1 secondes par surface de volatilité. Cette amélioration s'accompagne d'une augmentation systématique de la précision, avec une réduction de 12.5\% de l'erreur quadratique moyenne sur les volatilités implicites et une amélioration de 13.6\% de l'erreur absolue moyenne. Le taux de convergence atteint 98\% contre 87\% pour les méthodes traditionnelles, démontrant une robustesse supérieure face aux difficultés d'optimisation.

L'analyse de robustesse confirme la supériorité de l'approche neuronale à travers différents régimes de marché et conditions de volatilité. Les tests sur données synthétiques valident la capacité de récupération des paramètres, tandis que la validation sur données réelles démontre la viabilité pratique de l'approche dans des conditions de trading réalistes.

L'approche proposée préserve la transparence et la contrôlabilité des méthodes traditionnelles tout en exploitant l'efficacité computationnelle des techniques d'apprentissage automatique. Cette caractéristique facilite l'intégration dans les systèmes existants de gestion des risques et répond aux exigences de validation réglementaire.

Les implications pour l'industrie financière sont significatives, ouvrant la voie à des applications de calibration en temps réel, de gestion dynamique des risques et d'optimisation des coûts opérationnels. La réduction drastique des temps de calcul permet l'implémentation de calibrations intra-day et de stress testing avec des milliers de scénarios, transformant ainsi les possibilités de gestion des risques.

Cette recherche contribue à établir les fondements d'une nouvelle génération d'outils de finance quantitative qui combinent la rigueur théorique traditionnelle avec la puissance computationnelle de l'intelligence artificielle moderne. Les résultats démontrent qu'il est non seulement possible mais avantageux d'adopter ces nouvelles approches pour la calibration de modèles financiers complexes.

\textbf{Mots-clés :} Modèle de Heston, Deep Learning, Calibration de modèles, Options SPX Weekly, Volatilité stochastique, Finance quantitative, Réseaux de neurones, Optimisation hybride
