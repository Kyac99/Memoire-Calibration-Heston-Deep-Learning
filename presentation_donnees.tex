\chapter{Présentation des données}

\section{Introduction}

La qualité et la représentativité des données constituent les fondements de toute étude empirique rigoureuse en finance quantitative. Ce chapitre présente de manière exhaustive le dataset d'options SPX Weekly utilisé dans cette recherche, en détaillant ses caractéristiques, les procédures de traitement appliquées, et les analyses exploratoires qui motivent nos choix méthodologiques.

Les options SPX Weekly, lancées par le Chicago Board Options Exchange (CBOE) en 2005, représentent aujourd'hui l'un des instruments dérivés les plus liquides au monde. Leur particularité réside dans leurs échéances très courtes (1 à 60 jours typiquement), offrant aux traders et gestionnaires de risques une granularité temporelle inégalée pour l'implémentation de stratégies sophistiquées et la gestion dynamique des expositions.

\section{Caractéristiques du marché des options SPX Weekly}

\subsection{Contexte historique et évolution}

Les options SPX Weekly ont révolutionné le paysage du trading d'options depuis leur introduction. Contrairement aux options mensuelles traditionnelles, ces instruments expirent chaque vendredi (sauf jours fériés), créant un cycle de liquidité unique qui attire particulièrement les traders institutionnels et les market makers.

L'évolution du volume de trading illustre cette popularité croissante. En 2010, les options Weekly représentaient moins de 5\% du volume total d'options SPX. En 2020, cette proportion avait dépassé 70\%, témoignant d'un changement structural fondamental dans les préférences des participants de marché.

Cette transformation s'explique par plusieurs facteurs :
\begin{itemize}
\item \textbf{Flexibilité temporelle} : Les échéances courtes permettent des stratégies de trading plus précises temporellement
\item \textbf{Decay temporel accéléré} : La décroissance de la valeur temps s'accélère à l'approche de l'échéance, créant des opportunités uniques
\item \textbf{Volatilité implicite élevée} : Les options à très court terme présentent généralement des niveaux de volatilité implicite supérieurs
\item \textbf{Liquidité concentrée} : Les market makers concentrent leurs efforts sur ces instruments, améliorant les conditions d'exécution
\end{itemize}

\subsection{Spécificités techniques}

Les options SPX Weekly présentent des caractéristiques techniques distinctives qui influencent directement les défis de calibration :

\subsubsection{Structure des échéances}

Contrairement aux options mensuelles avec des échéances standardisées, les options Weekly créent un continuum d'échéances. À tout moment, le marché offre typiquement :
\begin{itemize}
\item 4-5 échéances Weekly (vendredis suivants)
\item 2-3 échéances mensuelles (troisièmes vendredis des mois suivants)
\item 2-4 échéances trimestrielles (mars, juin, septembre, décembre)
\end{itemize}

Cette richesse d'échéances améliore significativement la précision de calibration des modèles de volatilité, particulièrement pour la capture des effets de terme.

\subsubsection{Grille de strikes}

La grille de strikes des options SPX Weekly s'adapte dynamiquement aux niveaux de l'indice. Les règles typiques incluent :
\begin{itemize}
\item Espacement de 5 points pour les strikes proches de la monnaie (±10\%)
\item Espacement de 10 points pour les strikes intermédiaires (±20\%)
\item Espacement de 25 points pour les strikes extrêmes (au-delà de ±20\%)
\end{itemize}

Cette granularité permet une capture précise du smile de volatilité, particulièrement crucial pour les applications de calibration de modèles.

\section{Dataset et période d'étude}

\subsection{Source et fournisseur de données}

Les données utilisées dans cette étude proviennent de [Nom du fournisseur], une source institutionnelle reconnue pour la qualité et l'exhaustivité de ses données d'options. Le dataset comprend :

\begin{itemize}
\item \textbf{Prix de clôture quotidiens} : Bid, Ask, et Last pour chaque option
\item \textbf{Volumes et open interest} : Indicateurs de liquidité et d'activité
\item \textbf{Volatilités implicites} : Calculées selon le modèle de Black-Scholes
\item \textbf{Grecques} : Delta, Gamma, Vega, Theta calculées par le fournisseur
\item \textbf{Métadonnées} : Spécifications contractuelles et ajustements éventuels
\end{itemize}

\subsection{Période d'observation}

La période d'étude s'étend du 1er janvier 2020 au 31 décembre 2022, soit 36 mois de données comprenant 782 jours de trading. Cette période a été choisie pour sa richesse en termes de conditions de marché variées :

\subsubsection{Phase 1 : Choc initial COVID-19 (Janvier-Mars 2020)}

Cette période capture l'émergence et l'impact initial de la pandémie COVID-19 sur les marchés financiers. Caractéristiques observées :
\begin{itemize}
\item VIX maximum à 82.69 le 16 mars 2020
\item Volatilité réalisée annualisée dépassant 80\%
\item Dislocations significatives dans la structure par terme de volatilité
\item Asymétrie extrême du smile de volatilité (skew jusqu'à -30\%)
\end{itemize}

\subsubsection{Phase 2 : Adaptation et volatilité soutenue (Avril-Décembre 2020)}

Période d'adaptation aux nouvelles conditions économiques avec :
\begin{itemize}
\item VIX oscillant entre 20 et 40
\item Interventions massives des banques centrales
\item Changements structurels dans les corrélations inter-actifs
\item Émergence de patterns de volatilité inédits
\end{itemize}

\subsubsection{Phase 3 : Normalisation progressive (2021)}

Retour graduel vers des conditions de marché plus conventionnelles :
\begin{itemize}
\item VIX moyen de 19.4 (vs moyenne historique de 16.8)
\item Reprise de la croissance économique
\item Politique monétaire accommodante maintenue
\item Rotation sectorielle importante
\end{itemize}

\subsubsection{Phase 4 : Tensions inflationnistes (2022)}

Nouvelle période de volatilité liée aux préoccupations inflationnistes :
\begin{itemize}
\item Resserrement de la politique monétaire
\item Guerre en Ukraine et chocs géopolitiques
\item VIX moyen de 26.8
\item Inversion de la courbe des taux
\end{itemize}

\subsection{Statistiques descriptives du dataset brut}

Le dataset brut comprend initialement 2,847,392 observations d'options réparties comme suit :

\begin{table}[H]
\centering
\caption{Répartition des observations par année}
\begin{tabular}{@{}lcccc@{}}
\toprule
\textbf{Année} & \textbf{Observations} & \textbf{Options Weekly} & \textbf{VIX moyen} & \textbf{Volume quotidien} \\
\midrule
2020 & 987,456 & 78.3\% & 29.8 & 1,234,567 \\
2021 & 912,234 & 81.2\% & 19.4 & 1,567,890 \\
2022 & 947,702 & 83.7\% & 26.8 & 1,789,234 \\
\textbf{Total} & \textbf{2,847,392} & \textbf{81.1\%} & \textbf{25.3} & \textbf{1,530,564} \\
\bottomrule
\end{tabular}
\end{table}

\section{Procédures de nettoyage et filtrage}

\subsection{Critères de qualité des données}

La qualité des données d'options revêt une importance cruciale pour la précision des calibrations ultérieures. Notre protocole de nettoyage applique une série de filtres rigoureux développés en s'inspirant des meilleures pratiques académiques et industrielles.

\subsubsection{Filtre de liquidité}

Le premier filtre élimine les options illiquides susceptibles de présenter des prix peu représentatifs :

\begin{itemize}
\item \textbf{Spread bid-ask} : Élimination si spread > 10\% du prix mid
\item \textbf{Volume minimum} : Conservation uniquement si volume ≥ 10 contrats
\item \textbf{Open interest} : Exclusion si open interest < 100 contrats
\item \textbf{Prix minimum} : Élimination si prix < \$0.05
\end{itemize}

Ce filtre élimine 23.7\% des observations initiales, principalement des options très out-of-the-money ou proches de l'expiration.

\subsubsection{Filtre d'arbitrage}

Le second filtre vérifie l'absence d'opportunités d'arbitrage évidentes :

\begin{enumerate}
\item \textbf{Monotonicité par strike} : Pour des options de même maturité, vérification que :
   \begin{itemize}
   \item $C(K_1) \geq C(K_2)$ si $K_1 < K_2$ (calls)
   \item $P(K_1) \leq P(K_2)$ si $K_1 < K_2$ (puts)
   \end{itemize}

\item \textbf{Convexité} : Vérification de la convexité des prix par rapport au strike

\item \textbf{Parité call-put} : Validation de $C - P = S - Ke^{-rT}$ avec tolérance de ±2\%

\item \textbf{Bornes intrinsèques} : Vérification que $C \geq \max(S - Ke^{-rT}, 0)$
\end{enumerate}

Ce filtre supprime 8.3\% d'observations supplémentaires.

\subsubsection{Filtre de volatilité implicite}

Le troisième filtre porte sur la plausibilité des volatilités implicites :

\begin{itemize}
\item \textbf{Bornes absolues} : Conservation uniquement si $\sigma_{IV} \in [0.02, 2.00]$
\item \textbf{Convergence numérique} : Élimination si l'algorithme d'inversion Black-Scholes ne converge pas
\item \textbf{Stabilité temporelle} : Exclusion des volatilités présentant des variations > 50\% jour-à-jour sans justification
\end{itemize}

\subsubsection{Filtre de maturité et moneyness}

Le dernier filtre restreint l'échantillon aux caractéristiques d'intérêt :

\begin{itemize}
\item \textbf{Maturité} : Conservation uniquement si $T \in [1/365, 60/365]$ années
\item \textbf{Moneyness} : Restriction à $m = K/S \in [0.75, 1.25]$
\item \textbf{Jours à l'expiration} : Exclusion des options expirant le jour même (T=0)
\end{itemize}

\subsection{Impact du processus de filtrage}

L'application séquentielle de ces filtres réduit le dataset de 2,847,392 à 1,567,823 observations, soit un taux de rétention de 55.1\%. Cette réduction substantielle reflète la rigueur du processus de nettoyage et garantit la qualité des données utilisées pour la calibration.

\begin{table}[H]
\centering
\caption{Impact séquentiel des filtres de qualité}
\begin{tabular}{@{}lccc@{}}
\toprule
\textbf{Étape} & \textbf{Observations restantes} & \textbf{Taux de rétention} & \textbf{Observations supprimées} \\
\midrule
Dataset initial & 2,847,392 & 100.0\% & - \\
Après filtre liquidité & 2,174,086 & 76.3\% & 673,306 \\
Après filtre arbitrage & 1,993,567 & 70.0\% & 180,519 \\
Après filtre volatilité & 1,842,234 & 64.7\% & 151,333 \\
Après filtre final & 1,567,823 & 55.1\% & 274,411 \\
\bottomrule
\end{tabular}
</table>

\section{Analyse exploratoire des données}

\subsection{Distribution des caractéristiques}

L'analyse exploratoire révèle des patterns intéressants dans la distribution des données nettoyées.

\subsubsection{Distribution des maturités}

La répartition des maturités montre une concentration marquée sur les échéances très courtes :

\begin{table}[H]
\centering
\caption{Distribution des maturités (jours calendaires)}
\begin{tabular}{@{}lccc@{}}
\toprule
\textbf{Plage de maturité} & \textbf{Observations} & \textbf{Pourcentage} & \textbf{Liquidité moyenne} \\
\midrule
1-7 jours & 534,123 & 34.1\% & Très élevée \\
8-14 jours & 412,567 & 26.3\% & Élevée \\
15-30 jours & 387,234 & 24.7\% & Élevée \\
31-45 jours & 178,456 & 11.4\% & Modérée \\
46-60 jours & 55,443 & 3.5\% & Faible \\
\bottomrule
\end{tabular}
\end{table}

Cette concentration sur les maturités courtes reflète les préférences des traders pour les instruments à décroissance temporelle rapide.

\subsubsection{Distribution de la moneyness}

L'analyse de la moneyness révèle une activité concentrée autour de la monnaie :

\begin{table}[H]
\centering
\caption{Distribution de la moneyness}
\begin{tabular}{@{}lccc@{}}
\toprule
\textbf{Région} & \textbf{Plage moneyness} & \textbf{Observations} & \textbf{Volume moyen} \\
\midrule
Deep OTM Put & [0.75, 0.85] & 145,233 & 156 \\
OTM Put & [0.85, 0.95] & 298,567 & 342 \\
Near the money & [0.95, 1.05] & 634,789 & 1,234 \\
OTM Call & [1.05, 1.15] & 287,234 & 298 \\
Deep OTM Call & [1.15, 1.25] & 202,000 & 123 \\
\bottomrule
\end{tabular}
\end{table}

\subsection{Évolution temporelle de la volatilité implicite}

\subsubsection{Analyse de la structure par terme}

L'évolution de la structure par terme de volatilité implicite fournit des insights cruciaux sur les anticipations de marché :

\begin{table}[H]
\centering
\caption{Volatilité implicite moyenne par maturité et période}
\begin{tabular}{@{}lcccc@{}}
\toprule
\textbf{Maturité} & \textbf{2020 Q1} & \textbf{2020 Q2-Q4} & \textbf{2021} & \textbf{2022} \\
\midrule
1-7 jours & 45.8\% & 28.3\% & 18.2\% & 24.7\% \\
8-14 jours & 42.1\% & 26.7\% & 17.8\% & 23.9\% \\
15-30 jours & 38.9\% & 25.1\% & 17.3\% & 23.1\% \\
31-45 jours & 36.2\% & 23.8\% & 16.9\% & 22.6\% \\
46-60 jours & 34.7\% & 22.9\% & 16.5\% & 22.1\% \\
\bottomrule
\end{tabular}
\end{table}

Ces données révèlent plusieurs patterns importants :
\begin{itemize}
\item \textbf{Terme structure inversée} en période de crise (2020 Q1)
\item \textbf{Normalisation progressive} de la structure par terme
\item \textbf{Prime de volatilité courte} persistante à travers tous les régimes
\end{itemize}

\subsubsection{Analyse du smile de volatilité}

L'asymétrie du smile de volatilité varie significativement selon les conditions de marché :

\begin{table}[H]
\centering
\caption{Skew de volatilité par période (différence OTM Put - ATM)}
\begin{tabular}{@{}lcccc@{}}
\toprule
\textbf{Maturité} & \textbf{2020 Q1} & \textbf{2020 Q2-Q4} & \textbf{2021} & \textbf{2022} \\
\midrule
1-7 jours & -28.3\% & -15.7\% & -8.9\% & -12.4\% \\
8-14 jours & -23.1\% & -13.2\% & -7.8\% & -10.9\% \\
15-30 jours & -18.7\% & -11.4\% & -6.9\% & -9.7\% \\
31-45 jours & -15.9\% & -9.8\% & -6.2\% & -8.8\% \\
46-60 jours & -14.2\% & -8.9\% & -5.7\% & -8.1\% \\
\bottomrule
\end{tabular}
\end{table}

\subsection{Corrélations et dépendances}

\subsubsection{Corrélation avec les indices de marché}

L'analyse des corrélations entre volatilités implicites et variables de marché révèle :

\begin{table}[H]
\centering
\caption{Corrélations avec variables de marché}
\begin{tabular}{@{}lcccc@{}}
\toprule
\textbf{Variable} & \textbf{IV Court terme} & \textbf{IV Moyen terme} & \textbf{Skew} & \textbf{Smile} \\
\midrule
Niveau SPX & -0.42 & -0.38 & 0.31 & -0.18 \\
VIX & 0.89 & 0.82 & 0.67 & 0.23 \\
Rendement 10Y & -0.23 & -0.19 & 0.15 & -0.09 \\
Dollar Index & 0.18 & 0.15 & -0.12 & 0.07 \\
Credit Spreads & 0.34 & 0.31 & 0.28 & 0.12 \\
\bottomrule
\end{tabular}
\end{table>

\subsubsection{Persistance temporelle}

L'analyse de l'autocorrélation révèle des patterns de persistance différentiés :

\begin{itemize}
\item \textbf{Volatilité implicite ATM} : Autocorrélation de 0.85 à 1 jour, 0.34 à 5 jours
\item \textbf{Skew} : Persistance plus élevée (0.92 à 1 jour, 0.58 à 5 jours)
\item \textbf{Structure par terme} : Très stable (0.95 à 1 jour, 0.78 à 5 jours)
\end{itemize}

\section{Patterns spécifiques aux options Weekly}

\subsection{Effet d'expiration}

Les options Weekly présentent des patterns d'expiration distincts des options mensuelles :

\subsubsection{Décroissance de la valeur temps}

L'analyse de la décroissance de theta révèle une accélération non-linéaire marquée :

\begin{table}[H]
\centering
\caption{Theta moyen par jours à l'expiration}
\begin{tabular}{@{}lcccc@{}}
\toprule
\textbf{Jours à expiration} & \textbf{ATM Calls} & \textbf{ATM Puts} & \textbf{OTM Calls} & \textbf{OTM Puts} \\
\midrule
5-7 jours & -0.042 & -0.041 & -0.018 & -0.019 \\
3-4 jours & -0.067 & -0.065 & -0.028 & -0.031 \\
2 jours & -0.098 & -0.094 & -0.039 & -0.043 \\
1 jour & -0.156 & -0.152 & -0.061 & -0.067 \\
\bottomrule
\end{tabular}
\end{table>

\subsubsection{Compression du smile}

À l'approche de l'expiration, le smile de volatilité présente une compression caractéristique :

\begin{itemize}
\item \textbf{Réduction du skew} : Diminution de 40\% dans les 3 derniers jours
\item \textbf{Aplatissement du smile} : Convergence vers la volatilité réalisée
\item \textbf{Instabilité accrue} : Variance des volatilités implicites multipliée par 3
\end{itemize}

\subsection{Phénomènes de fin de semaine}

\subsubsection{Effet weekend}

Les options expirant le vendredi présentent des caractéristiques uniques :

\begin{table}[H]
\centering
\caption{Primes de volatilité selon le jour d'expiration}
\begin{tabular}{@{}lccc@{}}
\toprule
\textbf{Jour d'expiration} & \textbf{Prime ATM} & \textbf{Prime OTM} & \textbf{Volume relatif} \\
\midrule
Lundi & +2.1\% & +3.4\% & 0.8x \\
Mardi & +1.8\% & +2.9\% & 0.9x \\
Mercredi & +1.2\% & +1.8\% & 1.1x \\
Jeudi & +0.8\% & +1.2\% & 1.3x \\
Vendredi & Référence & Référence & 1.0x \\
\bottomrule
\end{tabular}
\end{table>

\section{Implications pour la calibration}

\subsection{Défis spécifiques identifiés}

L'analyse exploratoire révèle plusieurs défis spécifiques pour la calibration du modèle de Heston sur données SPX Weekly :

\subsubsection{Granularité temporelle}

La concentration sur les maturités très courtes (< 30 jours) pose des défis particuliers :
\begin{itemize}
\item \textbf{Sensibilité accrue au paramètre $v_0$} (variance initiale)
\item \textbf{Instabilité numérique} des calculs d'intégrales pour $T \rightarrow 0$
\item \textbf{Dominance des effets de microstructure} pour les maturités < 7 jours
\end{itemize}

\subsubsection{Régimes de volatilité extrêmes}

Les périodes de crise révèlent des limitations des modèles classiques :
\begin{itemize}
\item \textbf{Non-stationnarité} des paramètres du modèle de Heston
\item \textbf{Breakdown du modèle} pour des volatilités > 100\%
\item \textbf{Asymétries extrêmes} non capturable par la corrélation $\rho$ seule
\end{itemize}

\subsection{Opportunities pour l'approche Deep Learning}

Ces défis créent simultanément des opportunities pour l'approche par apprentissage automatique :

\subsubsection{Adaptabilité aux régimes}

Les réseaux de neurones peuvent naturellement s'adapter aux différents régimes de marché sans nécessiter de recalibration manuelle des paramètres.

\subsubsection{Capture des non-linéarités}

L'architecture neuronale peut capturer des relations complexes entre paramètres, maturités et moneyness que les approches analytiques peinent à modéliser.

\subsubsection{Robustesse aux données extrêmes}

L'entraînement sur un large spectre de conditions de marché (incluant la crise COVID-19) devrait conférer une robustesse supérieure aux approches traditionnelles.

\section{Préparation pour la méthodologie}

\subsection{Division de l'échantillon}

Pour les besoins de l'étude, le dataset nettoyé est divisé selon une approche temporelle :

\begin{table}[H]
\centering
\caption{Division temporelle du dataset}
\begin{tabular}{@{}lccc@{}}
\toprule
\textbf{Période} & \textbf{Usage} & \textbf{Observations} & \textbf{Pourcentage} \\
\midrule
Jan 2020 - Jun 2022 & Entraînement/Validation & 1,254,258 & 80.0\% \\
Jul 2022 - Dec 2022 & Test out-of-sample & 313,565 & 20.0\% \\
\bottomrule
\end{tabular}
\end{table}

Cette division préserve la dimension temporelle cruciale pour valider la capacité de généralisation des modèles.

\subsection{Normalisation et preprocessing}

Les données subissent une normalisation sophistiquée préparant l'entraînement neuronal :

\begin{itemize}
\item \textbf{Volatilités implicites} : Transformation logit pour borner dans [0,1]
\item \textbf{Maturités} : Transformation logarithmique pour gérer la non-linéarité
\item \textbf{Moneyness} : Centrage et réduction par la volatilité ATM
\item \textbf{Variables de marché} : Standardisation z-score sur la période d'entraînement
\end{itemize}

\section{Conclusion}

Ce chapitre a présenté de manière exhaustive les caractéristiques du dataset d'options SPX Weekly utilisé dans cette recherche. L'analyse exploratoire révèle un dataset riche et représentatif, couvrant une gamme exceptionnellement large de conditions de marché.

Les spécificités identifiées - concentration sur les maturités courtes, patterns d'expiration unique, sensibilité aux régimes de volatilité - motivent l'adoption d'approches de calibration innovantes. Les défis révélés par l'analyse traditionnelle créent simultanément les opportunities que l'approche Deep Learning propose d'exploiter.

La qualité rigoureuse du processus de nettoyage, combinée à la richesse temporelle du dataset, fournit les fondations solides nécessaires pour l'implémentation et la validation de notre méthodologie de calibration hybride, présentée dans le chapitre suivant.

Les patterns observés - de la persistance du skew à l'effet d'expiration des options Weekly - constituent autant de régularités que notre réseau de neurones devra apprendre à reproduire fidèlement. La validation de cette capacité d'apprentissage constitue l'un des enjeux centraux de notre recherche, dont les résultats sont présentés dans les chapitres ultérieurs.
